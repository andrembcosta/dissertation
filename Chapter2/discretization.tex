\section{Discretization}
\label{section: Chapter2/discretization}

We begin this section by summarizing the strong form of the general initial boundary value problem of interest:
\begin{mdframed}[
  frametitle={The initial boundary value problem for the time interval $t \in I = [t_0, t_f]$},
  frametitlebackgroundcolor=gray!20,
  backgroundcolor=gray!5,
  linewidth=0pt,
  nobreak=true
  ]
  \vspace{-1em}
  \begin{align*}
    \text{Linear momentum balance: } & \rho_0 \bta = \divergence \bfP + \rho_0 \btb,                       &  & \body_0 \times ]t_0, t[,                          \\
                                     & \bfP \normal_0 = \btt,                                              &  & \partial_t\body_0 \times ]t_0, t[,                \\
                                     & \btu = \btu_g,                                                      &  & \partial_u\body_0 \times ]t_0, t[,                \\
    \text{Fracture evolution: }      & \divergence \bs{\xi} - f = 0,                                       &  & \mathcal{I}_t(\body_0) \times ]t_0, t[,           \\
                                     & \bs{\xi}\cdot\normal_0 = 0,                                         &  & \mathcal{I}_t(\partial_d\body_0) \times ]t_0, t[, \\
    \text{Heat transfer: }           & \rho_0c_v\dot{T} = \rho_0 q - \divergence \bth + \delta + \delta_T, &  & \body_0 \times ]t_0, t[,                          \\
                                     & \bth \cdot \normal_0 = \bar{h}_n,                                   &  & \partial_h\body_0 \times ]t_0, t[,                \\
                                     & \bth \cdot \normal_0 = h(T - T_0),                                  &  & \partial_r\body_0 \times ]t_0, t[,                \\
                                     & T = T_g,                                                            &  & \partial_T\body_0 \times ]t_0, t[,                \\
    \text{Initial conditions: }      & \btu = \btu_0,                                                      &  & \body_0, t = t_0,                                 \\
                                     & \dot{\btu} = \btv_0,                                                &  & \body_0, t = t_0,                                 \\
                                     & d = d_0,                                                            &  & \body_0, t = t_0,                                 \\
                                     & T = \bar{T}_0,                                                      &  & \body_0, t = t_0.                                 
  \end{align*}
\end{mdframed}
The yield surface and the flow rule (and their corresponding constitutive updates) are defined only after specific constitutive assumptions are made in the following chapters (e.g. \Cref{section: Chapter5}).

To derive the weak form, we begin by introducing the trial spaces $\bs{\mathcal{U}}_t$, $\mathcal{D}_t$ and $\mathcal{T}_t$:
\begin{subequations}
  \begin{align}
    \bs{\mathcal{U}}_t & = \{ \btu \in \mathcal{H}^1(\body_0)^d \mid \bfu = \bfu_g \text{ on } \partial_u\body_0 \}, \\
    \mathcal{D}_t      & = \{ d \in \mathcal{H}^1(\body_0) \mid \dot{d} > 0 \text{ on } \mathcal{I}_t(\body_0) \},   \\
    \mathcal{T}_t      & = \{ T \in \mathcal{H}^1(\body_0) \mid T = T_g \text{ on } \partial_T\body_0 \},            
  \end{align}
\end{subequations}
and the weighting spaces $\bs{\mathcal{V}}$, $\mathcal{C}$ and $\mathcal{E}$:
\begin{subequations}
  \begin{align}
    \bs{\mathcal{V}} & = \{ \btw \in \mathcal{H}^1(\body_0)^d \mid \btw = \bs{0} \text{ on } \partial_u\body_0 \}, \\
    \mathcal{C}      & = \{ c \in \mathcal{H}^1(\body_0) \},                                                       \\
    \mathcal{E}      & = \{ e \in \mathcal{H}^1(\body_0) \mid T = 0 \text{ on } \partial_T\body_0 \}.              
  \end{align}
\end{subequations}
The weak form can be derived as:
\begin{mdframed}[
    frametitle={The weak form},
    frametitlebackgroundcolor=gray!20,
    backgroundcolor=gray!5,
    linewidth=0pt,
    nobreak=true
  ]
  Given parameters in the IBVP, find $\btu \in \bs{\mathcal{U}}_t$, $d \in \mathcal{D}_t$ and $T \in \mathcal{T}_t$, $t \in [t_0, t_f]$, such that $\forall \btw \in \bs{\mathcal{V}}$, $\forall c \in \mathcal{C}$ and $\forall e \in \mathcal{E}$,
  \begin{subequations}
    \begin{align}
      \left( \btw, \rho_0\bta \right) + \left( \grad \btw, \bfP \right) - \left( \btw, \rho_0 \btb \right) - \left< \btw, \btt \right>_{\partial_t\body_0} & = 0, \\
      \left( \grad c, \bs{\xi} \right) + \left( c, f \right)                                                                                               & = 0, \\
      \begin{split}
        - \left( e, \rho_0c_v\dot{T} \right) + \left( e, \rho_0q \right) + \left( \grad e, \bth \right) + \left( e, \delta+\delta_T \right)& \\
        - \left< e, \bar{h}_n \right>_{\partial_h\body_0} - \left< e, h(T-T_0) \right>_{\partial_r\body_0} &= 0,
      \end{split}
    \end{align}
  \end{subequations}
  with projections of initial conditions
  \begin{subequations}
    \begin{align}
      \left( \btw, \btu - \btu_0 \right)       & = 0, \\
      \left( \btw, \dot{\btu} - \btv_0 \right) & = 0, \\
      \left( c, d - d_0 \right)                & = 0, \\
      \left( e, T - \bar{T}_0 \right)          & = 0. 
    \end{align}
  \end{subequations}
\end{mdframed}
Using the Galerkin method, with finite dimensional function spaces $\widetilde{\bs{\mathcal{U}}}_t \subset \bs{\mathcal{U}}_t$, $\widetilde{\bs{\mathcal{V}}} \subset \bs{\mathcal{V}}$, $\widetilde{\mathcal{D}}_t \subset \mathcal{D}_t$, $\widetilde{\mathcal{C}} \subset \mathcal{C}$, $\widetilde{\mathcal{T}}_t \subset \mathcal{T}_t$, $\widetilde{\mathcal{E}} \subset \mathcal{E}$, we arrive at the spatially discrete form of the problem:
\begin{mdframed}[
    frametitle={The semidiscrete Galerkin form},
    frametitlebackgroundcolor=gray!20,
    backgroundcolor=gray!5,
    linewidth=0pt,
    nobreak=true
  ]
  Given parameters in the IBVP, find $\btu^h \in \widetilde{\bs{\mathcal{U}}}_t$, $d^h \in \widetilde{\mathcal{D}}_t$ and $T^h \in \widetilde{\mathcal{T}}_t$, $t \in [t_0, t_f]$, such that $\forall \btw^h \in \widetilde{\bs{\mathcal{V}}}$, $\forall q^h \in \widetilde{\mathcal{C}}$ and $\forall e^h \in \widetilde{\mathcal{E}}$,
  \begin{subequations}
    \begin{align}
      \left( \btw^h, \rho_0\bta \right) + \left( \grad \btw^h, \bfP \right) - \left( \btw^h, \rho_0 \btb \right) - \left< \btw^h, \btt \right>_{\partial_t\body_0} & = 0, \label{eq: semidiscrete momentum balance}   \\
      \left( \grad c^h, \bs{\xi} \right) + \left( c^h, f \right)                                                                                                   & = 0, \label{eq: semidiscrete fracture evolution} \\
      \begin{split}
        - \left( e^h, \rho_0c_v\dot{T} \right) + \left( e^h, \rho_0q \right) + \left( \grad e^h, \bth \right) + \left( e^h, \delta+\delta_T \right)& \\
        - \left< e^h, \bar{h}_n \right>_{\partial_h\body_0} - \left< e^h, h(T^h-T_0) \right>_{\partial_r\body_0} &= 0,
      \end{split} \label{eq: semidiscrete heat conduction}
    \end{align}
  \end{subequations}
  with projections of initial conditions
  \begin{subequations}
    \begin{align}
      \left( \btw^h, \btu^h - \btu_0 \right)       & = 0, \\
      \left( \btw^h, \dot{\btu^h} - \btv_0 \right) & = 0, \\
      \left( c^h, d^h - d_0 \right)                & = 0, \\
      \left( e^h, T^h - \bar{T}_0 \right)          & = 0. 
    \end{align}
  \end{subequations}
\end{mdframed}
The semidiscrete momentum balance equation \eqref{eq: semidiscrete momentum balance} is discretized using the generalized-$\alpha$ method. The semidiscrete fracture evolution equation \eqref{eq: semidiscrete fracture evolution} and the heat conduction equation \eqref{eq: semidiscrete heat conduction} are discretized using the backward-Euler time integration. Discretizations of the plasticity constitutive updates are discussed in \Cref{section: Chapter5}.
The discrete inequality for crack irreversibility is satisfied node-wise with a primal-dual active set (PDAS) strategy. See e.g.\  \citet{heister2015primal} for implementational details of such a solver. The solver is also generally available in numerical toolboxes, e.g.\  PETSc \cite{petsc-web-page}. The discrete approximation is calculated using a fixed-point iterative solution scheme outlined from \cite{HuGary2020}. A simple fixed-point iterative Newton-based PDAS solution algorithm is outlined in \Cref{alg: PDAS-newton}. The algorithm is observed to be effective for all numerical examples included in this dissertation. Note that notation is slightly abused in the algorithm for simplicity: all state variables with subscripts or superscripts represent their discrete counterpart; the time step is indexed by the counter $n$ in the subscript; the fixed-point iteration is indexed by the counter $a$ in the subscript; the Newton iteration is indexed by the counter $k$ in the superscript; the constant mass matrix is denoted by $M$; and $c$ is the numerical parameter controlling the update of the active set.

\begin{algorithm}[!htb]
  \small
  \caption{A fixed-point iterative PDAS-Newton solution algorithm. }
  \begin{algorithmic}[1]
    \State Given $\btu_n$, $T_n$, $d_n$, ${\defgrad^p}_n$, $\ep_n$.
    \State Set $a \leftarrow 0$.
    \State Set $\btu_{n+1,a} \leftarrow \btu_n$, $T_{n+1,a} \leftarrow T_n$, $d_{n+1,a} \leftarrow d_n$, ${\defgrad^p}_{n+1,a} \leftarrow {\defgrad^p}_n$, $\ep_{n+1,a} \leftarrow \ep_n$.
    \Repeat
    \State Compute $\dot{d}$ using $d_{n+1,a}$ and $d_n$ with the appropriate time integrator.
    \State Solve for $\btu_{n+1,a+1}$, ${\defgrad^p}_{n+1,a+1}$, $\ep_{n+1,a+1}$, and $T_{n+1,a+1}$ using the discretized incremental variational (sub-) problem
    \begin{align*}
      \inf_{\btu_{n+1,a+1}} \sup_{T_{n+1,a+1}} L_n\left( \inf_{{\defgrad^p}_{n+1,a+1}, \ep_{n+1,a+1}} \varphi_n \right).
    \end{align*}
    \State Compute $\dot{\btu}$ using $\btu_{n+1,a+1}$ and $\btu_n$ with the appropriate time integrator.
    \State Set $k \leftarrow 0$.
    \State Set $d_{n+1,a+1}^k \leftarrow d_{n+1,a}$.
    \State Assemble the gradient $g$ and the hessian $H$ for the discretized incremental variational (sub-) problem
    \begin{align*}
      \inf_{d_{n+1,a+1}^k} L_n.
    \end{align*}
    \State Compute the initial Lagrange multiplier $\lambda = -M^{-1}g$.
    \State Identify the initial active/inactive as $\mathcal{A} = \{ i \vert \lambda_i > 0 \}$, $\mathcal{I} = \{ i \vert \lambda_i \leqslant 0 \}$.
    \Repeat
    \State Eliminate dofs on $\mathcal{A}$ and $\mathcal{I}$ to obtain the reduced gradient $\widehat{g}$ and reduced Hessian $\widehat{\widehat{H}}$.
    \State Compute the Newton update on the inactive set as $\widehat{p} = - \widehat{\widehat{H}}^{-1} \widehat{g}$.
    \State Update the solution on the inactive set to obtain $d_{n+1,a+1}^{k+1}$.
    \State Update $g$ and $H$.
    \State Update $\lambda = -M^{-1}(g+Hp)$.
    \State Update $\mathcal{A} = \{ i \vert (\lambda-cp)_i > 0 \}$, $\mathcal{I} = \{ i \vert (\lambda-cp)_i \leqslant 0 \}$.
    \State Update $k \leftarrow k+1$.
    \Until {$d_{n+1,a+1}$ and $\mathcal{A}$ have converged.}
    \State Update $d_{n+1,a+1} \leftarrow d_{n+1,a+1}^k$.
    \State Update $a \leftarrow a+1$.
    \Until {$\btu_{n+1}$, $T_{n+1}$ and $d_{n+1}$ have converged.}
    \State Update $\btu_{n+1} \leftarrow \btu_{n+1,a}$, $T_{n+1} \leftarrow T_{n+1,a}$, $d_{n+1} \leftarrow d_{n+1,a}$.
  \end{algorithmic}
  \label{alg: PDAS-newton}
\end{algorithm}
