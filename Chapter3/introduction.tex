\section{Introduction}

A wide range of approaches for model-based simulations of hydraulic fracturing have been developed over the past several decades \cite{adachi2007computer, lecampion2018numerical}.  These range from production-level reservoir modeling tools such as  ResFrac\cite{mcclure2017three, mcclure2018resfrac}, to GEOS\cite{settgast2012simulation, settgast2014simulation, settgast2017fully}, PyFrac\cite{zia2020pyfrac}, and others.  Many of the models and associated codes assume fracture networks that remain planar, but in recent years strides have been made towards modeling cracks that evolve in arbitrary ways in response to fluid-driven loads. High-resolution models for complex fracture evolution generally fall into two categories: sharp interface models that explicitly model the fracture surface, and diffuse crack approaches that effectively smear the geometry over the underlying grid or mesh.  Techniques that represent the crack as a sharp interface can be advantageous when the fracture configuration is relatively simple, but representing complex geometric evolution can be challenging  \cite{gupta2014simulation, gupta2018coupled, shauer2022three}.  By contrast, diffuse crack models offer more flexibility for representing complex fracture evolution, but introduce other challenges such as the lack of a well-defined fracture surface and increased computational expense\cite{heider2021review}.  In this work, we introduce a multi-resolution scheme for hydraulic fracturing simulation that makes use of both sharp and diffuse crack representations within a single framework. The objective is to establish a methodology that makes use of the advantageous aspects of both sharp and diffuse crack models while circumventing some of the drawbacks.  

Over the past several decades, the phase-field model for fracture \cite{francfort1998revisiting, bourdin2000numerical, karma2001phase} has emerged as a promising approach for constructing robust simulations of complex crack evolution.  The method has shown considerable success for simulating fracture evolution in quasi-brittle materials, and there have been several recent efforts to extend the approach to hydraulic fracturing. In what follows, we review some prior works of particular relevance to the current manuscript.  For additional references in this topic, we refer the reader to the recent review by Heider \cite{heider2021review}.  

The first attempts towards a phase-field model for hydraulic fracture began with extensions of the traditional phase-field model to pressurized cracks, as in Bourdin et al. \cite{bourdin2012variational} and Wheeler et al. \cite{wheeler2014augmented}. Subsequently, fluid flow in the fractures, and also poromechanics were considered. Miehe et al. \cite{miehe2015minimization, miehe2016phase} developed a thermodynamically consistent framework, from minimization principles, to couple poromechanics, fluid-flow and phase-field fracture. The flow problem was modeled via the Darcy's equation, containing a permeability coefficient that used the phase-field variable and the crack opening to mimic the cubic relationship from the lubrication theory in the crack region. Mikelic et al. \cite{mikelic2015phase1, mikelic2015phase2} developed a model that separated the domain into fracture and reservoir, by using the phase-field variable as an indicator function. They also considered the flow inside the fracture as a Darcy flow, but their model treated the fracture as a three-dimensional entity, which led to a different permeability tensor compared to \cite{miehe2015minimization, miehe2016phase}.  Yet another approach concerns the work of Wilson and Landis \cite{wilson2016phase}, who proposed a model that included fluid velocities as primary variables. This allowed for a more detailed description of the flow within the fracture, which was modeled by a Brinkman-type equation \cite{brinkman1949calculation}. The phase-field parameter acted as an indicator of the flow regime, between Darcy flow (away from cracks) and Stokes flow (inside cracks).  Finally, the recent work of Chukwudozie et al.\ \cite{chukwudozie2019variational} presented a different model, wherein the lubrication theory equations were included in the weak form by means of a $\Gamma$-convergent regularization. 

The use of a phase-field to represent a fracture network in a diffuse manner certainly facilitates the representation of complex geometric evolution, including crack branching and merging. However, it also requires the use of meshes or grids that are capable of resolving the regularization length, making these approaches computationally expensive. One approach to improving the efficiency of the method is the use of adaptive mesh refinement, such as in \cite{heister2015primal, lee2017iterative, Wick-adaptive-2020,Gupta-adaptive-2022}. In the specific case of hydraulic fracturing, another challenge concerns the crack opening or aperture, a field that is tightly coupled with the fluid pressure within fractures. In a phase-field setting, due to the lack of an explicit crack surface, extracting the aperture or accounting for its effects requires additional considerations.  All of the aforementioned  works present some way to account for the aperture within a diffuse setting, but the robustness of these approaches remains unclear \cite{lecampion2018numerical}. For a review of the most frequently used methods to calculate the crack aperture from phase-field simulations, see the recent work of Yoshioka et al.\ \cite{yoshioka2020crack}.

Outside of the context of hydraulic fracturing, some researchers in the phase-field community have developed ``hybrid" approaches, wherein the phase-field formulation was combined with a sharp crack representation. The motivation for these approaches varies, from ``cutting" the mesh to remove artificial traction transmission and circumvent element distortion \cite{geelen2018optimization} to reducing the overall computational cost \cite{giovanardi2017hybrid, muixi2021combined}. In the work of Giovanardi et al.\ \cite{giovanardi2017hybrid}, phase-field subproblems in the vicinity of  crack tips were used to propagate a global, discrete crack. The eXtended Finite Element Method (XFEM)\cite{moes1999finite} was used to place fracture discontinuities in the displacement field within the background global mesh. More recently, Muixi et al.\ \cite{muixi2021combined} created an approach that uses the phase-field method only at the crack tips, and XFEM in the rest of the domain. In contrast to \cite{giovanardi2017hybrid}, there is no overlap of the representations in crack tip areas.

The success of these hybrid approaches for purely mechanical cases opens the door for their extension to hydraulic fracturing.  Such approaches are appealing because in principle they can circumvent the need for a complicated reconstruction of the crack opening from the phase-field.  This area is relatively unexplored, although there have been some recent efforts that are similar in spirit, such as the recent work of Sun et al.\ \cite{sun2020hybrid}.  They developed a Finite Element-Meshfree method to represent the crack surfaces in a discrete fashion. The computed displacement field was used to obtain a driving force which was employed within a phase-field evolution equation near the crack tips. This approach eliminated the need for the reconstruction of crack openings from the diffuse crack representation, but it also largely decoupled the phase field from the equations governing the force balance near the crack tips.  

The approach presented in this work, which we refer to as a multi-resolution method, extends the concept of a hybrid phase-field method to hydraulic fracturing. It discretizes the problem at a global level using a sharp interface approach based on the Embedded Finite Element Method of  Cusini et al.\ \cite{cusini2021simulation}.  It then approaches the simulation of fracture evolution by coupling the global fields with a phase-field fracture problem posed over a subdomain in the vicinity of the crack front.  This approach has several advantages.  The crack aperture and flow inside the fracture are handled at the global scale using techniques that work well and are efficient when the crack geometry is known.  Then the phase-field model is employed in the subdomain  to effectively update the crack geometry.  This framework lends itself to incorporation with a wide range of existing hydraulic fracturing solvers, as the phase-field sub-problem is agnostic about the type of numerical treatment used in the global domain. 
In the current work, we adopt relatively simple assumptions regarding material behavior, such as small deformations and linear poroelasticity, as they allow us to verify our scheme against analytical solutions. However, in principle the approach could be extended to model crack growth in a much broader class of poroelastic materials, such as hydrogels~\cite{Oyen:jop2021}. 

The paper is structured as follows. In Section \ref{formulation}, we present the governing equations and constitutive assumptions for both hydraulic fracturing and phase-field for fracture. In Section \ref{numerics}, we propose our multi-resolution framework and present numerical schemes to discretize both the hydrofracture and the phase-field subproblems. In Section \ref{results_section}, we apply our method to study hydraulic fracturing in some simple scenarios, in order to verify its accuracy. These include the well-known KGD\cite{geertsma1969rapid, zheltov19553} problems and a case of non-planar fracture propagation around a stiff inclusion. Finally, in Section 5, we provide a summary and some concluding remarks.


