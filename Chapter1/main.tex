\chapter{Introduction}
\label{section: Chapter1}

Fracture can be observed in virtually all engineering applications across several scales. It is important to understand the behavior of materials, components and structures with the presence of fracture. To that end, model-based simulations are widely adopted to help understand fracture-related phenomena. In the past two decades, variational approaches to fracture have become increasingly popular, among which gradient damage models and phase-field models have received the most attention. On the one hand, these models have been successfully applied to a variety of engineering problems involving brittle/quasi-brittle fracture, cohesive fracture, ductile fracture as well as their coupling with other physics. On the other hand, the development of the theoretical variational framework was mostly confined within the regime of brittle/quasi-brittle fracture. This dissertation is concerned with the gap between the fundamental variational framework and practical engineering applications.

\section{Background}

%% Motivation for fluid-driven fracture

    % Shale gas extraction

    % EGS

    % Carbon storage

%% Literature review on traditional approaches (including 3D)

    % Detournay's analytical methods
    %%%%%%%%%%%%%%%%%%%%%%ADACHI PAPER

    In recent years, there has been a return to fundamental research, with significant effort being devoted to understanding the different regimes of propagation in hydraulic fracturing. The objective of this research has been to gain insight into the properties of the classical hydraulic fracture models rather than to develop new models to deal with the complex and challenging new environments in which hydraulic fractures are being developed. This work was developed on two fronts. First, an exhaustive analysis of the near-tip processes (using methods based on asymptotic theory) was undertaken, which has significantly extended the pioneering results of [40] for the zero-toughness, impermeable case; and of [41] for the zero-toughness, leak-off dominated case. This ongoing effort has accomplished not only the inclusion of toughness and fluid lag, but also the development of all the pertinent intermediate asymptotic regimes [42–44]. The second front includes the analysis of the dominant dimensionless groups that control the hydraulic fracturing process [45,46]. This work has shown that hydraulic fractures can be categorized within a parametric space, whose extremes are controlled by leak-off, toughness, or viscosity dominated processes. In general, a hydraulic fracture evolves with time within this parametric space, following trajectories that are determined by the rock properties (elastic moduli, toughness, leak-off coefficient), fluid viscosity, and injection rate. Within this framework, semi-analytical and numerical solutions have been developed for simple geometries (KGD and penny-shaped) for different asymptotic regimes, such as zero toughness impermeable [47–49]; small toughness impermeable [50]; finite toughness impermeable [51,26]; large toughness impermeable [52,49]; zero toughness permeable [53] regime; and finite toughness permeable [54] regime solutions. These solutions have not only yielded an understanding of the evolution of hydraulic fractures in and between different propagation regimes, but have also provided useful benchmarks for numerical simulators. Another important consequence of this research work is that the newly developed scaling laws can be used to define the range of parameters required to properly model the growth of a hydraulic fracture at the field or laboratory scale, or at least to properly interpret and extrapolate experimental data. For example, in most field-scale treatments, the dominant factor that controls hydraulic fracture growth is viscosity, and not toughness (the latter quantity has been historically used to assign fracture growth control in hydraulic fracturing simulators). However, in laboratory tests, where block sizes of 1 cubic foot are typical, toughness is the controlling mechanism, even when highly viscous fluids are injected. Hence, direct application of experimental results to field scale may lead to misleading conclusions. Recent experimental results [55–57] have provided physical evidence of the validity of these similarity solutions. The scaling law methodology was crucial in identifying the appropriate ranges of parameters in order to design the experiments to capture these different physical solutions.
    %%%%%%%%%%%%%%%%%%%%%%

    A wide range of approaches for model-based simulations of hydraulic fracturing have been developed over the past several decades \cite{adachi2007computer, lecampion2018numerical}.  These range from production-level reservoir modeling tools such as  ResFrac\cite{mcclure2017three, mcclure2018resfrac}, to GEOS\cite{settgast2012simulation, settgast2014simulation, settgast2017fully}, PyFrac\cite{zia2020pyfrac}, and others.  Many of the models and associated codes assume fracture networks that remain planar, but in recent years strides have been made towards modeling cracks that evolve in arbitrary ways in response to fluid-driven loads. High-resolution models for complex fracture evolution generally fall into two categories: sharp interface models that explicitly model the fracture surface, and diffuse crack approaches that effectively smear the geometry over the underlying grid or mesh.  Techniques that represent the crack as a sharp interface can be advantageous when the fracture configuration is relatively simple, but representing complex geometric evolution can be challenging  \cite{gupta2014simulation, gupta2018coupled, shauer2022three}.  By contrast, diffuse crack models offer more flexibility for representing complex fracture evolution, but introduce other challenges such as the lack of a well-defined fracture surface and increased computational expense\cite{heider2021review}.  In this work, we introduce a multi-resolution scheme for hydraulic fracturing simulation that makes use of both sharp and diffuse crack representations within a single framework. The objective is to establish a methodology that makes use of the advantageous aspects of both sharp and diffuse crack models while circumventing some of the drawbacks.  

    Any mathematical model that tries to represent the hydraulic fracturing phenomena has to account for at least three different processes. The deformation of the rock, the fluid flow and the fracture propagation. \cite{adachi2007computer}. These three processes are strongly coupled, which restricts the derivation of analytical solutions to very few problems, mostly when the fracture has a predefined shape. Some examples are the PKN model \cite{perkins1961widths, sneddon1946opening, nordgren1972propagation} and the KGD model \cite{zheltov19553, geertsma1969rapid}. Although limited by restrictive assumptions, these models were extremely useful in the early times of hydraulic fracturing, providing engineers with good estimations of treatment parameters.

    Two decades later, Simonson \cite{simonson1978containment} extended the PKN model to allow for the computation of the crack height, which is similar to the crack opening in this approach. His model was named Pseudo-3D (P3D) and involves the discretization of the crack into cells, in which PKN-like solutions are used \cite{adachi2007computer}. Therefore, it is a numerical approach, although much simpler and computationally cheaper than models based on finite elements or finite differences, which will be discussed shortly. A thorough comparison of the PKN, KGD and P3D models, which includes comparisons with experiments can be found in \cite{warpinski1994comparison}.

    Although these methods have been very useful in practice, they still are not able to capture other important phenomena that affect hydraulic fracturing, such as the presence of confining (\textit{in situ}) stresses, pre-existing (natural) fractures, fluid leak-off, the poroelastic behavior of the rocks, variations in material properties within a rock, proppant transport, arbitrary propagation direction and etc. All these issues have been demonstrated by experiments to affect the fracture propagation and therefore, computational algorithms that can address them are necessary. 

    Most attempts in this direction were done by the use of general purpose numerical methods which approximate the solution to the coupled system of partial differential equations that arise from the hydrofracture-mechanical model. For example, Boone et al. \cite{boone1990numerical, boone1991simulation} used the finite element method (FEM) to solve the coupled poroelasticity problem and the finite difference method to solve the fluid equation within the fractures, but they assumed the crack path \textit{a priori}. Carter et al. \cite{carter2000simulating} and Simoni and Secchi \cite{simoni2003cohesive} remove the assumption of given crack path and performed simulations using the FEM. In these cases, the crack path had to follow the underlying mesh. They also had to use a phenomenological criteria to compute the crack propagation direction.

    This type of criteria is also necessary when one uses more sophisticated schemes to represent the crack geometries, such as the eXtended finite element method (XFEM)\cite{moes1999finite} or enhanced strain approaches \cite{linder2007finite, borja2008assumed}. These schemes provide a way to represent cracks in arbitrary directions, independent of the discretization. They have been used by many authors in the context of hydraulic fracture, for example in \cite{lecampion2009extended, gordeliy2013coupling, gordeliy2015enrichment} and \cite{mohammadnejad2013extended}. These methods tend to become very complex in 3D, even in the case of dry fractures. Some interesting attempts in the particular case of hydraulic fracture have been tried in \cite{gupta2014simulation, gupta2018coupled}.

    % Armando's work

    % Imperial College group

    % Chinese group 

%% Phase-field method for hydraulic fracture

    % A few paragraphs about the phase-field method and its advantages

    % Bourdin and Wheeler's early work

    % Other derived works coupling poromechanics

    % Thorough comparision with tables and etc

    % Some limitations and possible improvements

    Over the past several decades, the phase-field model for fracture \cite{francfort1998revisiting, bourdin2000numerical, karma2001phase} has emerged as a promising approach for constructing robust simulations of complex crack evolution.  The method has shown considerable success for simulating fracture evolution in quasi-brittle materials, and there have been several recent efforts to extend the approach to hydraulic fracturing. In what follows, we review some prior works of particular relevance to the current manuscript.  For additional references in this topic, we refer the reader to the recent review by Heider \cite{heider2021review}.

    The first attempts towards a phase-field model for hydraulic fracture began with extensions of the traditional phase-field model to pressurized cracks, as in Bourdin et al. \cite{bourdin2012variational} and Wheeler et al. \cite{wheeler2014augmented}. Subsequently, fluid flow in the fractures, and also poromechanics were considered. Miehe et al. \cite{miehe2015minimization, miehe2016phase} developed a thermodynamically consistent framework, from minimization principles, to couple poromechanics, fluid-flow and phase-field fracture. The flow problem was modeled via the Darcy's equation, containing a permeability coefficient that used the phase-field variable and the crack opening to mimic the cubic relationship from the lubrication theory in the crack region. Mikelic et al. \cite{mikelic2015phase1, mikelic2015phase2} developed a model that separated the domain into fracture and reservoir, by using the phase-field variable as an indicator function. They also considered the flow inside the fracture as a Darcy flow, but their model treated the fracture as a three-dimensional entity, which led to a different permeability tensor compared to \cite{miehe2015minimization, miehe2016phase}.  Yet another approach concerns the work of Wilson and Landis \cite{wilson2016phase}, who proposed a model that included fluid velocities as primary variables. This allowed for a more detailed description of the flow within the fracture, which was modeled by a Brinkman-type equation \cite{brinkman1949calculation}. The phase-field parameter acted as an indicator of the flow regime, between Darcy flow (away from cracks) and Stokes flow (inside cracks).  Finally, the recent work of Chukwudozie et al.\ \cite{chukwudozie2019variational} presented a different model, wherein the lubrication theory equations were included in the weak form by means of a $\Gamma$-convergent regularization. 

    The use of a phase-field to represent a fracture network in a diffuse manner certainly facilitates the representation of complex geometric evolution, including crack branching and merging. However, it also requires the use of meshes or grids that are capable of resolving the regularization length, making these approaches computationally expensive. One approach to improving the efficiency of the method is the use of adaptive mesh refinement, such as in \cite{heister2015primal, lee2017iterative, Wick-adaptive-2020,Gupta-adaptive-2022}. In the specific case of hydraulic fracturing, another challenge concerns the crack opening or aperture, a field that is tightly coupled with the fluid pressure within fractures. In a phase-field setting, due to the lack of an explicit crack surface, extracting the aperture or accounting for its effects requires additional considerations.  All of the aforementioned  works present some way to account for the aperture within a diffuse setting, but the robustness of these approaches remains unclear \cite{lecampion2018numerical}. For a review of the most frequently used methods to calculate the crack aperture from phase-field simulations, see the recent work of Yoshioka et al.\ \cite{yoshioka2020crack}.

%% Hybrid approaches

    Outside of the context of hydraulic fracturing, some researchers in the phase-field community have developed ``hybrid" approaches, wherein the phase-field formulation was combined with a sharp crack representation. The motivation for these approaches varies, from ``cutting" the mesh to remove artificial traction transmission and circumvent element distortion \cite{geelen2018optimization} to reducing the overall computational cost \cite{giovanardi2017hybrid, muixi2021combined}. In the work of Giovanardi et al.\ \cite{giovanardi2017hybrid}, phase-field subproblems in the vicinity of  crack tips were used to propagate a global, discrete crack. The eXtended Finite Element Method (XFEM)\cite{moes1999finite} was used to place fracture discontinuities in the displacement field within the background global mesh. More recently, Muixi et al.\ \cite{muixi2021combined} created an approach that uses the phase-field method only at the crack tips, and XFEM in the rest of the domain. In contrast to \cite{giovanardi2017hybrid}, there is no overlap of the representations in crack tip areas.

    The success of these hybrid approaches for purely mechanical cases opens the door for their extension to hydraulic fracturing.  Such approaches are appealing because in principle they can circumvent the need for a complicated reconstruction of the crack opening from the phase-field.  This area is relatively unexplored, although there have been some recent efforts that are similar in spirit, such as the recent work of Sun et al.\ \cite{sun2020hybrid}.  They developed a Finite Element-Meshfree method to represent the crack surfaces in a discrete fashion. The computed displacement field was used to obtain a driving force which was employed within a phase-field evolution equation near the crack tips. This approach eliminated the need for the reconstruction of crack openings from the diffuse crack representation, but it also largely decoupled the phase field from the equations governing the force balance near the crack tips. 

    % Giovannardi including newest work from Formmagia on HF

    % Rudy's continuous discontinuous

    % Muixí's work

    % Recent work from chinese group

%% Commercial codes and intro on GEOSX

    % Check review paper from Bin Chen

    % Is there any review on HPC, GPU computing ?

%% INTRO MR PAPER


%%INTRO PRESSURE PAPER
One of these tools is the phase-field method for fracture \cite{bourdin2000numerical}. Initially developed for traction-free cracks, the method has since been extended to account for pressure loading on the surfaces of cracks, as in \cite{bourdin2012variational, wheeler2014augmented, mikelic2015quasi, peco2017influence, jiang2022phase}.  These various formulations exhibit real differences in terms of their structure and form when it comes to how the pressure loads are incorporated.  
The objective of this work is to examine the impact of the various choices, and to compare them to 
a relatively new formulation for pressurized crack surfaces in a phase-field for fracture context \cite{hu2021variationalthesis}.  
The main contributions of this work are: (a) to show that established formulations for pressure-driven fracture in the phase-field
context have limitations when cohesive processes are involved; (b) to demonstrate that the new formulation, derived from variational principles, can address these limitations and be easily combined with phase-field models of cohesive fracture; and (c) to illustrate the advantages and disadvantages of the various models in terms of accuracy in obtaining various quantities of interest.  

Phase-field methods for fracture regularize sharp crack representations through the use of a scalar phase or damage field whose evolution is governed by minimization principles.  
Such methods first appeared, in different forms, in the works of Bourdin et al. \cite{bourdin2000numerical} and Karma et al. \cite{karma2001phase}. The model introduced in Bourdin et al. \cite{bourdin2000numerical} was obtained by a regularization of the variational formulation of fracture developed in Francfort and Marigo \cite{francfort1998revisiting}, using ideas from Ambrosio and Tortorelli \cite{ambrosio1990approximation}. It has been widely adopted in the mechanics community and extended for use in a variety of fracture mechanics problems,  such as ductile failure \cite{alessi2014gradient, ambati2015phase, miehe2016phase, borden2016phase, hu2021variationalpaper}, hydraulic fracture \cite{wilson2016phase, chukwudozie2019variational, mikelic2015phase1, santillan2018phase, miehe2016phase}, dessication problems \cite{maurini2013crack, heider2020phase, cajuhi2018phase, hu2020frictionless}, dynamic fracture\cite{bourdin2011time, borden2012phase, hofacker2013phase, schluter2014phase, li2016gradient, kamensky2018hyperbolic, moutsanidis2018hyperbolic}, fracture in biomaterials \cite{wu2020fracture, raina2016phase, nagaraja2021phase, gultekin2016phase, gultekin2018numerical} and many more. Some recent reviews can be found in \cite{ambati2015review, wu2020phase, francfort2021variational}.

With regard to the use of the phase-field method for hydraulic fracture problems, one challenge concerns how best to incorporate surface loads that result from pressures on crack faces that are diffuse.  One approach is to regularize the resulting surface tractions with an approach that is very similar to how the crack surface energy is regularized. Early work along these lines focused on crack surfaces loaded by constant pressures, as in Bourdin et al. \cite{bourdin2012variational} and Wheeler et al.\cite{wheeler2014augmented}. Since these early developments, these models have been used extensively for the study of pressurized fractures, for example in \cite{tanne2022loss, zulian2021large, yoshioka2019comparative, yoshioka2020crack}.
They were also extended and modified to account for fluid flow inside the fractures and poroelasticity in the surrounding medium \cite{miehe2016phase, mikelic2015phase1, chukwudozie2019variational, wilson2016phase, santillan2018phase, heider2017phase, li2022hydro}. The reader is referred to the recent review by  Heider \cite{heider2021review} for additional works on phase-field methods for hydraulic fracture.  The various models all employ some form of ``indicator function" that assists in the regularization of the surface load itself.  Despite several different indicator functions being proposed, the implication of the particular choice of indicator on the accuracy of the models has yet to be thoroughly examined.  

In this manuscript, a new formulation for the study of pressurized fractures, first proposed in the thesis of Hu~\cite{hu2021variationalthesis} is also examined. In particular, it is studied in combination with a cohesive version of the phase-field for fracture method, which was proposed in the recent works of \cite{lorentz2011convergence, geelen2019phase, wu2017unified}.  This facilitates the study of pressurized fracture in quasi-brittle materials and reduces the sensitivity of the effective strength to the regularization length. To ensure that the cohesive fracture behavior is preserved, the implicit traction-separation law is evaluated for a simple one-dimensional problem and shown to be insensitive to the applied pressure with the new formulation. 
Fracture initiation and propagation examples are also examined to highlight advantages and limitations of the model. 



\section{Research contributions}

%% summary
This dissertation aims to extend the existing variational framework to account for thermal-mechanical-fracture coupling. The proposed framework is able to account for large deformation, inelastic deformation, thermal effects (i.e. heat conduction, heat convection, and heat generation), viscous effects, and potentially dynamic effects. From a fracture mechanics point of view, the framework is flexible enough to model brittle, quasi-brittle, cohesive, and ductile fracture. The applicability to each type of fracture is examined in the following chapters: A brittle/quasi-brittle instantiation is applied to model three-dimensional intergranular fracture and fission-gas-induced fragmentation in microstructures, a cohesive fracture model is applied to revisit the soil desiccation problem with a focus on random fracture properties, and a family of ductile fracture models are developed to simulate quasi-static ductile failure under isothermal conditions as well as thermal-induced oxide spallation.

\begin{itemize}
  \item %% contributions in chapter 3: 3D intergranular fracture, fission-gas-induced fragmentation
        In \Cref{section: Chapter3}, phase-field for brittle/quasi-brittle fracture models are employed to simulate intergranular cracking with multiple bubbles on the grain boundaries in 3D and over-pressurized fission-gas-induced fragmentation at the microstructural level. To account for pressurized crack surfaces, we present an extension of the quasi-brittle model by including regularized external work done by pressure into the total energy of the system. The regularized external work is derived based on a phase-field approximation of the sharp interface. We demonstrate that the prediction of the dependence of the critical fracture strength on porosity based on 3D simulations is better than the 2D prediction. The effects of multi-bubble interaction and partial recrystallization in HBS are also investigated.
  \item %% contributions in chapter 4: contact split, interfacial energy, random properties
        In \Cref{section: Chapter4}, a phase-field for cohesive fracture model is used to simulate pervasive cracking in thin films. A new strain energy split is proposed to enforce frictionless contact conditions in the vicinity of diffuse fracture surfaces. In contrast to existing splits that have been proposed, our approach completely prevents tractions from being transmitted across fully damaged surfaces that are loaded in tension. We construct a probabilistic model for the critical fracture energy and the fracture toughness, modeled as (potentially correlated) random fields, and demonstrate, through forward analysis and by solving an inverse problem, how crack network morphology can be influenced by stochastic spatially-varying material properties.
  \item %% contributions in chapter 5: variational ductile fracture, coalescence dissipation, thermal effects
        In \Cref{section: Chapter5}, we present a variational model for ductile fracture, and demonstrate, through numerical examples, that the model has the following properties: An unperturbed elastic-plastic response can be obtained and the plastic hardening law remain unmodified prior to crack initiation even when the plastic energy is degraded as a function of the phase-field variable; a regularization-length independent critical fracture strength and softening response can be obtained by a family of rational degradation functions; an novel dual kinetic potential, termed with \emph{the coalescence dissipation}, can be incorporated to introduce an alternative form of coupling between plasticity and fracture. Thermal effects are then incorporated to model spallation of the oxide scale in a high-temperature heat exchanger.
\end{itemize}


\section{Organization of the dissertation}

First, the variational framework is presented in \Cref{section: Chapter2}. General kinematics and constraints assumed throughout this dissertation are summarized in \Cref{section: Chapter2/kinematics}. \Cref{section: Chapter2/thermodynamics} provides a brief review of thermodynamic conservations and laws in their global and local forms.
The proposed variational framework is presented in \Cref{section: Chapter2/minimization}, and discretization of the governing equations is presented in \Cref{section: Chapter2/discretization}.

The variational framework is applied to solve engineering problems in \Cref{section: Chapter3,section: Chapter4,section: Chapter5}. \Cref{section: Chapter3} simulates intergranular fracture and fission-gas-induced fracture in microstructures. \Cref{section: Chapter4} revists the soil dessication problem and explores the effect of random fracture properties. \Cref{section: Chapter5} benchmarks the ductile fracture model with a three-point bending experiment and one of the Sandia Fracture Challenges, and models oxide spallation in high-temperature heat exchangers. Each chapter begins with a general introduction to provide context and an overview of state-of-the-art methods and models. Then, the specific constitutive choices are made, and the resulting governing equations are presented. Each chapter is concluded with verifications of the variational model and some concrete numerical examples.


\section{Notation}
\label{section: intro/notation}

In what follows, deterministic scalar, vectors, second-order tensors, and fourth-order tensors are denoted by $a$ (or $A$), $\bta$ (or $\btA$), $\bfA$, and $\mathbb{A}$, respectively.

Let $\body$ be a collection of points $\btX \in \mathbb{R}^d$, $d \in \{1, 2, 3\}$. Scalar- and vector-valued random fields defined on the probability space $(\Theta, \Sigma, \mathbb{P})$, indexed by $\body$, are denoted as $\{ A(\btX), \btX \in \body \}$ and $\{ \btA(\btX), \btX \in \body \}$, respectively.
At any fixed material point $\btX \in \body$, $a(\btX)$ and $\bta(\btX)$ are random variables defined on the probability space $(\Theta, \Sigma, \mathbb{P})$. For any fixed $\theta \in \Theta$, $a(\theta)$ and $\bta(\theta)$ are realizations of the random variables.
Similarly, $\btX \mapsto a(\btX;\theta)$ and $\btX \mapsto \bta(\btX;\theta)$ are realizations of the random fields $\{ A(\btX), \btX \in \body \}$ and $\{ \btA(\btX), \btX \in \body \}$.

Einstein summations are assumed wherever applicable unless otherwise stated. For any vectors $\bta$ and $\btb$ of the same size, the inner product is defined as $\bta \cdot \btb = a_ib_i$ where $a_i$ and $b_i$ are components of the vectors. The associated vector norm is $\norm{\bta}^2 = a \cdot a$. Similarly, for any second-order tensors $\bfA$ and $\bfB$ of the same size, the inner product is defined as $\bfA : \bfB = \tr(\bfA^T \bfB)$. The associated Frobenius norm writes $\norm{\bfA}_F = \sqrt{\bfA : \bfA}$.

