\section{Research contributions}

%% summary
This dissertation extends the existing variational framework to account for thermal-mechanical-fracture coupling. The proposed framework is able to account for large deformation, inelastic deformation, thermal effects (i.e. heat conduction, heat convection, and heat generation), viscous effects, and potentially dynamic effects. From a fracture mechanics point of view, the framework is flexible enough to model brittle, quasi-brittle, cohesive, and ductile fracture. The applicability to each type of fracture is examined in the following chapters: a brittle/quasi-brittle instantiation is applied to model three-dimensional intergranular fracture and fission-gas-induced fragmentation in microstructures; a cohesive fracture model is applied to revisit the soil desiccation problem with a focus on random fracture properties; and a family of ductile fracture models are developed to simulate quasi-static ductile failure under isothermal conditions as well as thermal-induced oxide spallation.

\begin{itemize}
  \item %% contributions in chapter 2: Revised formulation of pressurized cracks
        In \Cref{section: Chapter2}, we present a new formulation for including crack-face pressure loads and discuss how this formulation can be achieved by modifying the trial space in the traditional variational principle or by introducing a new functional dependent on the rates of the primary variables. The key differences between the new formulation and existing models for pressurized cracks in a phase-field setting are emphasized. Model-based simulations developed with discretized versions of the new formulation and existing models are then used to illustrate the advantages and differences. In order to analyze the results, a domain form of the J-integral is developed for diffuse cracks subjected to pressure loads.
  \item %% contributions in chapter 3: Multi-resolution approach in 2D
        In \Cref{section: Chapter3}, we propose a multi-resolution approach for constructing model-based simulations of fluid-driven fracture. The approach consists of a hybrid scheme that couples a discrete crack representation in a global domain to a phase-field representation in a local subdomain near the crack tip. The multi-resolution approach addresses issues such as the computational expense of accurate hydraulic fracture simulations and the difficulties associated with reconstructing crack apertures from diffuse fracture representations. The efficacy of the method is illustrated through various benchmark problems in hydraulic fracturing, as well as a new study of fluid-driven crack growth around a stiff inclusion.
  \item %% contributions in chapter 4: Extension to 3D planar or possibly non-planars
        In \Cref{section: Chapter4}, we present an extension of the multi-resolution approach to three dimensions, which involves a modification to the propagation scheme used in \Cref{section: Chapter3}. We also show some results of simulations of reservoir scale problems with multiple fractures and realistic material data. Considerations about efficient implementations in supercomputers as well as extensions to nonplanar cases are also discussed.
\end{itemize}
