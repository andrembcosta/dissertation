\section{Introduction}
\label{section: cohesive/intro}

In this work, a phase-field for cohesive fracture model is used to simulate pervasive cracking in thin films. A new elastic energy split is proposed to enforce  frictionless contact conditions  in the vicinity of diffuse fracture surfaces. In contrast to existing splits that have been proposed, our approach completely prevents tractions from being transmitted across fully damaged surfaces that are loaded in tension. Spatial variations of material properties are incorporated into our model, and we demonstrate, through forward analysis and by solving an inverse problem, how crack network morphology can be influenced by stochastic spatially-varying material properties.

The variational approach to fracture was proposed by \citet{Francfort98}, and the phase-field implementation is attributed to \citet{Bourdin2000}.
\citet{bourdin2008variational} provides an overview.  In the variational approach, crack surfaces are represented by a  surface density function in terms of an auxiliary phase-field.  This naturally gives rise to a regularization involving a length scale parameter. Such an approach is known to be thermodynamically consistent and recent works have illustrated its potential to be predictive for a wide range of fracture problems. Phase-field for fracture models have succeeded in capturing complex crack patterns, including branching and coalescence in both two and three dimensions \cite{karma_2001, karma_2004, henry_2004, spatschek_2007, amor_2009}.  The approach has also been used to study the fracture of thin films, for example by \citet{baldelli2013delamination, baldelli2014variational}.

To prevent crack growth under compression, phase-field models of fracture typically employ some form of tension-compression asymmetric split in the elastic energy. Two popular approaches are the spectral decomposition \cite{miehe_2010_p1, miehe_2010_p2} and the volumetric-deviatoric decomposition \cite{AMOR20091209}. Both the spectral decomposition and the volumetric-deviatoric decomposition are variational and lead to thermodynamically consistent fracture models.
Although these variational approaches represent crack growth reasonably well for a broad range of problems, they do not completely preclude the transmission of tractions across fully damaged surfaces.   In part, this is by design.  The decompositions are designed to allow compressive tractions to be transmitted across fully damaged surfaces.  But since they are developed from the full stress or strain as opposed to the surface tractions, they can also allow umwanted tensile or shear tractions to be transmitted. This can give rise to spurious macro-scale responses in some situations.
\citet{strobl2015novel} proposed constitutive relations that satisfy the boundary conditions on diffuse crack surfaces by taking into account the crack orientation. Although the use of such a constitutive model violated the fundamental variational structure of phase-field models, it was shown to yield better results in some fracture problems \cite{strobl2016constitutive}.  A similar approach was presented by  \citet{fei2019phase, fei2020phasefield} to extend the model to better account for frictional contact conditions at crack surfaces.  Finally, we mention the recent work of \citet{Landis-fatigue}, in which the standard spectral split was modified to better guard against crack growth in compression.

In this work, we propose a variational elastic energy split that enforces frictionless contact conditions.  Importantly, the split effectively prevents the transmission of tractions across fully-damaged surfaces that are loaded in tension.
The proposed split leads to a decomposition that is shown to be important in mode II and mixed-mode fracture problems.
The new phase-field model is then applied to study the classical problem of the fracture of thin films bonded to thick substrates.  Due to the accumulation of inelastic strain and the stiffness mismatch between the film and the substrate, cracks form either transversely through the thickness of the film, referred to as ``channeling'', or at the interface, referred to as ``debonding''.  Film-substrate systems exist across a wide span of application problems, from technological systems such as nuclear fuel pellet--cladding interactions (PCI) \cite{roberts1977pellet, michel20083d, jernkvist1995model, nagase2004effect, billone2008cladding}
to more common civil infrastructure examples such as pavement asphalts \cite{el2005calibration, el2009methodology, saar2010automatic, button2007guidelines, yildirim2006field}.
The study of fracture in these systems is driven by an interest in both understanding the underlying mechanisms and in preventing crack nucleation and evolution.  In the current work, we confine attention to transverse channeling cracks in the film.  When a system of cracks nucleate, branch and coalesce, complicated crack networks appear in the thin films. Hutchinson and Suo~\cite{hutchinson1991mixed} provides a
comprehensive review of crack patterns in film-substrate systems.

The fracture of thin films bonded to substrates has been studied using model-based simulations based on a wide range of methodologies.  These problems are challenging to simulate because they involve pervasive crack nucleation, branching, and coalescence.   Phenomenological spring models were developed and employed by \citet{crosby1997fragmentation, leung2000pattern} and \citet{ sadhukhan2011crack}. \citet{zhang2017modeling} applied a cohesive zone method to study the film-substrate interface debonding, and \citet{sanchez2014modeling}
proposed a mesh fragmentation technique to simulate three-dimensional crack morphologies. Although such techniques are known to produce mesh-dependent results, the resulting crack networks were found to be satisfactory. \citet{liang2003evolving, sukumar2003modeling} and \citet{huang2003modeling} modeled thin film cracking using the eXtended Finite Element Method (XFEM) which allowed for the cracks to evolve independently of the finite-element mesh.

The aforementioned analytical and numerical efforts have provided a great deal of insight into the factors controlling the fracture of thin films.  In particular, the mechanical properties that govern the spacing between cracks have been studied by \citet{hutchinson1991mixed, xia2000crack} and most recently by \citet{yin2008explicit}. In contrast,  the connection between spatial variations in material properties and the morphology of the resulting crack patterns has received fairly little attention. This is despite experimental evidence that crack morphologies are sensitive to spatial variations in material properties, see, e.g. \citet{kitsunezaki2016shaking, kitsunezaki2017stress, halasz2017effect, kitsunezaki2017memory} and \citet{nakahara2018mechanism}. To our knowledge, while some previous model-based simulations of thin-film fracture have incorporated random material properties, they have not specifically considered spatial fluctuations and the construction of proper stochastic models. More broadly, the literature on probabilistic modeling for fracture simulations remains scarce. In \cite{Acton2018}, variability in fracture strength was estimated from multiscale simulations, and spatially-varying mesoscale properties were subsequently integrated into an asynchronous spacetime discontinuous Galerkin finite element based fracture model to study the impact on fragmentation. The integration of apparent elasticity coefficients in a multiscale-informed phase-field formulation was investigated in \cite{Hun2019}, with the aim of reproducing variability in the macroscopic response. The fracture toughness was modeled as deterministic and homogeneous, and was identified by solving an inverse problem. In \cite{Peco2019}, a continuum mapping of the meso-scale structure to the packing fraction at the macro-scale was employed to introduce a random component to phase-field simulations of surfactant-induced fracture of particle rafts. In this work, we construct a probabilistic model for the critical fracture energy and the fracture toughness, modeled as (potentially correlated) random fields, and examine their influence on the random morphology of the resulting fracture patterns.

Our model-based simulations of thin-film fracture for systems with spatially-variable material properties are facilitated in this work through the adoption of an extension of phase-field models to cohesive fracture \cite{wu2016thermodynamically, wu2017unified, geelen2019phase}. Phase-field for cohesive fracture models incorporate the original proposition by \citet{xia2000crack} regarding a ``critical fracture energy''. The regularization length is decoupled from the material properties in these approaches, and the critical strength and the fracture toughness of the material are no longer strongly correlated. This independence permits simulations of drying in systems with realistic material properties for typical clays or soils, for example, without compromising the magnitude of the critical strength or requiring  regularization lengths that are larger than the specimen dimensions.  It also allows systems with spatially-variable fracture properties to be studied using a spatially-constant regularization length,  greatly improving the fidelity of the numerical calculations.

This paper is organized as follows. \Cref{s: theory} provides the theoretical background for phase-field models of fracture, the new decomposition, associated solution algorithmns, and stochastic models for fracture properties.  In \Cref{s: example}, several numerical examples of benchmark problems in quasi-static fracture mechanics are provided to illustrate the efficacy of the new approach to enforcing frictionless contact conditions.  The main results of this work are provided in \Cref{s: application}, where the new model is used to study the formation of crack networks in thin films and in soil dessication problems.  In particular, model-based simulations are used to examine how the fragment statistics and fracture morphologies are influenced by spatial variations in the fracture properties.  Comparisons with both theory and experimental observations are provided.  Finally, a summary and some concluding remarks are provided in the last section.
