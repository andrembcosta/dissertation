\section{Organization of the dissertation}

First, the variational framework is presented in \Cref{section: Chapter2}. General kinematics and constraints assumed throughout this dissertation are summarized in \Cref{section: Chapter2/kinematics}. \Cref{section: Chapter2/thermodynamics} provides a brief review of thermodynamic conservations and laws in their global and local forms.
The proposed variational framework is presented in \Cref{section: Chapter2/minimization}, and discretization of the governing equations is presented in \Cref{section: Chapter2/discretization}.

The variational framework is applied to solve engineering problems in \Cref{section: Chapter3,section: Chapter4,section: Chapter5}. \Cref{section: Chapter3} simulates intergranular fracture and fission-gas-induced fracture in microstructures. \Cref{section: Chapter4} revists the soil dessication problem and explores the effect of random fracture properties. \Cref{section: Chapter5} benchmarks the ductile fracture model with a three-point bending experiment and one of the Sandia Fracture Challenges, and models oxide spallation in high-temperature heat exchangers. Each chapter begins with a general introduction to provide context and an overview of state-of-the-art methods and models. Then, the specific constitutive choices are made, and the resulting governing equations are presented. Each chapter is concluded with verifications of the variational model and some concrete numerical examples.
