    % FROM: A review on phase-field modeling of hydraulic fracturing


    % Hydraulic fracturing can be broadly defined as the process by which a fracture initiates and propagates due to hydraulic loading (i.e., pressure) applied by a fluid inside the fracture [1]. Examples and applications of hydraulic fracturing are abundant in geomechanics. Magma-driven dykes can be considered as natural examples, usually on the scale of tens of kilometers [2–4]. On the application side, fracturing of oil and gas reservoirs using a mixture of viscous hydraulic fluids and sorted sand (proppant) is the most commonly used reservoir stimulation technique [31]. Other applications of hydraulic fracturing include the disposal of waste drill cuttings underground [5], heat production from geothermal reservoirs [6], goafing [7] and fault reactivation [8] in mining, and the measurement of in situ stresses [108,9].

    % FROM: Computer simulation of hydraulic fractures

    % The method of propagating fractures with hydraulically pressurized fluids is common in many engineering applications, particularly in the field of geology. Certainly the most notable of these is the stimulation of oil and gas wells, but other applications include waste disposal (Bell, 2004), in situ stress measurement (Aamodt and Kuriyagawa, 1981), the stimulation of ground water wells (Banks et al., 1996; Less and Andersen, 1994), and geothermal reservoir development (Ghassemi, 2012). The advent of massive hydraulic fracturing and horizontal drilling in recent decades has made the extraction of oil and gas from unconventional reservoirs, particularly shale, economically viable. Additionally, heat extraction from subsurface geothermal systems has the potential to be a significant source of renewable carbon-free energy both world-wide and in the U.S. (Bertani, 2012; Matek, 2015). As such, the optimization of the many applications related to fluid driven fracturing is becoming increasingly important.
    
    % FROM: Phase-field modeling of hydraulic fracture 

    % As a technique to fracture underground rock formation using pressurized fluid, hydraulic fracturing has been applied extensively in such diverse areas as reservoir stimulation, in-situ stress estimation, caving and fault reaction in mining, and environmental subsurface remediation [3, 141, 221, 270, 375]. In the context of reservoir stimulation, shale gas production has been significantly improved by the wide adoption of multi-stage hydraulic fracturing and horizontal drilling techniques. Hydraulic fracturing plays a critical role in recovering oil and gas from unconventional reservoirs, where it creates fracture networks to provide the transport path in tight formations. Since the first field test performed on a gas well at the Hugoton field in 1947 [44], hydraulic fracturing has been extensively researched by both academia and industries using experimental tests, field trials and numerical simulations [83, 168].

    % FROM: A Review of Hydraulic Fracturing Simulation

    % Hydraulic fracturing technologies have many applications in different fields of engineering and, thus, have attracted many theoretical, experimental, and numerical studies in the last decades. To give some examples, hydraulic fracturing is utilized in enhanced geothermal systems (EGS) to improve the permeability of high temperature but low permeable rock layers, so that the efficiency of geothermal systems increases. In petroleum engineering, hydraulic fracturing (fracking) is also employed using fluids containing chemical additions to recover natural gas in unusual forms such as coal seam gas, shale, and tight gas. An overview of the development of this technology in petroleum engineering can be found in, e.g., Smith and Montgomery [1]. Despite the major economic gains of this technology, critics of hydraulic fracturing point to a potential triggering of seismic events, particularly in the case of its application within the EGS. Other objections arise from the potential increase of air and noise pollution or groundwater and surface water contamination if chemical additions are mixed with the injection fluid. The latter concerns highlight the necessity of a thorough understanding of cracking processes and the importance of developing accurate and efficient tools that can forecast crack initiation and propagation.

    % FROM: A review on phase-field modeling of hydraulic fracturing

    % % Carbon-storage and sequestration

    % Future industrial-scale CO2 sequestration operations are expected to induce pressure disturbances over vast areas. Over such areas, finding caprocks that are perfectly homogeneous and impermeable is not likely. Any given caprock may be discontinuous and may contain heterogeneities, such as faults and fractures of various sizes (i.e., from small meter-scale fractures to kilometer-scale faults). Often, the fracturing within a reservoir having multiple natural fractures leads to a complex fracture network. As part of CO2 sequestration operations in brine aquifers, caprock fracturing could increase the risk of CO2 leakage. To safely sequester CO2 in deep saline aquifers, we must understand the multiple fracture-propagation mechanism of caprock. When a fracture begins to grow, we must determine what the direction of the crack propagation is, and how the propagating fracture would interact with other discontinuities. Will the fracture cross existing discontinuities, or will it be terminated? Answers to these questions can assist in assessing the stability of caprock with respect to CO2 sequestration.

    % FROM: TOUGH–RDCA modeling of multiple fracture interactions in caprock during CO2 injection into a deep brine aquifer

    % % EGS

    % - Geothermal energy is a promising approach to generate clean energy. However, it requires the injection and recovery of water into hot rocks that are deep in the subsurface and tend to have low permeability, making the productivity bad in the natural state. One of the main solutions to this issue is to stimulate the rock through hydraulic fracturing, so that the efficiency of geothermal systems increases. This means that, the economic viability of an EGS site is dependent on an effective fracturing treatment. Thus, being able to thoroughly understand the hydraulic fracturing process is crucial to operate such systems safely and effectively. 

    % To give some examples, hydraulic fracturing is utilized in enhanced geothermal systems (EGS) to improve the permeability of high temperature but low permeable rock layers, so that the efficiency of geothermal systems increases.

    % For EGS, the effectiveness of the stimulation treatment determines the performance of a target site.

    % Hydraulic fracturing consists in enhancing the permeability of a natural formation by injecting a fracturing fluid (e.g., water) at a high pressure. This technique has been widely used in many geo-engineering applications, such as unconventional oil and gas production [1–3] and enhanced geothermal systems (EGS) [46], in which rock masses are nearly impermeable. For EGS, the effectiveness of the stimulation treatment determines the performance of a target site. Thus, being able to thoroughly understand the hydraulic fracturing process is crucial to operate such systems safely and effectively. As a consequence, there has been a growing interest in the development of numerical approaches to model hydraulic fracturing.

    % FROM: A phase-field model for hydraulic fracture nucleation and propagation in porous media