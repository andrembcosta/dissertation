\chapter{Conclusions and Future Work}
\label{section: Conclusion}

This is the conclusion and suggestions for future work.

\section{Concluding Remarks}

This manuscript examines various models for phase-field fracture incorporating pressure loads on diffuse crack faces.  This includes the analysis of a new formulation that can be obtained by assuming that the pressure load does not act on the virtual extension of a crack, or alternatively through a careful accounting in the minimization procedure. The new formulation is referred to as the 
 ``unloaded virtual crack formulation"\eqref{uvc}. In order to verify the accuracy of the various models for propagating cracks, a new form of the J-Integral for pressurized cracks in the phase-field context is derived.

The \eqref{uvc} formulation proposed herein allows for a unified treatment of crack nucleation and propagation in scenarios involving either brittle or cohesive fracture, and provides for better accuracy in some problems compared to existing formulations of the \eqref{lvc} type. As it allows for the use of the same governing equation for the damage parameter, its computational implementation within existing phase-field solvers is also simpler. In future work, its applicability to problems involving plastic deformation and strength-based fracture nucleation will be studied. In addition to that, modifications to accelerate the convergence of the model with respect to the phase-field parameter $\ell$ will also be considered.

\section{Acknowledgments}

%The authors would like to acknowledge the support and funding from Duke University and Argonne National Laboratory, which was essential to the development of this work.  
The partial support of A.\ Costa and  J.E.\ Dolbow by the National Science Foundation, through grant CMMMI-1933367 to Duke University, is gratefully acknowledged.  Author T.\ Hu gratefully acknowledges the support of Argonne National Laboratory.  Argonne National Laboratory is managed and operated by UChicago Argonne LLC.\ for the U.S.\ Department of Energy under Contract No.\ DE-AC02-06CH11357.