\section{Thermodynamics}
\label{section: Chapter2/thermodynamics}

Recall that the motions of the body are described by the deformation mapping $\bs{\upphi}: \body_0 \times [t_0, t] \mapsto \mathbb{R}^d$.
Let $\rho_0$ be the density in the reference configuration, $\btb$ be the distributed body force per unit mass, $\bfP$ be the first Piola-Kirchhoff stress, $\normal_0$ be the outward normal in the reference configuration, $u$ be the internal energy density, $k = \rho_0 \defrate \cdot \defrate$ be the kinetic energy density, $\mathcal{P}^\text{ext}$ be the external power expenditure, $s$ be the entropy density, $q$ be the distributed heat source per unit mass, and $\bth$ be the heat flux. The densities $u$, $k$ and $s$ are all defined per unit volume in the reference configuration.
The motions of the body must obey the following conservations and thermodynamic laws:

\begin{itemize}
  \item Conservation of mass:
        \begin{align}
          \dfrac{\diff{}}{\diff{t}} \int\limits_{\body'} \rho_0 \diff{V} = 0.
        \end{align}
  \item Conservation of linear momentum:
        \begin{align}
          \dfrac{\diff{}}{\diff{t}} \int\limits_{\body'} \rho_0 \defrate \diff{V} = \int\limits_{\body'} \rho_0 \btb \diff{V} + \int\limits_{\bodyboundary'} \bfP\normal_0 \diff{A}.
        \end{align}
  \item Conservation of angular momentum:
        \begin{align}
          \dfrac{\diff{}}{\diff{t}} \int\limits_{\body'} \bs{\upphi} \times (\rho_0 \defrate) \diff{V} = \int\limits_{\body'} \bs{\upphi} \times (\rho_0 \btb) \diff{V} + \int\limits_{\bodyboundary'} \bs{\upphi} \times (\bfP\normal_0) \diff{A}.
        \end{align}
  \item The first law of thermodynamics:
        \begin{equation}
          \begin{aligned}
            \dfrac{\diff{}}{\diff{t}} \int\limits_{\body'} u \diff{V} + \dfrac{\diff{}}{\diff{t}} \int\limits_{\body'} k \diff{V} = \int\limits_{\body'} \mathcal{P}^\text{ext} \diff{V} + \int\limits_{\body'} \rho_0 q \diff{V} - \int\limits_{\bodyboundary'} \bth \cdot \normal_0 \diff{A}.
          \end{aligned}
        \end{equation}
  \item The second law of thermodynamics:
        \begin{align}
          \dfrac{\diff{}}{\diff{t}} \int\limits_{\body'} s \diff{V} - \int\limits_{\body'} \dfrac{\rho_0 q}{T} \diff{V} + \int\limits_{\bodyboundary'} \dfrac{\bth \cdot \normal_0}{T} \diff{A} \geqslant 0.
        \end{align}
\end{itemize}
All the above conservation and thermodynamic laws hold for any arbitrary sub-body $\body' \subset \body$, hence they can be written in the following local form:
\begin{subequations}
  \begin{align}
     & \dot{\rho_0} = 0,                                                                                                                \\
     & \rho_0 \bta = \divergence \bfP + \rho_0\btb,                                                                                     \\
     & \bfP\defgrad = \defgrad\bfP^T,                                                                                                   \\
     & \dot{u} + \dot{k} = \mathcal{P}^\text{ext} + \rho_0 q - \divergence \bth,                                                        \\
     & \dot{s}^\text{int} = \dot{s} - \dfrac{\rho_0 q}{T} + \divergence \dfrac{\bth}{T} \geqslant 0. \label{eq: dissipation inequality} 
  \end{align}
\end{subequations}
Alternatively, local form of the second law can be written in terms of the internal energy density as
\begin{align}
  \dot{s}^\text{int} = \mathcal{P}^\text{ext} - \dot{u} - \dfrac{1}{T} \bth \cdot \grad T. \label{eq: dissipation inequality 2}
\end{align}

For convenience, the collection of thermodynamic state variables (working with the Helmholtz free energy), kinematic degrees of freedom, and internal variables are defined as
\begin{align}
  \mathcal{S} = \left\{ \mathcal{K}, T \right\}, \quad \mathcal{K} = \left\{ \bs{\upphi}, \bfZ \right\}, \quad \bfZ = \left\{ \bfF^p, \ep, d \right\}.
\end{align}
The rates of change of the kinematic variables are denoted as $\mathcal{V} = \{ \defrate, \dot{\defgrad}^p, \epdot, \dot{d} \}$. The generalized velocities of the kinematic state variables are collected in the set
\begin{align}
  \dot{\Lambda} = \{ \defrate, \dot{\defgrad}, \dot{\defgrad}^p, \epdot, \dot{d}, \grad \dot{d} \}.
\end{align}
Recall that the local thermodynamic state is assumed to be depend only on $\bs{\upphi}$, $\bfZ$ and $s$, i.e.
\begin{align}
  u = \hat{u}(\Lambda, s), \quad T = \hat{T}(\Lambda, s).
\end{align}
It is convenient to work with the Helmholtz free energy density (per unit volume) with $T$ as a state variable by introducing the Legendre transformation
\begin{align}
  \psi(\Lambda, T) = \inf_s \left[ u(\Lambda, s) - Ts \right].
\end{align}
The local form of the second law can then be rewritten using the identity $\dot{\psi} = \dot{u} - \dot{T}s$ and the fact that $\mathcal{P}^\text{ext} = \mathcal{P}^\text{int}$:
\begin{align}
  \dot{s}^\text{int} & = \delta - \dfrac{1}{T} \bth \cdot \grad T,  \quad \delta = \mathcal{P}^\text{int} - \dot{\psi} - \dot{T}s. \label{eq: dissipation inequality 3} 
\end{align}
where $\delta$ shall be referred to as the internal dissipation density (per unit volume). Combining \eqref{eq: dissipation inequality} and \eqref{eq: dissipation inequality 3} yields the energy balance in the \emph{entropy form}:
\begin{align}
  T\dot{s} = \rho_0 q - \divergence \bth + \delta. \label{eq: energy balance entropy form}
\end{align}

The internal power expenditure $\mathcal{P}^\text{int}$ can be expressed in terms of generalized forces as
\begin{align}
  \mathcal{P}^\text{int} = \bfP:\dot{\bfF} + \bfT:\dot{\defgrad}^p + Y\epdot + f\dot{d} + \bs{\xi}\cdot\grad\dot{d}.
\end{align}
It is assumed that each of the generalized forces can be additively decomposed into an equilibrium part, thermodynamically conjugate to the Helmholtz free energy density, and a viscous part, i.e.
\begin{subequations}
  \begin{align}
    \bfP     & = \bfP^\text{eq}(\Lambda, T) + \bfP^\text{vis}(\dot{\defgrad}, T; \Lambda),       \\
    \bfT     & = \bfT^\text{eq}(\Lambda, T) + \bfT^\text{vis}(\dot{\defgrad}^p, T; \Lambda),     \\
    Y        & = Y^\text{eq}(\Lambda, T) + Y^\text{vis}(\epdot, T; \Lambda),                     \\
    f        & = f^\text{eq}(\Lambda, T) + f^\text{vis}(\dot{d}, T; \Lambda),                    \\
    \bs{\xi} & = \bs{\xi}^\text{eq}(\Lambda, T) + \bs{\xi}^\text{vis}(\grad\dot{d}, T; \Lambda), 
  \end{align}
\end{subequations}
where the viscous forces must vanish as the rate diminishes to preclude viscous dissipation in quasi-static processes, i.e.
\begin{subequations}
  \begin{align}
    \lim_{\norm{\dot{\defgrad}} \to 0^+}\bfP^\text{vis}   & = 0, \\
    \lim_{\norm{\dot{\defgrad}^p} \to 0^+}\bfT^\text{vis} & = 0, \\
    \lim_{\abs{\epdot} \to 0^+}Y^\text{vis}               & = 0, \\
    \lim_{\abs{\dot{d}} \to 0^+}f^\text{vis}              & = 0, \\
    \lim_{\norm{\grad\dot{d}} \to 0^+}\bs{\xi}^\text{vis} & = 0. 
  \end{align}
\end{subequations}
For convenience, the equilibrium forces and the viscous forces are collected in the sets
\begin{align}
  \mathcal{F}^\text{eq} = \left\{ \bfP^\text{eq}, \bfT^\text{eq}, Y^\text{eq}, f^\text{eq}, \bs{\xi}^\text{eq} \right\}, \quad \mathcal{F}^\text{vis} = \left\{ \bfP^\text{vis}, \bfT^\text{vis}, Y^\text{vis}, f^\text{vis}, \bs{\xi}^\text{vis} \right\}.
\end{align}

The rate of the Helmholtz free energy density can be expanded as
\begin{align}
  \dot{\psi} = \psi_{,\defgrad}:\dot{\defgrad} + \psi_{,\defgrad^p}:\dot{\defgrad}^p + \psi_{\ep}\epdot + \psi_{,d}\dot{d} + \psi_{,\grad d}\cdot\grad d + \psi_{,T}\dot{T}. \label{eq: rate of helmholtz}
\end{align}
Inserting the identity \eqref{eq: rate of helmholtz} into \eqref{eq: dissipation inequality 3} and applying the Coleman-Noll procedure lead to several thermodynamic restrictions on the constitutive relations:
\begin{align}
  \bfP^\text{eq} = \psi_{,\defgrad}, \quad \bfT^\text{eq} = \psi_{,\defgrad^p}, \quad Y^\text{eq} = \psi_{,\ep}, \quad f^\text{eq} = \psi_{,d}, \quad \bs{\xi}^\text{eq} = \psi_{,\grad d}, \quad -s = \psi_{,T}. \label{eq: constitutive restrictions}
\end{align}
Substituting \eqref{eq: constitutive restrictions} and \eqref{eq: rate of helmholtz} into \eqref{eq: dissipation inequality 3} simplifies the definition of the internal dissipation density:
\begin{align}
  \delta = \bfP^\text{vis}:\dot{\defgrad} + \bfT^\text{vis}:\dot{\defgrad}^p + Y^\text{vis}\epdot + f^\text{vis}\dot{d} + \bs{\xi}^\text{vis}\cdot\grad d.
\end{align}

Using the identities \eqref{eq: rate of helmholtz} and \eqref{eq: constitutive restrictions}, the rate of the entropy density can be expanded as
\begin{align}
  \dot{s} = -\dot{\psi}_{,T} = - \bfP^\text{eq}_{,T}:\dot{\defgrad} - \bfT^\text{eq}_{,T}:\dot{\defgrad}^p - Y^\text{eq}_{,T}\epdot - f^\text{eq}_{,T}\dot{d} - \bs{\xi}^\text{eq}_{,T}\cdot\grad d - \psi_{,TT} \dot{T}. \label{eq: expanding s dot}
\end{align}
By introducing the heat capacity per unit mass at constant $\Lambda$:
\begin{align}
  \rho_0c_v = -T\psi_{,TT}, \label{eq: heat capacity}
\end{align}
the energy balance \eqref{eq: energy balance entropy form} can be rewritten as
\begin{align}
  \rho_0 c_v \dot{T} = \rho_0 q - \divergence \bth + \delta + \delta_T, \label{eq: energy balance temperature form}
\end{align}
where $\delta_T$ is the dissipation density accounting for the thermal effects in the thermodynamic conjugates:
\begin{align}
  \delta_T = T\left( \bfP^\text{eq}_{,T}:\dot{\defgrad} + \bfT^\text{eq}_{,T}:\dot{\defgrad}^p + Y^\text{eq}_{,T}\epdot + f^\text{eq}_{,T}\dot{d} + \bs{\xi}^\text{eq}_{,T}\cdot\grad d \right), \label{eq: heat generation due to energetic terms}
\end{align}
e.g. $T\bfP^\text{eq}_{,T}:\dot{\defgrad}$ accounts for thermoelastic effects, and $TY^\text{eq}_{,T}\epdot$ accounts for thermoplastic softening.
