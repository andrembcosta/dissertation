\section{Theory}
\label{section: Chapter4/theory}

%%%%%%%%%%%%%%%%%%%%%%%%%%%%%%%%%%%%%%%%%%%%%%%%%%%%%%%%%%%%%%%%%%%%%%%%%%%%%%%%%%%%%%%%%%%
%%%%%%%%%%%%%%%%%%%%%%%%%%%%%%%%%%%%%%%%%%%%%%%%%%%%%%%%%%%%%%%%%%%%%%%%%%%%%%%%%%%%%%%%%%%
%%%%%%%%%%%%%%%%%%%%%%%%%%%%%%%%%%%%%%%%%%%%%%%%%%%%%%%%%%%%%%%%%%%%%%%%%%%%%%%%%%%%%%%%%%%
%%%%%%%%%%%%%%%%%%%%%%%%%%%%%%%%%%%%%%%%%%%%%%%%%%%%%%%%%%%%%%%%%%%%%%%%%%%%%%%%%%%%%%%%%%%
%%%%%%%%%%%%%%%%%%%%%%%%%%%%%%%%%%%%%%%%%%%%%%%%%%%%%%%%%%%%%%%%%%%%%%%%%%%%%%%%%%%%%%%%%%%
%%%%%%%%%%%%%%%%%%%%%%%%%%%%%%%%%%%%%%%%%%%%%%%%%%%%%%%%%%%%%%%%%%%%%%%%%%%%%%%%%%%%%%%%%%%
%%%%%%%%%%%%%%%%%%%%%%%%%%%%%%%%%%%%%%%%%%%%%%%%%%%%%%%%%%%%%%%%%%%%%%%%%%%%%%%%%%%%%%%%%%%
%%%%%%%%%%%%%%%%%%%%%%%%%%%%%%%%%%%%%%%%%%%%%%%%%%%%%%%%%%%%%%%%%%%%%%%%%%%%%%%%%%%%%%%%%%%
%%%%%%%%%%%%%%%%%%%%%%%%%%%%%%%%%%%%%%%%%%%%%%%%%%%%%%%%%%%%%%%%%%%%%%%%%%%%%%%%%%%%%%%%%%%
%%%%%%%%%%%%%%%%%%%%%%%%%%%%%%%%%%%%%%%%%%%%%%%%%%%%%%%%%%%%%%%%%%%%%%%%%%%%%%%%%%%%%%%%%%%
\subsection{Constitutive choices}

Similar constitutive choices made in \Cref{section: Chapter3} used to model cohesive fracture. In this chapter, the material is assumed to be isotropic while not necessarily homogeneous. Quasi-static and isothermal conditions are assumed to hold. No viscous regularization is incorporated, i.e. the dual kinetic potentials are zero. Following the variational framework (\Cref{chapter: framework}), the Helmholtz free energy density is decomposed as
\begin{align}
  \psi = \psi^e + \psi^f + \psi^i,
\end{align}
where $\psi^e$ is the strain-energy density and $\psi^f$ is the fracture energy density. In the two dimensional simplification of the film-substrate system, $\psi^i$ represents the interfacial energy density that arises from the stiffness mismatch between the film and the substrate at the interface.

The same definition of the strain-energy density as in \eqref{eq: chapter 3 strain-energy density} and consequently the same stress-strain relation \eqref{eq: chapter 3 stress strain relation} are used. The following crack geometric function and degradation function are used to model cohesive fracture:
\begin{align}
  \alpha = d, \quad g = \dfrac{(1-d)^2}{(1-d)^2+md(1+pd)}, \quad m = \dfrac{3 \Gc}{8 l \psi_c}.
\end{align}
This degradation function is hereinafter referred to as the Lorentz (or in general the Lorentz-type) degradation function\footnote{Later on, the Lorentz-type degradation function may be denoted as $g^\text{Lorentz}$ to distinguish from the quadratic degradation function $g^\text{Quadratic} = (1-d)^2$.}.

\begin{remark}
  With a view towards the following stochastic analysis of fracture properties, the use of a Lorentz-type degradation function is necessary. According to one-dimensional analyses in \cite{geelen2019phase,wu2017unified}, the use of Lorentz-type degradation functions decouples the phase-field regularization length from the critical fracture energy, in the sense that the critical fracture energy $\psi^c$ may now be treated as a separate material property and later on as a \emph{marginally independent} random field, and the phase-field regularization length can truly be a numerical regularization parameter which the $\Gamma$-convergence properties of the model depend upon.
\end{remark}

The interfacial energy density describes the stiffness mismatch between the film and the substrate at the interface. The interfacial energy density and its thermodynamic conjugates are defined as
\begin{subequations}
  \begin{align}
    \psi^i                & = \dfrac{1}{2}g\dfrac{G}{hH}\bs{\upphi}\cdot\bs{\upphi},      \\
    \psi^i_{,\bs{\upphi}} & = g\dfrac{G}{hH}\bs{\upphi},                                  \\
    \psi^i_{,d}           & = \dfrac{1}{2}g_{,d}\dfrac{G}{hH}\bs{\upphi}\cdot\bs{\upphi}. 
  \end{align}
\end{subequations}

The gonverning equations can be summarized as
\begin{align}
  \divergence \stress - g\dfrac{G}{hH}\bs{\upphi} & = \bs{0}, \label{eq: chapter 4 momentum balance} \\
  \divergence \bs{\xi} - f                        & = 0,                                             
\end{align}
supplemented by the constitutive relations \eqref{eq: chapter 3 stress strain relation} and
\begin{align}
  \bs{\xi} = \dfrac{3\Gc l}{4} \grad d + \psi^e_{,\grad d}, \quad f = \dfrac{3\Gc}{8l}\alpha_{,d} + \psi^e_{,d} + \dfrac{1}{2}g_{,d}\dfrac{G}{hH}\bs{\upphi}\cdot\bs{\upphi}.
\end{align}
Note that \eqref{eq: chapter 4 momentum balance} ressembles a shear-lag model, and the additional term in $\bs{\xi}$ comes from the contact split to be introduced in \Cref{section: Chapter4/theory/contact_split}.

%%%%%%%%%%%%%%%%%%%%%%%%%%%%%%%%%%%%%%%%%%%%%%%%%%%%%%%%%%%%%%%%%%%%%%%%%%%%%%%%%%%%%%%%%%%
%%%%%%%%%%%%%%%%%%%%%%%%%%%%%%%%%%%%%%%%%%%%%%%%%%%%%%%%%%%%%%%%%%%%%%%%%%%%%%%%%%%%%%%%%%%
%%%%%%%%%%%%%%%%%%%%%%%%%%%%%%%%%%%%%%%%%%%%%%%%%%%%%%%%%%%%%%%%%%%%%%%%%%%%%%%%%%%%%%%%%%%
%%%%%%%%%%%%%%%%%%%%%%%%%%%%%%%%%%%%%%%%%%%%%%%%%%%%%%%%%%%%%%%%%%%%%%%%%%%%%%%%%%%%%%%%%%%
%%%%%%%%%%%%%%%%%%%%%%%%%%%%%%%%%%%%%%%%%%%%%%%%%%%%%%%%%%%%%%%%%%%%%%%%%%%%%%%%%%%%%%%%%%%
%%%%%%%%%%%%%%%%%%%%%%%%%%%%%%%%%%%%%%%%%%%%%%%%%%%%%%%%%%%%%%%%%%%%%%%%%%%%%%%%%%%%%%%%%%%
%%%%%%%%%%%%%%%%%%%%%%%%%%%%%%%%%%%%%%%%%%%%%%%%%%%%%%%%%%%%%%%%%%%%%%%%%%%%%%%%%%%%%%%%%%%
%%%%%%%%%%%%%%%%%%%%%%%%%%%%%%%%%%%%%%%%%%%%%%%%%%%%%%%%%%%%%%%%%%%%%%%%%%%%%%%%%%%%%%%%%%%
%%%%%%%%%%%%%%%%%%%%%%%%%%%%%%%%%%%%%%%%%%%%%%%%%%%%%%%%%%%%%%%%%%%%%%%%%%%%%%%%%%%%%%%%%%%
%%%%%%%%%%%%%%%%%%%%%%%%%%%%%%%%%%%%%%%%%%%%%%%%%%%%%%%%%%%%%%%%%%%%%%%%%%%%%%%%%%%%%%%%%%%
\subsection{Enforcing the traction-free boundary condition}
\label{section: Chapter4/theory/contact_split}

Widely used active/inactive splits are spectral split \cite{miehe_2010_p1, miehe_2010_p2} and volumetric-deviatoric split \cite{amor_2009}.  Unfortunately, neither the spectral split nor the volumetric-deviatoric split completely prevent tensile or shear tractions from being transmitted across fully-damaged surfaces.  Accordingly, we now develop a split that is motivated by the consideration of frictionless contact conditions along fully damaged surfaces.

We begin by recalling the standard Kuhn-Tucker optimality conditions for frictionless contact under small deformation, which can be written as
\begin{equation}
  t_N \geqslant 0, \quad u_I \leqslant 0, \quad t_N u_I = 0,
\end{equation}
where $t_N$ is the contact pressure and $u_I$ denotes the interpenetration between the contact surfaces.  The contact pressure is typically obtained from the stress and the exterior normal of the contact surface.  In regularized models of fracture, however, it can be difficult to identify a unique surface based on the phase field.  Accordingly, we rely on the gradient of the phase field to construct an approximate normal $\xnormal$, viz.
\begin{equation}
  \xnormal = \frac{\grad d}{ \norm{\grad d} },
\end{equation}
and calculate the normal pressure as
\begin{equation}
  \label{eq: pressure}
  t_N = -\xnormal\cdot\stress\xnormal.
\end{equation}
Our split keys off the sign of the normal pressure, by isolating the cases of a positive pressure (for contact) from a negative pressure (for opening).  In particular, we construct a ``normal-tangential'' stress decomposition of the form
\begin{equation}
  \stress  = \stress_n^{+} + \stress_n^{-}  + \stress_t ,
\end{equation}
where
\begin{equation}
  \stress_n^\pm = \macaulay{-t_N}_\pm\xnormal\otimes\xnormal, \quad  \stress_t = \stress - \stress_n^+ - \stress_n^-,
\end{equation}

In the vicinity of the regularized crack surface, we then propose the use of a stress-based strain-energy split, hereafter referred to as the \textit{contact split},
that takes the form:
\begin{subequations}
  \begin{align}
     & \psi^e = g\psi^e_\activepart + \psi^e_\inactivepart , \label{eq: odd begin}                                                            \\
     & \psi^e_\activepart = \dfrac{1}{2}\stress_\activepart:\strain, \quad \psi^e_\inactivepart = \dfrac{1}{2}\stress_\inactivepart:\strain , \\
     & \stress_\activepart = \stress_n^+ + \stress_t, \quad \stress_\inactivepart = \stress_n^-.                                              
  \end{align}
\end{subequations}
After a lengthy but straight-forward derivation, the corresponding constitutive relation follows as:
\begin{align}
  \stress  = g(d)\stress_\activepart + \stress_\inactivepart , \label{eq: odd end}
\end{align}
and
\begin{subequations}
  \begin{align}
    {\psi^e_\activepart}_{,\grad d}   & = \dfrac{1}{\norm{\grad d}} \left[ 2 \macaulay{-t_N}_- \varepsilon_N \xnormal - H(t_N) \varepsilon_N \stress\xnormal - \macaulay{-t_N}_- \strain\xnormal \right], \\
    {\psi^e_\inactivepart}_{,\grad d} & = -{\psi^e_\inactivepart}_{,\grad d},                                                                                                                             
  \end{align}
\end{subequations}
with $\varepsilon_N = \xnormal\cdot\strain\xnormal$.

It bears emphasis that the proposed strain-energy split only serves to enforce the crack surface traction free boundary condition, and it is not reasonable to expect it to reliably govern crack nucleation and growth. In this chapter, to ensure that the contact split is applied only in the vicinity of the crack surface, we impose a simple threshold $d_c$ such that we apply
\begin{align*}
  \begin{cases}
    \text{the contact split \eqref{eq: odd begin} to \eqref{eq: odd end}, } d \geqslant d_c \\
    \text{the strain-based spectral split \cite{miehe_2010_p1, miehe_2010_p2}, } d < d_c.
  \end{cases}
\end{align*}

\begin{remark}
  Alternative to a threshold, in principle a total energy could be constructed using a blending function that transitions between a standard split and the contact split.
\end{remark}

It can be shown that the definition of the contact pressure \eqref{eq: pressure} follows by relaxing the frictionless constraint into a minimization problem, and $t_N$ is therefore the best scalar quantity one can choose to approximate the traction-free boundary condition:
\begin{subequations}
  \begin{align}
    \min\limits_{t_N} \quad & \ \norm{\stress_t(t_N)\xnormal}^2                                                                                    \\
    =                       & \ \norm{(\stress-\macaulay{-t_N}_+\xnormal\otimes\xnormal-\macaulay{-t_N}_- \xnormal\otimes\xnormal)\cdot\xnormal}^2 \\
    =                       & \ \norm{(\stress+t_N \xnormal\otimes\xnormal)\cdot\xnormal}^2                                                        \\
    =                       & \ t_N^2 + 2t_N\xnormal\cdot\stress\xnormal + \norm{\stress\xnormal}^2,                                               
  \end{align}
\end{subequations}
and the minimizer is
\begin{align}
  t_N = \argmin\limits_{t_N} \left( t_N^2 + 2t_N\xnormal\cdot\stress\xnormal + \norm{\stress\xnormal}^2 \right) = -\xnormal\cdot\stress\xnormal .
\end{align}

%%%%%%%%%%%%%%%%%%%%%%%%%%%%%%%%%%%%%%%%%%%%%%%%%%%%%%%%%%%%%%%%%%%%%%%%%%%%%%%%%%%%%%%%%%%
%%%%%%%%%%%%%%%%%%%%%%%%%%%%%%%%%%%%%%%%%%%%%%%%%%%%%%%%%%%%%%%%%%%%%%%%%%%%%%%%%%%%%%%%%%%
%%%%%%%%%%%%%%%%%%%%%%%%%%%%%%%%%%%%%%%%%%%%%%%%%%%%%%%%%%%%%%%%%%%%%%%%%%%%%%%%%%%%%%%%%%%
%%%%%%%%%%%%%%%%%%%%%%%%%%%%%%%%%%%%%%%%%%%%%%%%%%%%%%%%%%%%%%%%%%%%%%%%%%%%%%%%%%%%%%%%%%%
%%%%%%%%%%%%%%%%%%%%%%%%%%%%%%%%%%%%%%%%%%%%%%%%%%%%%%%%%%%%%%%%%%%%%%%%%%%%%%%%%%%%%%%%%%%
%%%%%%%%%%%%%%%%%%%%%%%%%%%%%%%%%%%%%%%%%%%%%%%%%%%%%%%%%%%%%%%%%%%%%%%%%%%%%%%%%%%%%%%%%%%
%%%%%%%%%%%%%%%%%%%%%%%%%%%%%%%%%%%%%%%%%%%%%%%%%%%%%%%%%%%%%%%%%%%%%%%%%%%%%%%%%%%%%%%%%%%
%%%%%%%%%%%%%%%%%%%%%%%%%%%%%%%%%%%%%%%%%%%%%%%%%%%%%%%%%%%%%%%%%%%%%%%%%%%%%%%%%%%%%%%%%%%
%%%%%%%%%%%%%%%%%%%%%%%%%%%%%%%%%%%%%%%%%%%%%%%%%%%%%%%%%%%%%%%%%%%%%%%%%%%%%%%%%%%%%%%%%%%
%%%%%%%%%%%%%%%%%%%%%%%%%%%%%%%%%%%%%%%%%%%%%%%%%%%%%%%%%%%%%%%%%%%%%%%%%%%%%%%%%%%%%%%%%%%
\subsection{Stochastic models for fracture properties}
\label{section: Chapter4/theory/stochastic}

In this chapter, we allow for material property inhomogeneity in the macroscopic continuum model. Specifically, we introduce the random field $\{\bs{P}(btX) = (P_1(btX), P_2(btX)), btX \in \body\}$ defined on the probability space $(\Theta, \Sigma, P)$, indexed by $\body$ and with values in $\field{R}_{>0} \times \field{R}_{>0}$, such that $\{P_1(btX), btX \in \body\}$
(respectively $\{P_2(btX), btX \in \body\}$) is a prior representation of $\{\Gc(btX), btX \in \body\}$ (respectively $\{\psi_c(btX), btX \in \body\}$).
Due to the restrictions on the state space, the bivariate random field $\{\bs{P}(btX), btX \in \body\}$ is non-Gaussian. In order to model nongaussianity, the random field $\{\bs{P}(btX), btX \in \body\}$ is \textit{defined} as
\begin{align}
  \bs{P}(btX) := \fspace{T}(\bs{\Xi}(btX), btX)~, \quad \forall btX \in \body~, \label{eq: def-translation}
\end{align}
where $\fspace{T}$ is a measurable, nonlinear mapping and $\{\bs{\Xi}(btX),btX \in \body\}$ is a centered Gaussian random field with values in $\field{R}^2$ to be defined momentarily. From a methodological standpoint, it should be noticed that the transformation $\fspace{T}$ is constructed \textit{a priori} and used to define $\{\bs{P}(btX), btX \in \body\}$ such that the latter constitutes a surrogate capturing some essential features of $\{(\Gc(btX),\psi_c(btX)), btX \in \body\}$
(while exhibiting a low dimensional parameterization). In general, the existence of a nonlinear mapping $\fspace{T}$ such that \eqref{eq: def-translation} holds for a \textit{given} non-Gaussian field is not guaranteed; see, e.g., \cite{Grigoriu2009}.

%%%%%%%%%%%%%%%%%%%%%%%%%%%%%%%%%%%%%%%%%%%%%%%%%%%%%%%%%%%%%%%%%%%%%%%%%%%%%%%%%%%%%%%%%%%
%%%%%%%%%%%%%%%%%%%%%%%%%%%%%%%%%%%%%%%%%%%%%%%%%%%%%%%%%%%%%%%%%%%%%%%%%%%%%%%%%%%%%%%%%%%
%%%%%%%%%%%%%%%%%%%%%%%%%%%%%%%%%%%%%%%%%%%%%%%%%%%%%%%%%%%%%%%%%%%%%%%%%%%%%%%%%%%%%%%%%%%
%%%%%%%%%%%%%%%%%%%%%%%%%%%%%%%%%%%%%%%%%%%%%%%%%%%%%%%%%%%%%%%%%%%%%%%%%%%%%%%%%%%%%%%%%%%
%%%%%%%%%%%%%%%%%%%%%%%%%%%%%%%%%%%%%%%%%%%%%%%%%%%%%%%%%%%%%%%%%%%%%%%%%%%%%%%%%%%%%%%%%%%
%%%%%%%%%%%%%%%%%%%%%%%%%%%%%%%%%%%%%%%%%%%%%%%%%%%%%%%%%%%%%%%%%%%%%%%%%%%%%%%%%%%%%%%%%%%
%%%%%%%%%%%%%%%%%%%%%%%%%%%%%%%%%%%%%%%%%%%%%%%%%%%%%%%%%%%%%%%%%%%%%%%%%%%%%%%%%%%%%%%%%%%
%%%%%%%%%%%%%%%%%%%%%%%%%%%%%%%%%%%%%%%%%%%%%%%%%%%%%%%%%%%%%%%%%%%%%%%%%%%%%%%%%%%%%%%%%%%
%%%%%%%%%%%%%%%%%%%%%%%%%%%%%%%%%%%%%%%%%%%%%%%%%%%%%%%%%%%%%%%%%%%%%%%%%%%%%%%%%%%%%%%%%%%
%%%%%%%%%%%%%%%%%%%%%%%%%%%%%%%%%%%%%%%%%%%%%%%%%%%%%%%%%%%%%%%%%%%%%%%%%%%%%%%%%%%%%%%%%%%
\subsubsection{Construction of the Non-Gaussian Model}
\label{section: Chapter4/theory/stochastic/non_gaussian_model}

In this work, the transformation $\fspace{T}$ is constructed by imposing the family $\{f_{btX}\}_{btX\in\body}$ of first-order marginal distributions:
\begin{align}
  P_{\bs{P}(btX)}(d\bs{p}) = f_{btX}(\bs{p})\mathrm{d}\bs{p}~, \quad \forall btX \in \body~,
\end{align}
where $f_{btX}$ is the probability density function of $\bs{P}(btX)$, $btX$ being fixed in $\body$, and $\mathrm{d}\bs{p}=\mathrm{d}p_1\mathrm{d}p_2$ is the Lebesgue measure in $\field{R}^2$. It is assumed that $f_{btX}(\cdot) = f(\cdot; \bs{w}_{btX})$, where $\bs{w}_{btX}$ is a vector-valued hyperparameter indexed by $btX \in \body$. In order to simplify the analysis, we assume from now on that the aforementioned hyperparameter does not depend on location, and we write $f_{btX}(\cdot) = f(\cdot; \bs{w})$ using a slight abuse of notation.

The construction of $f$ can be achieved in many different ways. In what follows, the construction is performed by assuming that (i) $\Gc$ and $\psi_c$ (together with their inverses) are positive and have finite variance, for physical consistency, and (ii) the two fracture properties can exhibit some level of correlation. In accordance with information theory \cite{Jaynes1957a,Jaynes1957b} and more precisely, with the principle of maximum entropy \cite{Shannon1948a,Shannon1948b}, the first assumption leads to the consideration of Gamma marginal distributions. Following the notation introduced above, we shall impose that
\begin{align}
  P_{P_1(btX)}(dp_1) = f_\mathcal{G}(p_1; (\underline{p}_1,\delta_{1}))dp_1~, \quad \forall btX \in \body~,
\end{align}
and
\begin{align}
  P_{P_2(btX)}(dp_2) = f_\mathcal{G}(p_2; (\underline{p}_2,\delta_{2}))dp_2~, \quad \forall btX \in \body~,
\end{align}
where $f_\mathcal{G}(\cdot; (\underline{p}, \delta))$ denotes the univariate Gamma probability density function with mean $\underline{p}$ and coefficient of variation $\delta$. In this setting, the joint distribution of $\bs{P}(btX)$ must be constructed from the knowledge of the marginal laws associated with $P_1(btX)$ and $P_2(btX)$. This ill-posed problem does not admit a unique solution, and many possible forms were proposed in the case of a bivariate Gamma distribution (see, e.g., Chapter 8 in \cite{Balakrishnan2009} for a review). Here, we use the bivariate Gamma distribution derived by Moran using a copula \cite{Moran1969}:
\begin{align}
  f(\bs{p}) = \frac{1}{\sqrt{1-\rho^2}}\exp\left\{-\frac{1}{2(1-\rho^2)}\left[(\rho \tilde{p}_1)^2-2\rho \tilde{p}_1\tilde{p}_2+(\rho \tilde{p}_2)^2\right]\right\} f_\mathcal{G}(p_1; (\underline{p}_1,\delta_{1})) f_\mathcal{G}(p_2; (\underline{p}_2,\delta_{2})),
\end{align}
where $\rho \in (-1,1)$ is the (Pearson) correlation coefficient between $P_1(btX)$ and $P_2(btX)$ ($btX$ being fixed), and
\begin{align}
  \tilde{p}_i = \Phi^{-1}(F_\mathcal{G}(p_i; (\underline{p}_i,\delta_{i})))~, \quad i\in\{1,2\}~,
\end{align}
where $\Phi^{-1}$ is the inverse of the univariate Gaussian distribution function and $F_\mathcal{G}(\cdot; (\underline{p}_i,\delta_{i})))$ is the univariate Gamma distribution function with mean $\underline{p}_i$ and coefficient of variation $\delta_i$ (which is associated with $f_\mathcal{G}(\cdot; (\underline{p}_i, \delta_i))$), and that the extreme cases $\rho = \pm 1$ are also well defined; see \cite{Moran1969}. Notice that the vector of hyperparameters then reads as $\bs{w} = (\underline{p}_1,\delta_{1},\underline{p}_2,\delta_{2},\rho)$. It follows that
\begin{align}
  P_1(btX) = F_\mathcal{G}^{-1}(\Phi(\Upsilon_1(btX)); (\underline{p}_1,\delta_{1})))
\end{align}
and
\begin{align}
  P_2(btX) = F_\mathcal{G}^{-1}(\Phi(\Upsilon_2(btX)); (\underline{p}_2,\delta_{2})))
\end{align}
for any $btX$ fixed in $\body$, where $\bs{\Upsilon}(btX) = (\Upsilon_1(btX),\Upsilon_2(btX))$ is a centered Gaussian random variable with covariance matrix \cite{Moran1969}:
\begin{align}
  [C_{\bs{\Upsilon}}] =
  \begin{bmatrix}
    1    & \rho \\
    \rho & 1    
  \end{bmatrix}~.
\end{align}
Upon using the Cholesky factorization $[C_{\bs{\Upsilon}}] = [L_{\bs{\Upsilon}}]^\mathrm{T} [L_{\bs{\Upsilon}}]$, it can be deduced that $\{\bs{P}(btX), btX \in \body\}$ can be defined through
\begin{align}\label{eq: def-P1}
  P_1(btX) = F_\mathcal{G}^{-1}(\Phi(\Xi_1(btX)); (\underline{p}_1,\delta_{1})))~, \quad \forall btX \in \body~,
\end{align}
and
\begin{align}\label{eq: def-P2}
  P_2(btX) = F_\mathcal{G}^{-1}(\Phi(\rho \Xi_1(btX) + \sqrt{1-\rho^2} \Xi_2(btX)); (\underline{p}_2,\delta_{2})))~,\quad \forall btX \in \body~,
\end{align}
where $\{\bs{\Xi}(btX)=(\Xi_1(btX),\Xi_2(btX)), btX \in \field{R}^n\}$ is a centered Gaussian random field with statistically independent, centered Gaussian components such that $\bs{\Upsilon}(btX) = [L_{\bs{\Upsilon}}]^\mathrm{T}\bs{\Xi}(btX)$,
$\forall btX \in \body$. \Cref{eq: def-P1,eq: def-P2} define the nonlinear mapping $\fspace{T}$ introduced in \Cref{eq: def-translation} such that $\bs{P}(btX) = \fspace{T}(\bs{\Xi}(btX))$,
for all $btX \in \body$ (note that the spatial dependence of the transformation was dropped given the retained modeling assumptions). This relation can equivalently be stated as $\bs{P}(btX) = \fspace{T}^\ast(\bs{\Upsilon}(btX))$, with obvious notation. The Gaussian random fields are defined in the next section.

%%%%%%%%%%%%%%%%%%%%%%%%%%%%%%%%%%%%%%%%%%%%%%%%%%%%%%%%%%%%%%%%%%%%%%%%%%%%%%%%%%%%%%%%%%%
%%%%%%%%%%%%%%%%%%%%%%%%%%%%%%%%%%%%%%%%%%%%%%%%%%%%%%%%%%%%%%%%%%%%%%%%%%%%%%%%%%%%%%%%%%%
%%%%%%%%%%%%%%%%%%%%%%%%%%%%%%%%%%%%%%%%%%%%%%%%%%%%%%%%%%%%%%%%%%%%%%%%%%%%%%%%%%%%%%%%%%%
%%%%%%%%%%%%%%%%%%%%%%%%%%%%%%%%%%%%%%%%%%%%%%%%%%%%%%%%%%%%%%%%%%%%%%%%%%%%%%%%%%%%%%%%%%%
%%%%%%%%%%%%%%%%%%%%%%%%%%%%%%%%%%%%%%%%%%%%%%%%%%%%%%%%%%%%%%%%%%%%%%%%%%%%%%%%%%%%%%%%%%%
%%%%%%%%%%%%%%%%%%%%%%%%%%%%%%%%%%%%%%%%%%%%%%%%%%%%%%%%%%%%%%%%%%%%%%%%%%%%%%%%%%%%%%%%%%%
%%%%%%%%%%%%%%%%%%%%%%%%%%%%%%%%%%%%%%%%%%%%%%%%%%%%%%%%%%%%%%%%%%%%%%%%%%%%%%%%%%%%%%%%%%%
%%%%%%%%%%%%%%%%%%%%%%%%%%%%%%%%%%%%%%%%%%%%%%%%%%%%%%%%%%%%%%%%%%%%%%%%%%%%%%%%%%%%%%%%%%%
%%%%%%%%%%%%%%%%%%%%%%%%%%%%%%%%%%%%%%%%%%%%%%%%%%%%%%%%%%%%%%%%%%%%%%%%%%%%%%%%%%%%%%%%%%%
%%%%%%%%%%%%%%%%%%%%%%%%%%%%%%%%%%%%%%%%%%%%%%%%%%%%%%%%%%%%%%%%%%%%%%%%%%%%%%%%%%%%%%%%%%%
\subsubsection{Construction of the Underlying Gaussian Models}
\label{section: Chapter4/theory/stochastic/gaussian_model}

In accordance with the periodic assumption retained in the mechanical modeling framework, it is assumed that each random field $\{\Xi_i(btX), btX \in \field{R}^n\}$, $i \in \{1,2\}$, satisfies the invariance property $\Xi_i(btX) = \Xi_i(btX+pbtX')$ for all $btX \in \body$ and $btX' \in \field{Z}^n$
$P$-almost surely, where it is assumed (without loss of generality) that $\body = ([0,p])^n$. Consequently, the centered random field $\{\bs{\Xi}(btX), btX \in \field{R}^n\}$ is uniquely defined by its restriction to $\body$.  Assuming the stationary of this restriction in $\body$, the underlying Gaussian field is defined through the matrix-valued covariance function $\bs{\tau} \mapsto [\cor_{\bs{\Xi}}(\bs{\tau})] = \field{E}\{\bs{\Xi}(btX+\bs{\tau}) \otimes \bs{\Xi}(btX)\}$
such that
\begin{align}
  [\cor_{\bs{\Xi}}(\bs{\tau})] = \left[\begin{array}{cc}\cor_1(\bs{\tau}) & 0 \\0 & \cor_2(\bs{\tau}) \end{array}\right]~, \quad \forall \bs{\tau} \in ([0,p])^n~,
\end{align}
where $\cor_1$ and $\cor_2$ are the $p$-periodic covariance functions defining (the restrictions of) $\{\Xi_1(btX), btX \in \field{R}^n\}$ and $\{\Xi_2(btX), btX \in \field{R}^n\}$ (to $\body$), respectively.
For the sake of illustration, we assume similar covariance functions for the two Gaussian components, setting $\cor_1 = \cor_2 = \cor$,
and we consider the case of a separable covariance function in $\field{R}^2$:
\begin{align}
  \cor(\bs{\tau}) = \cor_1(\tau_1) \times \cor_2(\tau_2)~. \label{eq: composite covariance function}
\end{align}
We further assume that $\cor_1$ and $\cor_2$ only differ in the choice of hyperparameters, so that the above equation can be written, using an abuse of notation, as
\begin{align}
  \cor(\bs{\tau}) = \cor(\tau_1;L_1) \times \cor(\tau_2;L_2)~,
\end{align}
where $L_1 > 0$ and $L_2 > 0$ are model parameters controlling the correlation ranges of the fields along the directions defined by the canonical basis in $\field{R}^2$. The univariate covariance function $\cor$ is assumed to take the generic form
\begin{align}
  \cor_\phi(\tau; L) & = \exp\left( -c \phi(\tau; L)\right )~, \quad \forall \tau \in [0,p]~, \label{eq: def-generic-cov} 
\end{align}
where $c$ is a positive constant (that depends on both $\phi$ and $L$) to be defined and $\phi$ is a $p$-periodic function taken as
\begin{align}
  \phi(\tau; L) = \dfrac{|\sin\left( \pi\tau/p \right)|}{L} \quad \text{(Periodic Exponential, PE),} \label{eq: def-exp}
\end{align}
or
\begin{align}
  \phi(\tau; L) = \dfrac{\sin\left( \pi\tau/p \right)^2}{L^2} \quad \text{(Periodic Squared-Exponential, PSE)}~. \label{eq: def-sexp}
\end{align}
The subscript in the notation $\cor_\phi$ underlines the choice of a particular periodic function. The two functions defined by \eqref{eq: def-exp} and \eqref{eq: def-sexp} are specifically chosen so as to investigate the impact of sample path regularity on fracture simulation outcomes. In particular, and while both functions yield random fields of fracture properties that are mean-square continuous, the periodic squared-exponential covariance function obtained by combining \eqref{eq: def-generic-cov} and \eqref{eq: def-sexp} leads to mean-square differentiable fields which exhibit ``smooth'' realizations. On the contrary, the periodic exponential function defined by considering \eqref{eq: def-exp} yields random fields that are not mean-square differentiable and thus have much rougher realizations.

In order to ensure meaningful comparison between the covariance models, the normalization constants are determined such that the spatial correlation length along the direction of one main direction is equal to $L$, that is:
\begin{align}
  \int_0^{p/2} |\cor_\phi(\tau; L)| \mathrm{d}\tau = L~.
\end{align}
For later use, we introduce the normalized correlation length $L^* = L/p$. For the squared exponential model, the constant $c^{}_\text{PSE}$ is hence required to satisfy the nonlinear equation
\begin{align}
  \dfrac{1}{2}p\exp\left( -\dfrac{c^{}_\text{PSE}}{2L^2} \right)I_0(\dfrac{c^{}_\text{PSE}}{2L^2}) = L~, \label{eq: pse normalization}
\end{align}
where $I_0$ is the 0th order modified Bessel's function, whereas the constant $c^{}_\text{PE}$ associated with the exponential covariance kernel must satisfy
\begin{align}
  \dfrac{1}{2}p\left[ I_0\left(\dfrac{c^{}_\text{PE}}{L}\right) - S_0\left(\dfrac{c^{}_\text{PE}}{L}\right) \right] = L~, \label{eq: se normalization}
\end{align}
where $S_0$ is the 0th order modified Struve's function.
In this work, these equations are solved numerically using a Newton-Raphson solver.

It should be noticed that the covariance function $\bs{\tau} \mapsto [\cor_{\bs{\Upsilon}}(\bs{\tau})]$ defining the restriction of the Gaussian random field $\{\bs{\Upsilon}(btX), btX \in \field{R}^n\}$ to $\body$ is given by
\begin{align}
  [\cor_{\bs{\Upsilon}}(\bs{\tau})] = \left[\begin{array}{cc}\cor_1(\bs{\tau}) & \rho \cor_1(\bs{\tau}) \\\rho \cor_1(\bs{\tau}) & \rho^2 \cor_1(\bs{\tau}) + (1-\rho^2) \cor_2(\bs{\tau}) \end{array}\right]~, \quad \forall \bs{\tau} \in ([0,p])^n~,
\end{align}
which shows the role played by the parameter $\rho$ on the covariance function of $\{P_2(btX), btX \in \body\}$ (after the action of $\fspace{T}^*$).

%%%%%%%%%%%%%%%%%%%%%%%%%%%%%%%%%%%%%%%%%%%%%%%%%%%%%%%%%%%%%%%%%%%%%%%%%%%%%%%%%%%%%%%%%%%
%%%%%%%%%%%%%%%%%%%%%%%%%%%%%%%%%%%%%%%%%%%%%%%%%%%%%%%%%%%%%%%%%%%%%%%%%%%%%%%%%%%%%%%%%%%
%%%%%%%%%%%%%%%%%%%%%%%%%%%%%%%%%%%%%%%%%%%%%%%%%%%%%%%%%%%%%%%%%%%%%%%%%%%%%%%%%%%%%%%%%%%
%%%%%%%%%%%%%%%%%%%%%%%%%%%%%%%%%%%%%%%%%%%%%%%%%%%%%%%%%%%%%%%%%%%%%%%%%%%%%%%%%%%%%%%%%%%
%%%%%%%%%%%%%%%%%%%%%%%%%%%%%%%%%%%%%%%%%%%%%%%%%%%%%%%%%%%%%%%%%%%%%%%%%%%%%%%%%%%%%%%%%%%
%%%%%%%%%%%%%%%%%%%%%%%%%%%%%%%%%%%%%%%%%%%%%%%%%%%%%%%%%%%%%%%%%%%%%%%%%%%%%%%%%%%%%%%%%%%
%%%%%%%%%%%%%%%%%%%%%%%%%%%%%%%%%%%%%%%%%%%%%%%%%%%%%%%%%%%%%%%%%%%%%%%%%%%%%%%%%%%%%%%%%%%
%%%%%%%%%%%%%%%%%%%%%%%%%%%%%%%%%%%%%%%%%%%%%%%%%%%%%%%%%%%%%%%%%%%%%%%%%%%%%%%%%%%%%%%%%%%
%%%%%%%%%%%%%%%%%%%%%%%%%%%%%%%%%%%%%%%%%%%%%%%%%%%%%%%%%%%%%%%%%%%%%%%%%%%%%%%%%%%%%%%%%%%
%%%%%%%%%%%%%%%%%%%%%%%%%%%%%%%%%%%%%%%%%%%%%%%%%%%%%%%%%%%%%%%%%%%%%%%%%%%%%%%%%%%%%%%%%%%
\subsubsection{Fundamental Properties of the Random Field of Fracture Properties}

Based on the construction proposed in Sections \ref{section: Chapter4/theory/stochastic/non_gaussian_model} and \ref{section: Chapter4/theory/stochastic/gaussian_model}, the non-Gaussian random field $\{\bs{P}(btX), btX \in \body\}$ modeling the fracture properties satisfies the following properties:
\begin{enumerate}
  \item $P_i(btX) > 0$ for all $btX \in \body$ and $i \in \{1,2\}$, almost surely.
  \item $\{\bs{P}(btX), btX \in \body\}$ is of second order, $\field{E}\{\|\bs{P}(btX)\|^2\} < + \infty$ $\forall btX \in \body$.
  \item The field satisfies $\bs{P}((0, x_2)) = \bs{P}((p, x_2))$ and $\bs{P}((x_1, 0)) = \bs{P}((x_1, p))$, almost surely.
  \item The mean function $btX \mapsto \underline{\bs{p}}(btX) = \field{E}\{\bs{P}(btX)\}$ is given by $\underline{\bs{p}}(btX) = \underline{\bs{p}} = (\underline{p}_1, \underline{p}_2)$ for all $btX \in \body$.
  \item The covariance matrix at any location $btX \in \body$ reads as
        \begin{align}
          \field{E}\{(\bs{P}(btX) - \underline{\bs{p}}) \otimes (\bs{P}(btX) - \underline{\bs{p}})\} = \left[\begin{array}{cc} \underline{p}_1^2 \delta_1^2 &  \underline{p}_1 \underline{p}_2 \delta_1  \delta_2 \rho  \\[1mm] \underline{p}_1 \underline{p}_2 \delta_1  \delta_2 \rho & \underline{p}_2^2 \delta_2^2 \end{array}\right]~.
        \end{align}
\end{enumerate}
It can also be deduced that the random field $\{\bs{P}(btX), btX \in \body\}$ is mean-square continuous for the PE and PSE correlation functions introduced in Section \ref{section: Chapter4/theory/stochastic/gaussian_model}. Finally, the field is mean-square differentiable for the PSE covariance function.

Recall that in the properties above, $\underline{p}_1$ and $\underline{p}_2$ correspond to the desired mean values for the fracture toughness $\Gc$ and energy threshold $\psi_c$, and $\delta_1$ and $\delta_2$ are the coefficients of variation that control the statistical dispersions of these parameters. The parameter $\rho$ measures the level of correlation between the fracture properties.

%%%%%%%%%%%%%%%%%%%%%%%%%%%%%%%%%%%%%%%%%%%%%%%%%%%%%%%%%%%%%%%%%%%%%%%%%%%%%%%%%%%%%%%%%%%
%%%%%%%%%%%%%%%%%%%%%%%%%%%%%%%%%%%%%%%%%%%%%%%%%%%%%%%%%%%%%%%%%%%%%%%%%%%%%%%%%%%%%%%%%%%
%%%%%%%%%%%%%%%%%%%%%%%%%%%%%%%%%%%%%%%%%%%%%%%%%%%%%%%%%%%%%%%%%%%%%%%%%%%%%%%%%%%%%%%%%%%
%%%%%%%%%%%%%%%%%%%%%%%%%%%%%%%%%%%%%%%%%%%%%%%%%%%%%%%%%%%%%%%%%%%%%%%%%%%%%%%%%%%%%%%%%%%
%%%%%%%%%%%%%%%%%%%%%%%%%%%%%%%%%%%%%%%%%%%%%%%%%%%%%%%%%%%%%%%%%%%%%%%%%%%%%%%%%%%%%%%%%%%
%%%%%%%%%%%%%%%%%%%%%%%%%%%%%%%%%%%%%%%%%%%%%%%%%%%%%%%%%%%%%%%%%%%%%%%%%%%%%%%%%%%%%%%%%%%
%%%%%%%%%%%%%%%%%%%%%%%%%%%%%%%%%%%%%%%%%%%%%%%%%%%%%%%%%%%%%%%%%%%%%%%%%%%%%%%%%%%%%%%%%%%
%%%%%%%%%%%%%%%%%%%%%%%%%%%%%%%%%%%%%%%%%%%%%%%%%%%%%%%%%%%%%%%%%%%%%%%%%%%%%%%%%%%%%%%%%%%
%%%%%%%%%%%%%%%%%%%%%%%%%%%%%%%%%%%%%%%%%%%%%%%%%%%%%%%%%%%%%%%%%%%%%%%%%%%%%%%%%%%%%%%%%%%
%%%%%%%%%%%%%%%%%%%%%%%%%%%%%%%%%%%%%%%%%%%%%%%%%%%%%%%%%%%%%%%%%%%%%%%%%%%%%%%%%%%%%%%%%%%
\subsubsection{Stochastic Simulation Aspects}

Simulating realizations of the random field of fracture properties requires (i) drawing realizations of the underlying Gaussian random fields defined in \Cref{section: Chapter4/theory/stochastic/gaussian_model}, and (ii) evaluating the nonlinear transformations defined by \eqref{eq: def-P1} and \eqref{eq: def-P2}. Routines to compute the latter mappings are readily available in many scientific computing environments. Regarding the generation of the Gaussian fields, we presently resort to truncated Karhunen-Lo\`eve expansions. Each (mean-square continuous) random field $\{\Xi_i(btX), btX \in \field{R}^n\}$, $i \in \{1,2\}$, is thus expanded as
\begin{align}
  \Xi_i(btX) = \sum\limits_{k=1}^{q_i} \sqrt{\lambda_k} \eta_k \varphi_k(btX)~,
\end{align}
where $\{\eta_k\}_{k \geqslant 1}$ is a set of independent normalized Gaussian random variables, and $\{\lambda_k\}_{k \geqslant 1}$ and $\{\varphi_k\}_{k \geqslant 1}$ are the nonnegative eigenvalues (ordered as a nonincreasing sequence) and the associated orthonormal eigenfunctions of the covariance function that satisfy the following integral equation (which is a Fredholm integral equation of the second kind):
\begin{align}
  \int\limits_\body \cor(btX,btX')\varphi_k(btX') \diff{\bfX}' = \lambda_k\varphi_k(btX)~, \quad \forall btX \in \body~, \label{eq: FIE2}
\end{align}
where $\cor$ is the covariance kernel introduced in Section \ref{section: Chapter4/theory/stochastic/gaussian_model} (note the abuse of notation, with $\bs{\tau} = btX - btX'$). For arbitrary covariance functions, this integral equation can be solved by using a standard Galerkin formulation \cite{GhanemSpanos,LMK2010}: given a finite set of basis functions $\{\phi_j\}_{j=1}^{N_\text{node}} \subset L_2(\body)$,
each eigenfunction is approximated as
\begin{align}
  \varphi_k(btX) \approx \sum_{j=1}^{N_{\text{node}}} \alpha_{j}^{(k)}\phi_j(btX)~. \label{eq: eigenfunction approximation}
\end{align}
Substituting this approximation in the integral equation and enforcing the residual to be orthogonal to $\mathspan(\{\phi_j\}_{j=1}^{N_\text{node}})$, we obtain
\begin{align}
  \sum_{j=1}^{N_{\text{node}}} \alpha_{j}^{(k)} \int\limits_\body \int\limits_\body \cor(btX,btX') \phi_j(btX')\phi_\ell(btX) \diff{\bfX}' \diff{\bfX} = \sum_{j=1}^{N_{\text{node}}} \alpha_{j}^{(k)} \int\limits_\body \lambda_k \phi_j(btX)\phi_\ell(btX) \diff{\bfX}~. \label{eq: gevp}
\end{align}
Considering the above equation for all eigenfunctions leads to the generalized eigenvalue problem $[K] [A] = [\Lambda] [M] [A]$, where $[A]_{j\ell} = \alpha_j^{(\ell)}$, $[\Lambda]_{j\ell} = \lambda_j \delta_{j_\ell}$ (with $\delta_{j_\ell}$ the Kronecker delta), and $[K]$ and $[M]$ are the matrices with entries
\begin{align}
  [K]_{j\ell} = \int\limits_\body \int\limits_\body \cor(btX,btX') \phi_j(btX')\phi_\ell(btX) \diff{\bfX}' \diff{\bfX}~, \quad [M]_{j\ell} = \int\limits_\body \phi_j(btX)\phi_\ell(btX) \diff{\bfX}~.
\end{align}
For each Gaussian field $\{\Xi_i(btX), btX \in \field{R}^n\}$, the truncation order $q_i \leqslant N_{\text{node}}$ can be determined by analyzing the convergence of the error function
\begin{align}
  \varepsilon_{\text{KL}}(q) = 1-\left( \sum\limits_{k=1}^q \lambda_k \middle/ \sum\limits_{k=1}^{N_\text{node}} \lambda_k \right)~, \label{eq: truncation}
\end{align}
and by selecting $q_i$ such that $\varepsilon_{\text{KL}}(q_i) \leqslant \varepsilon_0$ for some given tolerance $0 < \varepsilon_0 \ll 1$.
