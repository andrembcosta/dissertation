\section{Extension to non-planar fractures}
\label{section: Chapter4/nonplanar}

Extending the multi-resolution approach presented herein to non-planar fractures is definitely the most natural follow-up task. However, this extension requires additional challenges to be addressed. In the last Section, a skeleton of a possible algorithm is described. A simplified nonplanar fracture problem in 2D is used to test it. Although the result is favorable, the extension of this algorithm to full 3D problems requires additional modifications whose robustness are unclear. Another possible direction of work is then proposed, but left for future work.

\subsection{Overview of a propagation algorithm for non-planar fractures}\label{basicNonplanarAlgo}

Algorithm \ref{fig:MR_planar_algo} showed how to use the multi-resolution method to treat planar fractures in 3D. To extend it to non-planar cases, certain parts must be modified. 

\begin{itemize}
    \item First, the algorithm for tracking the damage in the front elements described in subsection \ref{frontTrackingAlgo} must be modified to read a nonplanar damage field and assess if the element is fully fractured. 
    
    \item The update of the geometry becomes more complex for the same reason. Given a nonplanar damage field in an element, how to insert a planar\footnote{While it is possible to develop embedded methods that allow for curved cuts inside elements, those are rarely used in practice due to the complex integration techiniques needed. Because of that, in the current work, while strong assumptions on the embedded scheme are avoided, the focus is on first-order (i.e planar cuts) methods.} cut in the mesh that represents it well?
    
    \item Finally, while for planar cuts the continuity of the fracture surface is always guaranteed, in nonplanar case, local (element-wise) approximations of the damage field with discrete cuts may lead to a discontinuous fracture surfaces.
\end{itemize}

To address the first bullet point, one can start with the topological criterion proposed in \ref{frontTrackingAlgo}, where the number of edges and faces with damage above a threshold is used. In general, if the damage band is small compared to the global element size, there is likely no ambiguity in the identification of cutting points. However, if the regularization length or the mesh size in the local problem are not very small, some difficulties may arise. For example, a wide damage band approaching the element diagonally may cut two adjacent edges as shown in Figure \ref{fig:pathological_case}. This leads to a configuration where the number of cut edges and faces is large albeit the actual global element is not yet fractured. 

\begin{figure}
    \centering
    \includegraphics[width=\linewidth]{Chapter4/figures/nonplanar/pathology.png}
    \caption{This figure shows a pathological case for the topological criterion proposed in \ref{frontTrackingAlgo}. The element on the right is only partially damaged, however, the counts of cut edges and faces are higher than the ones for the element on the left, which is indeed fully cut.}
    \label{fig:pathological_case}
\end{figure}%

Overcoming this limitation likely requires the use of information about the damage in neighboring elements. One approach, used in the example problem shown next is to keep the topological criterion the same, but introduce a check on the cutting angle that is obtained after each element is cut. This check consists of comparing the normal to the new cut with the neighboring fractured elements. If the normals are too different (i.e $\textbf{n}\cdot\textbf{n}_{neighbor} << 1$), then, this newly insert cut is assumed to be due to a pathological case and then removed. The respective element is only considered for cutting again when the number of damaged faces increase.

This heuristic relies on the assumption that fractures do not kink strongly over element boundaries, which is only true if the fracture curves have a large radius of curvature compared to the global element size. 

Now, in terms of the second bullet point, regarding the extraction of the cutting plane from a damage field, an optimization algorithm such as \cite{geelen2018optimization} can be employed. In a nutshell, this method searchs over the space of planes embedded in an element and constructs an idealized damage profile around this plane, using the optimal profile given by $d(x) = (1-\text{dist}(x)/2\ell)^2$, where $\text{dist}(x)$ denotes the distance from $x$ to the plane. These idealized profiles are compared to the actual damage field and the one which is closest in the $L_2$ sense is assumed to be the correct cutting plane for that damage field.

However, integrating this optimization approach with the angle check described above introduces another issue. That heuristic works under the assumption that an attempted cut on an element which is not yet fractured will lead to a cutting plane that does not conform well to its neighbors. If an optimization approach is employed, that might not always be true. For example, in the case shown in Figure \ref{fig:pathological_case}, the optimization approach can find the cutting plane induced by the partial fracture which will conform well to the existing cuts, even though the element is only partially cut.

So, in order to use the angle check, instead of applying the optimization method and then comparing the calculated plane with its neighbors, a crude pre-cut is employed. The idea of these pre-cuts is to rely only on the relative position of the cut faces identified by the topological criterion. This is done to ensure that, if one is dealing with a pathological case for the topological criterion, the pre-cut will indicate that and fail the angle check. In a 2D scenario, these pre-cuts can be obtained for example by connecting the midpoints of the fractured faces. In 3D, the mapping becomes a bit more complex. Figure \ref{fig:pre_cuts} shows the basic pre-cut types in 3D. 

\begin{figure}[h]
    \centering
    \includegraphics[width=0.8\linewidth]{Chapter4/figures/nonplanar/pre_cuts.png}
    \caption{Pre-cut basic shapes.}
    \label{fig:pre_cuts}
\end{figure}


Even with these modifications, the third and last bullet point, that concerns the continuity of the fracture surface must be addressed. For general embedded fracture approaches, continuity of the fracture surface is a necessary feature for the construction of the approximation space for the displacement field. This imposes serious complications to any type of algorithm that tries to construct surface increments on an element to element basis such as the one used here. However, the EFEM/EDFM approach used in this work and first proposed in \cite{cusini2021simulation} does offer a possible way to handle this issue.

By construction, the EFEM method does not assume continuity of the displacement field across element boundaries. In fact, the enrichment employed to represent the displacement jump does not interact with the neighboring elements and the final aperture final might indeed be discontinuous. Still, the method does converge to the continuous solution in space. Figure XX, from \cite{cusini2021simulation} shows a typical aperture field computed with EFEM. This flexibility allows one to prescribe a fracture field that may have gaps in its geometry and still compute an accurate aperture field. 

On the other hand, for the computation of the flow problem, some sense of continuity is still required to calculate fluxes from a fracture cell to its neighbors. In fact, two compute the flux between two embedded cells, their common edge is used. That is, if one obtains a discontinuous fracture surface, there will be neighboring cells that do not share and edge and therefore, there won't be flow between them. 

Luckly, this issue can be circuvent by noting that the damage intersection on any edge is the same for all elements that contain that edge. Therefore, instead of computing the fluxes directly from the common edge of neighboring fracture cells, one can store the intersections of the damage field with the global element edges and use them to build segments in between two neighboring cells. These segments can them be used to compute the fluxes, ensuring that even neighboring cells that happen to be on a surface discontinuity still get a reasonable flux. The drawback of this approach is its obvious dependence on the EFEM/EDFM scheme. It is unclear whether such modifications can be done if one chooses to use a XFEM enrichment instead.

ADD FIGURE.

\begin{figure}[h]
    \centering
    \includegraphics[width=0.6\linewidth]{Chapter4/figures/nonplanar/nonplanar_flux.png}
    \caption{Flux in discontinuous fracture surfaces.}
    \label{fig:flux_calculations}
\end{figure}


While there appear to be answers to the three main challenges listed above, this work falls short of providing an example that shows all these modifications, mainly for a lack of time. Instead, only the first modification to the algorithm, which concerns the topological criterion and its enhancement with the angle check and pre-cuts is implemented and tested. The optimization scheme and the flux computations based on the damage interesections in the edges are left for future investigations. 

\begin{figure}[h]
    \centering
    \includegraphics[width=0.95\linewidth]{Chapter4/figures/nonplanar/nonplanar_algo.png}
    \caption{Algorithm modifications}
    \label{fig:nonplanar_algo}
\end{figure}


In the next section, the simplified algorithm described above is applied to a 2.5D problem where the crack path is curved. It serves to check the efficacy of the proposed changes in the process of identifying the crack path. The fluid flow and poroelastic effects are turned off for simplicity.

\subsection{Example problem}

This subsection presents a small 2D problem used to test the simplified algorithm presented in \ref{basicNonplanarAlgo}. This problem consists of an edge crack in a square plate, which is loaded in the top edge by an inclined traction as shown in Figure \ref{fig:nonplanar_schematic}. The material properties and geometric parameters are given in Table XX. All the fluid effects are removed, as the focus is only on the algorithm to track the fracture and insert the discrete cuts. The subdomain is set to be the entire domain, which simplifies the transfer of boundary conditions from the global to local problem. 

Despite the simplification in defining the cut direction, the discrete fracture is still capable of tracking the damage band quite well, as shown in Figure \ref{fig:crack_path}. Interestingly, this problem involves multiple cases of the damage band approaching the diagonal of an element, which constitutes a pathological case, described previously. This shows that the correction based on the neighboring cutting angles is indeed useful and necessary in this type of approach.

For a sake of comparison, this same problem was studied in \cite{giovanardi2017hybrid} and the crack path results are in good agreement with the one displayed in Figure \ref{fig:crack_path}. A relative straightforward improvement can be done with the introduction of an optimization method, such as \cite{geelen2018optimization} to produce cuts that better align with the damage band locally. However, the use of "crude" approximations is still important to test the angle criteria and prevent lighly damaged elements to be wrongly cut due to a pathological case, such as the one shown in Figure XX.

One the other hand, this example doesn't explore a lot of the difficulties that arise in full 3D problems. To begin, mapping damage on edges to the "crude" approximations shown in Figure XX becomes much more difficult in 3D. In addition to that, the angle criteria may work well in cracks that evolve progressively as the one in this example, however, it may get into a "locking" situation if two cracks are merging at an angle. This is, different neighbors may impose constraints in the cutting angle of an element that may be impossible to satisfy simultaneously. 

Overcoming these challegens may require a more severe reformulation of the algorithm used for planar fractures. One promising approach is to employ a global method of surface reconstruction after the solution of the phase-field problem. This is a research topic on its own, with important contributions developed in recent years, such as \cite{}. In \cite{}, the authors actually looked at surface reconstruction in a similar framework, to improve the simulation of plasticity using phase-field methods. They used the gradient of the damage to guide the identification of a discrete fracture surface embedded in a triangular mesh. REVIEW THIS. While the robustness of their reconstruction in the presence of crack merging and branching is still unclear, it certainly points to a promising direction that can be explored in the context of the current work in hydraulic fracture.

\begin{figure}[ht]
    \centering
    \begin{tikzpicture}
        \node {\pgfimage[interpolate=false,width=.4\textwidth]{/home/andre/projects/Dissertation/dissertation/Chapter4/figures/nonplanar/nonplanar_schematic.png}};
        \draw (-0.1\textwidth,0.04\textwidth) node {\large$a$};
        \draw (0.22\textwidth,0.0\textwidth) node {\large$L$};
        \draw (0.02\textwidth,0.23\textwidth) node {\large$\Delta\textbf{u}$};
    \end{tikzpicture}
    \caption{Edge crack spcimen.}
    \label{fig:nonplanar_schematic}
\end{figure}

\begin{figure}[h]
    % \centering
    \begin{subfigure}{.45\textwidth}
      \centering
      \includegraphics[width=0.765\linewidth]{Chapter4/figures/nonplanar/curved_crack_result_bianca.png}
      \caption{Crack path - Giovanardi et al. \cite{giovanardi2017hybrid}.}
      \label{fig:reference result}
    \end{subfigure}%
    \begin{subfigure}{.54\textwidth}
      \centering
      \includegraphics[width=0.83\linewidth]{Chapter4/figures/nonplanar/nonplanar_example.png}
      \caption{Crack path - current work}
      \label{fig:crack_path}
    \end{subfigure}%
      \caption{(a) Initial crack; (b) Final damage path.} 
      \label{fig:nonplanar_example}
\end{figure}