\documentclass[12pt]{article}
\usepackage{tabularx} 
\usepackage{amsmath}  
\usepackage{amsfonts}
\usepackage{authblk}
\usepackage{stmaryrd}
\usepackage{parskip}
\usepackage{floatrow}
\usepackage{subcaption}
\usepackage{graphicx}
\usepackage{amsmath}
\usepackage{mdframed}
\usepackage{mathtools}
\usepackage{amsfonts}
\usepackage{todonotes} 
\usepackage{amssymb}
\usepackage[section]{placeins}
\usepackage{placeins}
\usepackage{graphicx} 
\usepackage[margin=1.0in]{geometry} 
\usepackage[ruled,vlined]{algorithm2e}
\usepackage{cite} 
\usepackage[final]{hyperref} 
\hypersetup{
	colorlinks=true,       
	linkcolor=blue,        
	citecolor=blue,        
	filecolor=magenta,     
	urlcolor=black,   
}

\makeatletter
\AtBeginDocument{%
  \expandafter\renewcommand\expandafter\subsection\expandafter{%
    \expandafter\@fb@secFB\subsection
  }%
}
\makeatother
\renewcommand*{\Affilfont}{\itshape\small}

\begin{document}

\title{A Multi-Resolution Approach \\ to Hydraulic Fracture Simulation}

\author[a]{Andre Costa}
\author[b]{Matteo Cusini}
\author[c]{Tao Jin}
\author[b]{Randolph Settgast}
\author[a, *]{John E.\ Dolbow}

\affil[a]{Department of Mechanical Engineering and Material Science, Duke University, Durham, 27708, NC, USA} 

\affil[b]{Atmospheric, Earth and Energy Division, Lawrence Livermore National Laboratories, Livermore, 94550, CA, USA} 

\affil[c]{Department of Mechanical Engineering, University of Ottawa, Ottawa, K1N6N5, ON, Canada} 

\affil[*]{Corresponding author: John E.\ Dolbow, jdolbow@duke.edu}

\date{ }
\maketitle

\begin{abstract}
We present a multi-resolution approach for constructing model-based simulations of hydraulic fracturing, wherein flow through porous media is coupled with fluid-driven fracture.  The approach consists of a hybrid scheme that couples a discrete crack representation in a global domain to a phase-field representation in a local subdomain near the crack tip. 
The multi-resolution approach addresses issues such as the computational expense of accurate hydraulic fracture simulations and the difficulties associated with reconstructing crack apertures from diffuse fracture representations. 
In the global domain, a coupled system of equations for displacements and pressures is considered. The crack geometry is assumed to be fixed and the displacement field is enriched with discontinuous functions. Around the crack tips in the local subdomains, phase-field sub-problems are instantiated on the fly to propagate fractures in arbitrary, mesh independent directions. The governing equations and fields in the global and local domains are approximated using a combination of finite-volume and finite element discretizations.  The efficacy of the method is illustrated through various benchmark problems in hydraulic fracturing, as well as a new study of fluid-driven crack growth around a stiff inclusion.  

\end{abstract}

\input{intro.tex}
\section{Model formulation}\label{formulation}

In this section, we describe the models that are used to develop our multi-resolution scheme.  We start by presenting the governing equations to model hydraulic fracture in poroelastic rock that forms the basis for the solver at the global scale.  We then describe a phase-field model for fracture that forms the basis for the solver used in local subdomains near the crack tips.

\subsection{Governing equations for hydraulic fracture}

\begin{figure}[h]
    \centering
    \includegraphics[width=15cm]{img/HF_potato_regular.png}
    \caption{Schematic of a poroelastic rock with a fracture, inspired by \cite{landis2016birs}.}
    \label{fig:potato_rock}
\end{figure}

We consider a model that couples flow and elastic deformation in a porous media with evolving fracture surfaces.  Consider the domain $\Omega$ consisting of a porous rock, that is fully saturated with a single-phase Newtonian fluid. The external boundary is composed of both traction $\partial \Omega_t$ and displacement $\partial \Omega_u$ surfaces, viz.\  $\partial \Omega = \partial \Omega_t \cup \partial \Omega_u$. 

The domain contains fractures $\Gamma$, as shown in Figure \ref{fig:potato_rock}.  For simplicity, we assume quasi-static conditions and small-strain kinematics.   Neglecting inertial effects, the balance of linear momentum reads
\begin{equation}\label{linear momentum balance}
    \nabla \cdot \boldsymbol\sigma + \textbf{b} = \boldsymbol 0\ \text{on } \Omega\setminus\Gamma,
\end{equation}
where $\boldsymbol\sigma$ denotes the total Cauchy stress and $\textbf{b}$ the body force. The mechanical boundary conditions are given by
\begin{equation}\label{traction bc}
    \boldsymbol \sigma \cdot \textbf{n} = \textbf{t} \text{ on } \partial \Omega_t,
\end{equation}
\begin{equation}\label{displacement bc}
    \textbf{u} = \overline{\textbf{u}} \text{ on } \partial \Omega_u,
\end{equation}

where $\textbf{n}$ denotes the normal direction at any point on $\partial\Omega$,  $\textbf{t}$ are the applied tractions, and $\overline{\textbf{u}}$ denotes the prescribed displacements.

Fluid flow within the cracks gives rise to pressure loads on the crack surfaces, translating into the boundary condition 
\begin{equation}\label{fracture boundary condition}
    \boldsymbol\sigma^+\cdot \textbf{n}_{\Gamma} = -\boldsymbol\sigma^-\cdot \textbf{n}_{\Gamma} = -p_f\textbf{n}_{\Gamma} \text{ on } \Gamma,
\end{equation}
where $p_f$ is the pressure inside the crack and $\textbf{n}_{\Gamma}$ is the normal vector to $\Gamma$ at any point.  In this work, we assume that fractures are open and neglect contact conditions on crack faces.  For additional considerations to account for closed fractures, see the model described in Cusini et al.\cite{cusini2021simulation}.

In terms of the fluid flow, the fluid velocity $\textbf{v}_m$ in the matrix is governed by the mass balance 
\begin{equation}\label{mass balance matrix}
    \dfrac{\partial(\rho \phi)}{\partial t} + \nabla \cdot (\rho \textbf{v}_m) = Q_m + Q_{mf}\ \text{on } \Omega\setminus\Gamma.
\end{equation}
where $\rho$ denotes the fluid density and $\phi$ is the porosity, and $Q_m$ is a prescribed source term in the matrix. The source term $Q_{mf}$ accounts for the exchange of fluid between the matrix and the fractures.
The boundary is partitioned into pressure $\partial \Omega_p$ and flux $\partial\Omega_q$ portions, such that $\partial \Omega = \partial \Omega_p \cup \partial \Omega_q$, and the following boundary conditions are applied,

\begin{equation}\label{flux bc}
    \rho \textbf{v}_m \cdot \textbf{n} = \textbf{q} \text{ on } \partial \Omega_q,
\end{equation}

\begin{equation}\label{pressure bc}
    p_m = \overline{p}_m \text{ on } \partial \Omega_p,
\end{equation}

where $p_m$ denotes the pore-pressure and  $\overline{p}_m$ is a prescribed pressure.

In a comparable manner, the fluid velocity $\textbf{v}_f$ within fractures is governed by the mass balance
\begin{equation}\label{mass balance fracture}
    \dfrac{\partial(\rho w)}{\partial t} + \nabla_{\Gamma} \cdot (\rho w \textbf{v}_f) = Q_f + Q_{fm}\ \text{on } \Gamma,
\end{equation}
where $w$ denotes the normal fracture aperture, $Q_f$ is a prescribed source term within the fracture, and $Q_{fm}$ accounts for the exchange of fluid between the fracture and the matrix\footnote{the flux interactions $Q_{fm}$ and $Q_{mf}$ are modeled as in classical well models, following \cite{hajibeygi2011hierarchical}. This ensures the balance of mass between the fractures and matrix $\int_V Q_{mf} dV = -\int_{\Gamma}Q_{fm}d\Gamma$.}. In the above, $\nabla_{\Gamma}$ indicates a gradient operator taken on the lower dimensional manifold  $\Gamma$. Boundary conditions similar to \eqref{flux bc} and \eqref{pressure bc} can also be applied.

Constitutive relationships are required to close the system and tie the stresses to the displacements $\textbf{u}$ and the fluid velocities to the pressures. In particular, we adopt the basic assumptions of Biot's theory of poroelasticity \cite{biot1941general}, Darcy's law for the flow in the matrix, and a lubrication theory approximation for the flow in fractures. This gives rise to the following set of constitutive relationships for the stress, porosity, and velocities:

\begin{equation}\label{biot law}
    \boldsymbol \sigma = \mathbb{C}:\boldsymbol\epsilon(\textbf{u}) - \alpha p_m \mathbb{I} \ \text{on } \Omega\setminus\Gamma,
\end{equation}

\begin{equation}\label{linear porosity}
    \dot{\phi} = \alpha \nabla \cdot \dot{\textbf{u}} + \dfrac{\dot{p_m}}{N} \ \text{on } \Omega\setminus\Gamma,
\end{equation}

\begin{equation}\label{darcy law}
    \textbf{v}_m = -\dfrac{\kappa}{\mu}\nabla p_m \ \text{on } \Omega\setminus\Gamma,
\end{equation}

\begin{equation}\label{cubic law}
    \textbf{v}_f = -\dfrac{w^2}{12\mu}\nabla_{\Gamma} p_f \ \text{on } \Gamma.
\end{equation}

In the above, $p_m$ is the pore-pressure, $\mathbb{C}$ is the fourth-order isotropic tensor of drained elastic moduli,  $\boldsymbol\epsilon(\textbf{u})$ is the mechanical strain, $\alpha$ is the Biot coefficient, and $\mathbb{I}$ is the second-order identity tensor. The temporal evolution of the porosity is governed by the rate of dilatation and time rate of change in the pore pressure, as modulated by the modulus $N$.  The fluid velocity in the matrix is related to the gradient of the pressure through the ratio of the intrinsic permeability $\kappa$ to the viscosity $\mu$.    

The fluid is assumed to be linearly compressible.  For both the fluid in the fractures and the matrix, this implies that the density is updated from its reference value $\rho_{ref}$ based on the change in pressure according to
\begin{equation}\label{poorly compressibility}
    \rho = \rho_{ref} \left(1 + \dfrac{p  - p_{ref}}{K_F}\right) \ \text{on } \Omega,
\end{equation}
where $p_{ref}$ denotes a reference value for the pressure, and $K_F$ is the fluid bulk modulus. 

Finally, the initial conditions for the displacements and pressures are given by

\begin{equation}\label{u_ic}
    \textbf{u}(\textbf{x},0) = \textbf{u}^0  \ \text{on } \Omega\setminus\Gamma,
\end{equation}

\begin{equation}\label{pm_ic}
    p_m(\textbf{x},0)= p^0_m  \ \text{on } \Omega\setminus\Gamma,
\end{equation}

\begin{equation}\label{pf_ic}
    p_f(\textbf{x},0) = p^0_f  \ \text{on } \Gamma.
\end{equation}

For a given crack geometry, the combination of equations \eqref{linear momentum balance}, \eqref{mass balance matrix} and \eqref{mass balance fracture}, with constitutive assumptions \eqref{biot law} - \eqref{poorly compressibility}, boundary conditions \eqref{traction bc}-\eqref{fracture boundary condition},\eqref{flux bc},\eqref{pressure bc} and initial conditions \eqref{u_ic}-\eqref{pf_ic} leads to a system of equations whose solution can be approximated by many different numerical methods.
What remains is a model to describe the evolution of the crack geometry.

In the context of standard sharp interface approaches for hydraulic fracture, the evolution of the crack geometry is typically governed by a set of criteria that dictate whether or not a crack front extends and, if so, in what orientation.  For crack extension, a standard approach is to adopt Griffith's criteria \cite{griffith1921vi}, which states that crack propagation should occur when the energy release rate $G$ reaches the critical value $G_c$ for the material, i.e.\ $G \le G_c$.   In terms of changes to orientation, several different criteria are typically employed, such as the maximum hoop stress criteria \cite{10.1115/1.3656897, williams1972fracture, finnie1973note} or the maximum energy release rate condition \cite{ewing1976further, cotterell1965brittle, hussain1974strain}. Examples of works from the hydraulic fracture field employing such criteria include \cite{he2018modeling, jang2020analysis, grossman2019algorithm}. 

Although such methods have seen some success in simulating the propagation of hydraulically-driven cracks, even in three dimensions \cite{gupta2014simulation, gupta2018coupled,shauer2022three}, they struggle as crack evolution becomes sufficiently complex.  By contrast, regularized methods have seen far more success in treating complex geometric evolution. In the next subsection, we present the governing equations for a phase-field method for fracture, a regularized approach for representing fractures and their evolution.  Phase-field for fracture models generally start from a single energetic postulate that generalizes Griffith's criteria, and which is able to describe the entire fracture propagation process.

\subsection{The phase-field method for fracture}
\label{sec:pfm-fracture}

The phase-field method for fracture started as an approximation \cite{bourdin2000numerical} to the variational approach for fracture by Francfort and Marigo \cite{francfort1998revisiting} for the quasi-static propagation of fracture in brittle materials. This model essentially states that a crack should evolve in a way that minimizes a total energy functional, among all admissible states, which are those that contain the current crack set (so that no healing is possible). For the method adopted in this work, we follow the work of Chukwudozie et al.\ \cite{chukwudozie2019variational} and associate the following total energy to a crack configuration $\Gamma$ in a poroelastic brittle solid $\Omega$:
% \begin{multline}\label{variational approach to fracture}
%     \mathcal{E}(\textbf{u}, p_m, p_f, \Gamma) = \int\limits_{\Omega \setminus \Gamma}W(\boldsymbol{\epsilon}(\textbf{u}), p_m)d\Omega - \int\limits_{\partial\Omega_N} \textbf{t} \cdot \textbf{u} ds
%     - \int\limits_{\Omega\setminus \Gamma} \textbf{b} \cdot \textbf{u} d\Omega \\
%     + \int\limits_{\Gamma}p_f \llbracket \textbf{u} \cdot \textbf{n}_{\Gamma} \rrbracket ds + G_c \mathcal{H}^{n-1}(\Gamma),
% \end{multline}
where the tractions applied to the boundary are denoted by $\textbf{t}$, the normal to the crack is denoted by $\textbf{n}_{\Gamma}$ and $\mathcal{H}^{n-1}(\Gamma)$ is the $n-1$ dimensional Hausdorff measure of $\Gamma$. 

The strain energy density $W(\boldsymbol{\epsilon}(\textbf{u}), p_m)$ is postulated as,

\begin{equation}
    W(\boldsymbol{\epsilon}(\textbf{u}), p_m) = \dfrac{1}{2}\left( \boldsymbol\epsilon(\textbf{u}) - \dfrac{\alpha}{nK} p_m\mathbb{I}\right) : \mathbb{C} : \left( \boldsymbol\epsilon(\textbf{u}) - \dfrac{\alpha}{nK} p_m\mathbb{I}\right),
\end{equation}

with $K$ denoting the bulk modulus and $n$ the system's dimension (2 or 3). The phase-field regularization, based on the Ambrosio-Tortorelli \cite{ambrosio1990approximation} functional is then performed by the introduction of the damage parameter $d$ and the regularization length $\ell$,

\begin{multline}\label{Poroelastic PF funcional}
    \mathcal{E}_{\ell}(\textbf{u},d,p_m,p_f) = \int\limits_{\Omega }\widetilde{W}(\boldsymbol{\epsilon}(\textbf{u}), p_m, d)d\Omega - \int\limits_{\partial\Omega_N}\textbf{t} \cdot \textbf{u} ds
    - \int\limits_{\Omega} \textbf{b} \cdot \textbf{u} d\Omega
    + \int\limits_{\Omega}p_f \textbf{u} \cdot \nabla d d\Omega \\
    + \dfrac{G_c}{c_0}\int\limits_{\Omega }\left(\dfrac{\zeta(d)}{\ell} + \ell\nabla d\cdot\nabla d  \right)d\Omega,
\end{multline}

where the function $\zeta(d)$ is the local dissipation function, which is usually taken as $\zeta(d) = d$ or $d^2$. The constant $c_0$ is given by $c_0 = 4\int_0^1 \sqrt{\zeta(z)}dz$ and the regularized strain energy is defined by,

\begin{equation}\label{damaged strain energy}
    \widetilde{W}(\boldsymbol{\epsilon}(\textbf{u}), p_m, d) = \dfrac{1}{2}\left( (1-d)\boldsymbol\epsilon(\textbf{u}) - \dfrac{\alpha}{nK} p_m\mathbb{I}\right) : \mathbb{C} : \left( (1-d)\boldsymbol\epsilon(\textbf{u}) - \dfrac{\alpha}{nK} p_m\mathbb{I}\right),
\end{equation}

which is consistent with an assumption of damage arising in the sub pore scale \cite{chukwudozie2019variational}. Finally, our regularized crack evolution problem is then stated as a minimization principle for the functional $\mathcal{E}_{\ell}(\textbf{u},d,p_m,p_f)$, with respect to the variables $\textbf{u}$ and $d$, with the added condition that the damage process is irreversible and that the damage variable $d$ is bounded between the values of zero (undamaged state) and unity (fully broken state):
\begin{equation}\label{variational formulation of phase-field}
    \textbf{u}, d = \underset{\textbf{u},d}{{\operatorname{argmin}}} \ \mathcal{E}_{\ell}(\textbf{u},d,p_m,p_f)\ \text{,\ \   subject to } \dot{d} \ge 0, \text{ and } 0 \le d \le 1. 
\end{equation}

The following set of evolution equations can then be derived from the Karush–Kuhn–Tucker (KKT) \cite{karush1939minima, kuhn1951nonlinear} conditions:

\begin{equation}\label{basic u problem}
    \nabla \cdot \left( (1-d)^2\ \mathbb{C}:\boldsymbol\epsilon(\textbf{u}) -(1-d)\alpha p_m \mathbb{I}\right) + \textbf{b} = p_f\nabla d, 
\end{equation}

\begin{equation}\label{damage equation}
    S_d \coloneqq  \dfrac{2G_c\ell}{c_0}\Delta d - \dfrac{G_c}{c_0\ell}\zeta'(d)-g'(d)W(\boldsymbol{\epsilon}(\boldsymbol{\textbf{u}}),0) + \alpha p_m\nabla \cdot \textbf{u} + \nabla \cdot (p_f\textbf{u}) \le 0,
\end{equation}

\begin{equation}
    \dot{d} \ge 0,
\end{equation}

\begin{equation}
   0 \le d \le 1,
\end{equation}

\begin{equation}
    S_d\dot{d} = 0,
    \label{eq:ddot-strong}
\end{equation}

with boundary conditions, $\boldsymbol\sigma \cdot \textbf{n} = \textbf{t}$ on $\partial \Omega_N$ and $(2G_c\ell\nabla d +c_0p_f\textbf{u})\cdot \textbf{n} \ge 0$ on $\partial \Omega$. The damaged stress is defined as $\boldsymbol\sigma = (1-d)^2\ \mathbb{C}:\boldsymbol\epsilon(\textbf{u}) -(1-d)\alpha p_m \mathbb{I}$.

In some cases, it is useful to decompose the energy \eqref{damaged strain energy} into positive and negative parts, in order to provide asymmetry between tension and compression states. In most problems studied in this manuscript, we found this decomposition to be important, and, unless otherwise mentioned, the spectral decomposition introduced in Miehe et al.~\cite{miehe2010phase} is used. With this decomposition, the equations for the macro-scale force balance \eqref{basic u problem} and the micro-force balance \eqref{damage equation} are modified as

\begin{equation}\label{basic u problem spectral}\tag{22a}
    \nabla \cdot \left( (1-d)^2\ \mathbb{C}:\boldsymbol\epsilon^+(\textbf{u}) + \mathbb{C}:\boldsymbol\epsilon^-(\textbf{u})-(1-d)\alpha p_m \mathbb{I}\right) + \textbf{b} = p_f\nabla d, 
\end{equation}

\begin{equation}\label{damage equation spectral}\tag{23a}
    S_d \coloneqq  \dfrac{2G_c\ell}{c_0}\Delta d - \dfrac{G_c}{c_0\ell}\zeta'(d)-g'(d)W(\boldsymbol{\epsilon^+}(\boldsymbol{\textbf{u}}),0) + \alpha p_m\nabla \cdot \textbf{u} + \nabla \cdot (p_f\textbf{u}) \le 0,
\end{equation}

and the damaged stress becomes $\boldsymbol\sigma = (1-d)^2\ \mathbb{C}:\boldsymbol\epsilon^+(\textbf{u}) + \mathbb{C}:\boldsymbol\epsilon^-(\textbf{u})-(1-d)\alpha p_m \mathbb{I}$. In the above, $\boldsymbol\epsilon^+$ and $\boldsymbol\epsilon^-$ denote the positive and negative parts of the strain tensor as described in \cite{miehe2010phase}. 

Finally, to close this section, we note that the extent to which the solution to \eqref{variational formulation of phase-field} approaches a particular sharp model in hydraulic fracture in the limit as the regularization length $\ell\rightarrow 0$ has yet to be established.  Nevertheless, the particular phase-field model adopted here contains all of the requisite components to allow for a proof of concept of the multi-resolution approach.  





\section{Multi-resolution method }\label{numerics}

\subsection{General overview}

\begin{figure}[!htbp]
    \centering
    \includegraphics[width=\textwidth]{img/Section2/global_local_simple.png}
    \caption{Global domain $\Omega$ on the left and local domain $\Omega_L$ on the right. The surface corresponding to the local domain boundary $\partial\Omega_L$ is indicated within the global domain with the dashed red lines.  The local domain is magnified to highlight the phase-field representation of the global crack $\Gamma$.}
    \label{fig:section_2_figure}
\end{figure}

We propose a multi-resolution method to approximate the solution of the hydraulic fracture problem in porous media. It consists of coupling two problems between a global domain and a local subdomain, as shown in Figure~\ref{fig:section_2_figure}.   In the global domain problem, the governing equations \eqref{linear momentum balance}-\eqref{mass balance fracture} are discretized, and the crack is represented with a sharp geometry.  The crack geometry is assumed to be fixed during a solution step in the global domain, and relevant fields are calculated over the entire domain.  

By contrast, the local subdomain concerns only a portion of the entire domain, namely in the vicinity of crack tips.  It is encapsulated within the global domain as shown in Figure~\ref{fig:section_2_figure}. In the local subdomain, a discretization of the variational principle \eqref{variational formulation of phase-field} is used to simulate crack evolution.  During a solution step in the local subdomain, the pressure fields within the fracture and matrix are assumed to be fixed.  

The two problems are coupled in the following manner.  The displacement and pressure fields are extracted from a solution step in the global domain and passed to the local subproblem in different ways.  In particular, the global displacement fields are extracted along the surface $\partial\Omega_L$. These fields are then  applied as Dirichlet boundary conditions for the local subproblem.  The matrix pressure $p_m$ and pressure in the fracture $p_f$ are transfered as fields from the global domain to the local subdomain (see Section~\ref{sec:pressure_projection} for details), and assumed to be fixed for the local subproblem.   

The aforementioned operations provide everything that is needed for the local subproblem from the global domain.  Based on the imprint of the global crack geometry on the local subdomain, a regularized fracture surface is created (through an initial damage field) and then crack propagation is simulated in the local subdomain. Once an extension of the crack in a scale that can be represented in the global domain is identified, the sharp crack geometry in the global problem is updated accordingly.  

In principle, the aforementioned multi-resolution approach can be implemented using a number of different discretization methods in the global domain and the local subdomain.  In the sections that follow, we describe the particular choices used in this work as well as some important implementation details.  The method is described in a two-dimensional context, but many aspects can be readily extended to three-dimensional problems.  

\subsection{Global problem discretization}
\label{sec:global_disc}

The global problem in our multi-resolution approach encompasses the physics of fluid flow in both the fractures and the pore structure of our domain, as well as the deformation of the solid media. The set of governing equations consists of  \eqref{linear momentum balance}-\eqref{mass balance fracture}, combined with constitutive assumptions \eqref{biot law}-\eqref{poorly compressibility} and appropriate boundary conditions. 

In this work, we use the discretization method proposed by Cusini et al.\ \cite{cusini2021simulation} in the study of fluid flow in fractured porous media. 
The  domain $\Omega$ is partitioned with a  mesh $\mathcal{T}$. Then, the intersection of the fracture network $\Gamma$ with $\mathcal{T}$ defines the fracture triangulation $\mathcal{F}$. These meshes are then used to define discrete counterparts $\textbf{u}^h_G$, $p_m^h$ and $p_f^h$ of the unknown fields $\textbf{u}$, $p_m$ and $p_f$, as well as discrete approximations of equations  \eqref{linear momentum balance}, \eqref{mass balance matrix} and \eqref{mass balance fracture}.

A finite element approximation is constructed for the displacement field and employed in a standard Galerkin approximation to the global force balance \eqref{linear momentum balance}. The continuous part of the displacement field is approximated with a standard space $\mathcal{U}$  of 4-node bilinear shape functions $\{ {\boldsymbol{\eta}}_a \} $.  In the subset of elements that are ``cut" by the global crack geometry, a space $\mathcal{W}$ of discontinuous enrichment functions $\{\boldsymbol{\phi}_b\}$ is constructed using the formulation described in \cite{linder2007finite}.   

The full displacement field in the global problem is constructed using both continuous and discontinuous parts as
\begin{equation}
    \label{eq:displacement_approx}
    \textbf{u}^h_G = \underbrace{\sum_{a=1}^{n_u}u_a \boldsymbol{\eta}_a}_{\text{continuous part}} + \underbrace{\sum_{b=1}^{n_w}w_b\boldsymbol\phi_b}_{\text{discontinuous part}}, 
\end{equation}
where $\{u_a\}, \{ w_b\}$ are scalar degrees of freedom. 

The flow equations  \eqref{mass balance matrix} and \eqref{mass balance fracture} are discretized with a finite-volume method.   Piecewise-constant pressure fields are constructed for $p_m^h$ and $p_f^h$ over the matrix mesh $\mathcal{T}$ and the fracture mesh $\mathcal{F}$, respectively. Fluxes are computed using a two point flux approximation. The interaction between the flow in the fracture and matrix is effected via the embedded discrete fracture model (EDFM) \cite{lee2001hierarchical, hajibeygi2011hierarchical}. 

In terms of the temporal discretization, the parabolic nature of equations \eqref{mass balance matrix} and \eqref{mass balance fracture} gives rise to a stable time step for explicit schemes that scales with the cube of the mesh spacing \cite{adachi2007computer, lecampion2018numerical}. This upper bound is prohibitively small in most cases. Therefore, an implicit backward Euler method is used throughout. This gives rise to a fully-coupled system of nonlinear equations, whose solution is obtained with a Newton method, in a monolithic fashion. 

One limitation of the construction \eqref{eq:displacement_approx} is that the discontinuous enrichment functions are not capable of representing a crack tip that terminates inside of an element.  As such, any new extension of the crack geometry has to traverse from one side of a new element to another. The implementation of the fracture flow solver requires all cells to have non-zero volumes.  Accordingly, new fracture cells are assigned a  small aperture value ($w_0$). The total discrete aperture of the cell is thus given by $w_h = w_n + w_0$, where $w_n$ is the mechanical aperture that is consistent with the jump provided by the displacement field.  The minimum opening $w_0$ has been interpreted as a representation of the roughness of the fracture surfaces, providing a pathway for fluid flow even when the cracks are mechanically closed.  See, for example \cite{cusini2021simulation}.

\subsection{Local problem initialization} 

\subsubsection{Construction of subdomain, submesh and damage-fixed nodes}\label{subdomain_construction}

A local subdomain of size $L\times L$ is constructed by simply centering it on a global crack tip.  The size $L$ is selected to be an even integer multiplier of the global mesh spacing, leading to a square bounding box that conforms to the background mesh.  This choice is adopted for convenience in this work, although other constructions are possible, such as in  \cite{giovanardi2017hybrid}.

To obtain the submesh $\mathcal{T}_L$, we start by considering the restriction of the global triangulation $\mathcal{T}$ to the subdomain $\Omega_L$.  The mesh for the local subdomain is constructed by uniformly refining the set of elements in this restriction. This facilitates the transfer of nodal data from the global to the local problem.  The mesh size $h_{local}$ in the local subdomain is chosen to be sufficiently small to resolve the damage band, of size $\mathcal{O}(\ell)$. In this work, we use  $\ell / h_{local} \approx 4$.

\begin{figure}[h]
    \centering
    \includegraphics[width=\textwidth]{img/Section2/subset_kappa.png}
    \caption{Global domain $\Omega$ on the left and local domain $\Omega_L$ on the right. The nodes in the subset $\mathcal{K}$, where $d$ is set to $1$ are colored in yellow.}
    \label{fig:subset_kappa}
\end{figure} 

The crack is represented by prescribing $d = 1$ in a set $\mathcal{K} \in \mathcal{T}_L$ of nodes in the local subdomain, as shown in Figure \ref{fig:subset_kappa}. This set is constructed in two steps. First, all elements of $\mathcal{T}_L$ which are intersected by the global fracture triangulation $\mathcal{F}$ are identified.  Then, $\mathcal{K}$ is defined to be the set of all nodes that belong to any of these intersected elements.

\subsubsection{Transfer of global pressures to local mesh}
\label{sec:pressure_projection}

Due to the different levels of resolution between the global and local problems, we find it advantageous to transfer the pressure fields in a particular manner.  We note that in the global problem, the fracture pressures $p_f^h$ are available at the cell centers of 1D finite volumes, and the matrix pressures $p_m^h$ are available at the cell centers of 2D finite volumes.  These fields are transferred to quadrature points in the finite element mesh for the local subproblem using the transfer operators 
$\Pi^{\Omega_L}_f$ and $\Pi^{\Omega_L}_m$.

\begin{figure}[!htbp]
    \centering
    \includegraphics[width=6cm]{img/Section2/pressure_interpolation_blue_small.png}
    \caption{Illustration of global cells (in gray) in the neighborhood of a local element, indicated in pink. The matrix pressure at a quadrature point in the local element is calculated using an average of global pressures from neighboring cells $\{ i \}$, with weights corresponding to the distances $r_i$ between the quadrature point and the cell center.}
    \label{fig:pressure_interpolation}
\end{figure}

The operator $\Pi^{\Omega_L}_f$ that transfers the fracture pressure is straightforward, and inspired by a similar operation described in Santillán et al.\ \cite{santillan2018phase}. Given any point $ {\bf{x}} \in \Omega_L$, and a global fracture pressure field $p_f^h$, $\Pi^{\Omega_L}_fp_f^h({\bf{x}})$ is obtained by finding the closest cell of $\mathcal{F}$ to $x$ and taking the value of $p_f^h$ at this cell. More precisely,

\begin{equation}
    \Pi^{\Omega_L}_f p_f^h({\bf{x}}) = p_f^h(\underset{c \in \mathcal{F}}{\text{arg min }}\text{dist}(c,{\bf{x}} )).
\end{equation}

For the operator $\Pi^{\Omega_L}_m$ that transfers the matrix pressure field, we use an averaging procedure.  This has the effect of smoothing the matrix pressure field at the resolution of the local mesh.  At a quadrature point ${\bf{x}}_Q \in \Omega_L$, the matrix pressure in the local domain is obtained from a weighted average of the pressures in the global cells that surround the point.  Specifically, 
\begin{equation}
    \Pi^{\Omega_L}_m p_m^h({\bf{x}}_Q) = \dfrac{\sum_{i=1}^s r^{-1}_ip_m^h(c_i)}{\sum_{i=1}^s r^{-1}_i},
\end{equation}
where $r_i$ denotes the distance between the quadrature point location and the center of global cell $i$, as indicated in Figure~\ref{fig:pressure_interpolation}.  In the sum, all cells that neighbor the global cell containing quadrature point ${\bf{x}}_Q$ are used.  In the particular case when a quadrature point happens to reside in the center of the local element and $r_1 = 0$, the above sum is replaced with   
$ \Pi^{\Omega_L}_mp_m^h({\bf{x}}_Q) = p^h_m (c_1)$.

\subsection{Local problem discretization}

With the local subdomain properly identified and initialized, we now describe the additional steps to discretize the displacement and damage fields in the local subdomain and solve for their approximations.
The governing equations for the macro-force balance \eqref{basic u problem} and the damage evolution \eqref{damage equation ch3} are both treated with the finite element method.  The damage and displacement fields are both approximated using four-node bilinear quadrilateral elements. 

Let $\Omega_L$ denote the local domain. From the finite-element approximation $\textbf{u}_G^h$ computed in the global problem, we extract $\textbf{u}^h_G|_{\partial\Omega_L}$ and use it to constrain the displacements on the boundary of the subproblem \footnote{In the multi-resolution method of Muixi et al.\ \cite{muixi2021combined},  the displacement boundary conditions near the crack base were released. We did not find this to be necessary in our approach.}.  As such, the trial space $\boldsymbol{\mathcal{U}}^h$ is given by
\begin{equation}\label{disp subspace}
    \boldsymbol{\mathcal{U}}^h = \{ \textbf{u}_L^h \in H^1(\Omega_L)^n \mid \textbf{u}_L^h = \textbf{u}^h_G \text{ on } \partial\Omega_L \}.
\end{equation}

The function space $\mathcal{D}^h$ of admissible damage fields is given by 
\begin{equation}\label{damage subspace}
    \mathcal{D}^h = \{ d^h \in H^1(\Omega_L) \mid d^h = 1 \text{ on } \mathcal{K} \},
\end{equation}
where $\mathcal{K}$ denotes the set of damage-fixed nodes that correspond to the global crack at the beginning of the load step, as described in subsection~\ref{subdomain_construction}. 

The test spaces are given by $\boldsymbol{\mathcal{W}}^h = H_0^1(\Omega_L)^n$ for the displacements and $\mathcal{C}^h = \{ c^h \in H^1(\Omega_L) \mid c^h = 0 \text{ on } \mathcal{K} \}$ for the damage field. The Galerkin approximation to the problem on the local subdomain is then:
\medskip

\begin{mdframed}[
    frametitle={The spatially discrete form},
    frametitlebackgroundcolor=gray!20,
    backgroundcolor=gray!5,
    linewidth=0pt,
    nobreak=true
  ]
  Find $\textbf{u}_L^h \in \boldsymbol{\mathcal{U}}^h$ and $d^h \in \mathcal{D}^h$, such that $\forall \textbf{w}^h \in \boldsymbol{\mathcal{W}}^h$ and $\forall c^h \in \mathcal{C}^h$,
 
 \begin{align}
  \left( \nabla \textbf{w}^h, \boldsymbol\sigma_L^h \right) - \left( \textbf{w}^h, \Pi^{\Omega_L}_f p_f^h \nabla d^h \right) - \left( \textbf{w}^h, \textbf{b} \right) - \left< \textbf{w}^h, \textbf{t} \right>_{\partial_t\Omega} &= 0, \label{eq: semidiscrete momentum balance ch3}   \\
  \begin{split}
    \frac{2G_c^h\ell}{c_0}\left( \nabla c^h, \nabla d^h \right) + \frac{G_c^h}{c_0\ell}\left( c^h, \zeta'(d^h) \right) + \left( c^h, 2(d^h-1)W(\boldsymbol{\epsilon^+}(\textbf{u}_L^h),0) \right) \\
    + \left( c^h, \alpha \Pi^{\Omega_L}_m p_m^h\nabla \cdot \textbf{u}_L^h \right) + \left( \nabla c^h, \Pi^{\Omega_L}_f p_f^h\textbf{u}_L^h \right)  &= 0, \label{eq: semidiscrete damage evolution}    
  \end{split} 
\end{align}

\end{mdframed}
where we have used $(\cdot,\cdot)$ to denote the standard inner product in the $L^2(\Omega)$ space, and $\left< \cdot, \cdot \right>$ to denote its restriction on the boundary.   In the above,  $\boldsymbol\sigma^h_L$ denotes the discrete stress, given by 
\begin{equation}
    \boldsymbol\sigma^h_L = \left( (1-d^h)^2\ \mathbb{C}:\boldsymbol\epsilon(\textbf{u}_L^h) -(1-d^h)\alpha p_m^h \mathbb{I}\right).
\end{equation}

We employ an alternating minimization scheme to solve the coupled system of equations \eqref{eq: semidiscrete momentum balance ch3}-\eqref{eq: semidiscrete damage evolution}. Convergence is measured with respect to the $L_2$-norms of the change in the damage $d^h$ and  displacement fields $\textbf{u}_L^h$, using a relative tolerance of $10^{-4}$. With this approach, each of the equations becomes linear with respect to its primary variable, simplifying the solution process.

We note that \eqref{eq: semidiscrete damage evolution} does not explicitly include the irreversibility constraint  \eqref{eq:ddot-strong}. Within the multi-resolution scheme, the Dirichlet condition $d^h=1$ on the crack set $\mathcal{K}$ is sufficient to prevent the healing of the fracture surface relative to the initial global crack.  It does not, however, enforce irreversibility of the damage throughout the local subdomain, and it is possible that some crack healing could occur if regions are loaded and unloaded during a single time step.  
We did not find this to occur for the problems studied in this work, but it is possible that this could arise in more complex cases and the use of a more robust means of enforcing the constraint on the damage would be needed.  Options include, for example, the active set solver described in \cite{hu2020frictionless}.   

In terms of the local dissipation function $\zeta(d)$, in this work we use $\zeta(d) = d$ which corresponds to the AT-1 phase-field model of fracture.  This is selected due to the fact that it gives rise to a compactly supported damage field and a fully elastic stage prior to damage initiation \cite{pham2011gradient}.
When using the AT-1 model, it is necessary to explicitly enforce the constraint $d^h \ge 0$, as in some cases, this formulation can lead to negative damage values. In this work, this is effected  by invoking a lower threshold on the active part of the strain energy $W(\boldsymbol{\epsilon^+}(\textbf{u}_L^h),0) = 3G_c/8\ell$, as in \cite{miehe2016phase}. 

Finally, we note that phase-field models of fracture tend to give rise to a mesh and regularization length dependent critical fracture energy that is larger than $G_c$ \cite{bourdin2008variational}.  To account for this, we use a discrete value of $G_c^h$ that is obtained as a function of the local mesh spacing and regularization length, as
    \begin{equation}
        G_c^h = G_c \left( 1 + \dfrac{ h_{local} }{c_0\ell} \right)^{-1},
    \end{equation}
such that the effective critical fracture energy is very close to that of the material.  

\subsection{Crack propagation: translating local damage updates into global crack extensions}\label{propagation_step}


A key step of the algorithm concerns how changes to the damage in the local subdomain are translated into updates to the crack geometry in the global domain.  We provide details of our algorithm for this procedure here.  The scheme is presented in the context of a single crack tip that is propagating through the global domain. Although we do not consider much more complex cases in this work, such as changes in crack topology (due to crack branching or merging), we note that such problems have been examined in other works employing hybrid phase-field approaches.  This includes the recent work of Muixi et al.~\cite{muixi2021combined}, albeit in a purely mechanical context.

At each step in the solution algorithm, in the global problem, we keep track of two geometric entities, namely the global crack tip and the global tip element (Figure \ref{fig:all_tips}a). The global crack tip is defined by the interior endpoint of the crack.  The global tip element is the element that contains the tip on one of its sides, but is not yet ``cut" by the crack.  In essence, it is the global element just ahead of the propagating crack tip.   In the local subproblem, we identify a local version of the crack tip that is obtained from the discrete damage field $d^h$ (Figure \ref{fig:all_tips}b and \ref{fig:all_tips}c).  
The algorithm to extend the global crack geometry depends on the relative location of the global crack tip, global tip element, and local crack tip, as described below.

\begin{figure}[!h]
\centering
\begin{subfigure}{.33\textwidth}
  \centering
  \includegraphics[width=\linewidth]{img/Section2/propagation_items_1.png}
  \caption{}
  \label{fig:prop_items}
\end{subfigure}%
\begin{subfigure}{.33\textwidth}
  \centering
  \includegraphics[width=\linewidth]{img/Section2/propagation_items_2.png}
  \caption{}
  \label{fig:prop_items2}
\end{subfigure}
\begin{subfigure}{.33\textwidth}
  \centering
  \includegraphics[width=\linewidth]{img/Section2/propagation_items_3.png}
  \caption{}
  \label{fig:prop_items3}
\end{subfigure}
\caption{(a): Illustration of the global crack tip and global tip element. (b) Case when the local tip falls inside the tip element. (c) Case when the local tip falls outside the tip element.}
  \label{fig:all_tips}
\end{figure}

The enrichment strategy described in Section~\ref{sec:global_disc} requires global elements that are completely cut by the crack geometry.  As such, any extension of the crack geometry at the global scale must correspond to the global tip element being fractured.  Accordingly, global crack propagation is triggered whenever the local tip falls outside the tip element. In this case, the new global tip is identified by connecting the current global tip and the local tip. Since the local tip is outside the tip element, a new segment will intersect the perimeter of the tip element exactly once (neglecting the obvious intersection at the current global tip). This process is illustrated in Figure \ref{fig:tip_progression}. Any crack advance beyond this new tip location is neglected at this point, and the algorithm returns to a new global solve with an updated crack geometry.  

\begin{figure}
\centering
\begin{subfigure}{.331\textwidth}
  \centering
  \includegraphics[width=.99\linewidth]{img/Section2/schematic_1.png}
  \caption{}
  \label{fig:prop_1}
\end{subfigure}%
\begin{subfigure}{.33\textwidth}
  \centering
  \includegraphics[width=\linewidth]{img/Section2/schematic_2.png}
  \caption{}
  \label{fig:prop_2}
\end{subfigure}
\begin{subfigure}{.33\textwidth}
  \centering
  \includegraphics[width=\linewidth]{img/Section2/schematic_3.png}
  \caption{}
  \label{fig:prop_3}
\end{subfigure}
\caption{(a): Construction of the segment connecting global and local tips.  (b) New crack segment to be added. (c) Update of the global crack tip and global tip element.}
  \label{fig:tip_progression}
\end{figure}

In general, the problem of extracting a sharp crack front from a diffuse representation is not trivial. When the crack evolution involves complex topological changes, it is particularly difficult\cite{tamayo2015medial}. In the numerical examples studied in this Chapter, we take advantage of the relatively simple fracture geometry and predicted crack patterns to simplify this process.  In particular, we first identify all elements in the local subdomain with nodes whose damage values are all above a threshold $d_{tr}$. A similar approach was proposed in \cite{giovanardi2017hybrid}. The local crack tip is taken to be the center of the element in this set that is farthest from the base of the crack in the local subdomain.  The threshold used for this process is taken to be $d_{tr} = 1 - h_{local}/(2\ell)$, which is based on the estimate for a damage field near a crack tip given in \cite{yoshioka2020crack}. In essence, for a phase-field model of fracture, this threshold identifies nodes that are expected to correspond to the peaks of the discrete damage field.  For a damage band resolved with a mesh spacing of $\ell / h_{local} = 4$, this gives rise to a threshold of  $d_{tr} = 0.875$. 

\subsection{Algorithm summary}

Having described the solution strategies for both the global and local problems and the transfer of various quantities, we now detail the algorithm \eqref{mr_algorithm} that couples the two problems together to simulate crack propagation. Figure~\ref{fig:solution_algorithm} provides an illustration of the algorithm. Within each time step (outer loop), the algorithm employs an inner loop that allows the global problem to be updated as soon as any large enough change in the crack geometry is detected in the local subproblem.  The inner loop is terminated when the propagation step, described in subsection \ref{propagation_step} does not identify any crack advance.  The construction with two nested loops can be viewed as an implicit treatment of the fracture front position, which, according to Lecampion et al.\ \cite{lecampion2018numerical} tends to be more accurate and robust, permitting the use of larger time steps.

\medskip

\begin{algorithm}[H]\label{mr_algorithm}
\small
\SetAlgoLined
 Define initial and boundary conditions
 
 $n = 1$ \\
 $n_F = endStep$
 
 \While{$n \le n_F$}{
 
    \medskip
    
    $\mathcal{F}_n^{1} = \mathcal{F}_{n-1}$
 
    \For{$1 \le k \le maxIter\footnote{The index $k$ is only a dummy variable for this loop that searches for the correct fracture geometry at a given time step.}$}{

    \medskip
    
    (1) Solve \textbf{Global Problem} and obtain $\textbf{u}_G^{k,n}, p_m^{k,n}, p_f^{k,n}$

    \medskip
    
    (2) Construct subdomain $\Omega_L$ and submesh $\mathcal{T}_L$. Identify subset of cracked nodes $\mathcal{K}$.
    
    \medskip
    
    (3) Prescribe local boundary conditions $\textbf{u}^h_{L}|_{\partial\Omega_L} = \textbf{u}_G^{k,n}|_{\partial\Omega_L}$ and $d^h = 1$ on $\mathcal{K}$.
    
    \medskip
    
    (4) Construct local pressure fields $p_m= \Pi^{\Omega_L}_m (p_m^{k,n})$ and $p_f = \Pi^{\Omega_L}_f (p_f^{k,n})$
    
    \medskip

    (5) Solve the \textbf{Local Problem} to obtain $d^{k}$.

    \medskip
    
    (6) Use $d^{k}$ and the propagation step (subsection \ref{propagation_step}) to update discrete fracture $\mathcal{F}_n^{k}$.

    \medskip
    
    \If {$\mathcal{F}_n^{k+1}$ = $\mathcal{F}_n^{k}$}{
    
        \medskip
    
        n = n + 1
        
        \medskip
        
        $\mathcal{F}_n = \mathcal{F}_n^{k}$
        
        \medskip
            
        $\textbf{u}_G^n = \textbf{u}_G^{k,n}$
        
        \medskip
            
        $p_m^n = p_m^{k,n}$
        
        \medskip
        
        $p_f^n = p_f^{k,n}$
        
        \medskip
            
        break
        
        \medskip
    
        }

    }
    
 }
 \caption{Solution algorithm for multi-resolution hydraulic fracture}
\end{algorithm}

\begin{figure}[h]
    \centering
    \includegraphics[width=\linewidth]{img/Section2/algorithm_fancy.png}
    \caption{Multi-resolution solution algorithm.}
    \label{fig:solution_algorithm}
\end{figure}

\section{Results}
\label{sec:results}

We now present results for a set of problems that highlight the advantages, as well as some limitations, of the various models for pressurized cracks in a phase-field for fracture setting. In the first problem, the cohesive fracture of an uniaxial specimen in a pressurized environment is analyzed. We then consider the problem of crack nucleation from a pressurized hole in a medium subjected to far-field, biaxial compression.  
%Then, the attention moves to fracture initiation, which is in general, a difficult task for phase-field models. 
%To the authors' knowledge, it has only been investigated in scenarios where the fracture surfaces are traction-free. 
Finally, a crack propagation example is studied to verify that in the limit of a vanishing regularization length Griffith-like behavior is recovered with the new model.  In all cases, plane-strain conditions are assumed to hold.  

In the course of explaining the results obtained with the cohesive phase-field model, it will be useful to characterize the effective cohesive strength $\sigma_c$ of the material.  To that end we will rely on the following relationship between the cohesive strength and the nucleation energy:
\begin{equation}
  \label{eq:sigmacrit-from-psicrit}
   \sigma_c = \sqrt{ \dfrac{2E \psi_c }{(1-\nu^2)} },
\end{equation}
where $E$ denotes Young's modulus and $\nu$ Poisson's ratio.  This equation results from the analysis of a one-dimensional system subjected to uniaxial loading \cite{geelen2019phase}, and should be viewed as an approximation to the cohesive strength in more general loading conditions.  

% \begin{itemize}
%    \item {\color{red}Gary commented that we could switch the term quasi-brittle to cohesive fracture, but to my understanding, these are synonyms. I have used them interchangeably throughout the paper.}
%    \item {\color{red} Add that all examples are plane-strain with unit thickness.}
%\end{itemize}

\subsection{Uniaxial bar under traction in a pressurized environment}

%motivation
%The possibility of retrieving a cohesive fracture behavior with a phase-field model was established for traction-free cracks in \cite{lorentz2011convergence}, with the use of the rational degradation function \eqref{cohesive_degradation}. However, up to this point, the influence of a pressure load in this scenario has not been investigated. 

We consider the fracture behavior of a cohesive material with pressure loading on the crack faces.  The example is intended to examine the extent to which the pressure loading can artificially influence the apparent traction-separation law on the crack surface.  

%To shed light in this question, this first example deals with a case where a cohesive material is fractured in a pressurized environment. This situation serves to verify that the proposed formulation \eqref{uvc} can be combined with the phase-field model for cohesive fracture \cite{lorentz2011convergence, geelen2019phase}, described in subsection \ref{cohesive_frac} without compromising the apparent traction-separation law. This law, under the assumption of quasi-brittle fracture regime, is a material property, and thus, must be independent of the external pressure load.

\begin{figure}[h]
    \centering
    \includegraphics[width=.6\linewidth]{images/traction_separation/uniaxial_cohesive.pdf}
    \caption{Uniaxial cohesive bar.}
    \label{fig:1d_problem_schematic}
\end{figure}

\begin{table}[h]
\centering
\caption{Material properties for uniaxial bar}
\begin{tabular}[t]{lcc}
\hline
&Value &Unit \\
\hline
Young's modulus ($\text{E}$)&4.0$\times10^5$&MPa\\
Poisson's ratio ($\nu$)&0.2&--\\
 Nucleation energy  ($\psi_c$)&5.6$\times10^{-5}$&$\text{mJ mm}^{-3}$\\
Critical fracture energy ($G_c$)&0.12&$\text{mJ mm}^{-2}$\\
%Tensile strength ($\sigma_c$)&3.0&$\text{MPa}$\\
Residual stiffness ($\xi$)&1.0$\times10^{-8}$&--\\
\hline
\end{tabular}
\label{material_properties_p1}
\end{table}%

%\tablefootnote{$\sigma_c$ and $\psi_c$ are related through the expression $2\psi_c = \sigma_c^2(1-\nu^2)/E$ {\color{red} I don't like using a footnote to handle this - let's discuss} }

%{\color{red} Gary pointed out that critical fracture strength and critical fracture energy are somewhat confusing names for $\psi_c$ and $G_c$ So, I looked at what was used in Rudy's 2019 paper, and we have critical fracture energy per volume and per area, which I think is an even more confusing choice. I suggest, for $\psi_c$, "critical energetic strength", "energetic threshold" or "critical fracture energy". For $G_c$, "critical energy release rate" avoids any confusion I believe. }

%physical and numerical setup
The problem consists of a bar under a displacement controlled load in a pressurized chamber, as shown in Figure~\ref{fig:1d_problem_schematic}. The bar is assumed to be made of a linear elastic material that undergoes cohesive fracture, with a traction-separation law $F(s)$.
%A schematic of the problem, after the cracking process begins, is shown in Figure \ref{fig:1d_problem_schematic}. 
The bar has an undeformed length $2L = 400$ mm and width $2W = 2$ mm. The material properties are given in Table \ref{material_properties_p1}. Symmetry boundary conditions are invoked to reduce the computational domain to the top-right quarter of the bar. The applied load is modeled as a displacement boundary condition on the right end of the domain. The mesh consists of rectangular elements of size $h$ along the length direction and size 1 mm in the width direction.
The initial applied displacement increment is $\Delta u = 5\times10^{-4}$ mm. The displacement increment is adaptively refined when convergence is not obtained within a fixed set of iterations.   A more detailed description of the adaptive stepping procedure is provided in \cite{gaston2009moose, permann2020moose, lindsay20222}. 

Damage localization is triggered by introducing an small initial defect ($d = \mathcal{O}(\epsilon)$) on the left side of the domain.  In what follows, results are reported using $\ell = L/20 = 10$ mm and $h = \ell/10 = 1$ mm.  This choice of regularization length and mesh spacing was found to yield spatially-converged results.  
%In addition, all results were obtained using the linear indicator function $I(d) = d$.  
Different values of pressure, ranging from $0$ to $\sigma_c/3$ are considered. 

The problem is simulated using discretized versions of both the \ref{uvc} and \eqref{lvc} formulations.  
%Before investigating the response of various models for this problem, an analysis of convergence with respect to the regularization length $\ell$ and element size $h$ was performed. The analysis indicated that the results were insensitive to $\ell$ and $h$ if $\ell < L/10 = 20$ mm and $h < \ell/5$. The results shown next were obtained with $\ell = L/20 = 10$ mm and $h = \ell/10 = 1$ mm. The linear indicator function $I(d)$ was chosen, but with the \ref{uvc} formulation, the sensitivity to this indicator is minimal.
%reference results
%For comparison, this same problem is also solved with the formulation \eqref{lvc},  which has been employed in several studies, such as \cite{bourdin2012variational, wheeler2014augmented, peco2017influence, wilson2016phase, jiang2022phase}. 
For the indicator function $I(d)$, results are reported for: (1) $I(d) = d$, used for example in \cite{bourdin2012variational}; (2) $I(d) = d^2$, used in \cite{jiang2022phase} and (3) $I(d) = 2d-d^2$, used in \cite{wheeler2014augmented}.
%post-processing details
The effective traction-separation laws extracted from the set of simulations are shown in Figure \ref{fig:traction_separation_results}. To generate these curves, the traction is computed as the internal force measured in the center of the bar. The separation $s$ is the opening of the crack, calculated as $s = -\int\limits_{-\infty}^{\infty}\textbf{u}\cdot\nabla I(d) \text{dx}$ \cite{bourdin2012variational}. 

%which is the left of the computational domain (i.e $t = -R^{\text{left}}_x$ )

\begin{figure}[h]
\centering
\begin{subfigure}{.45\textwidth}
  \centering
  \includegraphics[width=\linewidth]{images/traction_separation/1d_nodash_bourdin_I_d.pdf}
  \caption{}
  \label{fig:traction_separation_bourdin_d}
\end{subfigure}%
\begin{subfigure}{.45\textwidth}
  \centering
  \includegraphics[width=\linewidth]{images/traction_separation/1d_nodash_bourdin_I_d2.pdf}
  \caption{}
  \label{fig:traction_separation_bourdin_d2}
\end{subfigure}%

\bigskip
\begin{subfigure}{.45\textwidth}
  \centering
  \includegraphics[width=\linewidth]{images/traction_separation/1d_bourdin_I_2d.pdf}
  \caption{}
  \label{fig:traction_separation_bourdin_2d}
\end{subfigure}
\begin{subfigure}{.45\textwidth}
  \centering
  \includegraphics[width=\linewidth]{images/traction_separation/1d_gary_I_d.pdf}
  \caption{}
  \label{fig:traction_separation_gary}
\end{subfigure}
  \caption{Traction-separation curves for pressurized uniaxial cohesive bar problem, obtained with various phase-field models: (a) \eqref{lvc} with linear indicator function; (b) \eqref{lvc} with quadratic indicator function; (c) \eqref{lvc} with $2d-d^2$ indicator function; and (d) Proposed approach \eqref{uvc} with linear indicator function. } 
  \label{fig:traction_separation_results}
\end{figure}

%discussion of results
The results for the various models are shown in Figure \ref{fig:traction_separation_results}, with tractions and pressures normalized by the critical stress $\sigma_c$ from \eqref{eq:sigmacrit-from-psicrit}.
As shown in Figure \ref{fig:traction_separation_results}, the proposed model \eqref{uvc} exhibits minimal sensitivity to the pressure magnitude in the traction-separation behavior.  
By contrast, with the \eqref{lvc} formulation, only the case with $I(d) = 2d - d^2$ exhibits comparable results. In the other two cases (Figures \ref{fig:traction_separation_bourdin_d} and \ref{fig:traction_separation_bourdin_d2}), the apparent traction-separation law shows a spurious dependence to the applied pressure. This is evident in the variations in the results as well as the presence of jumps in the aperture at sufficiently high pressures. The latter occur due to an instability of the partially damaged solutions as $d$ approaches 1. More precisely, shortly after the damage at the center of the bar reaches $d\approx 0.8$, it jumps to $d= 1$, which in turns lead to a jump in the aperture.  
 This jump is indicated via the squares that appear on selected curves in Figures \ref{fig:traction_separation_bourdin_d} and \ref{fig:traction_separation_bourdin_d2}. The use of smaller displacement increments was not observed to significantly impact these results.   By contrast, such instabilities were not observed for the simulations reported in  Figures \ref{fig:traction_separation_bourdin_2d} and \ref{fig:traction_separation_gary}.

%As a result, the traction-separation laws exhibit the gaps at the points indicated by the solid squares. This phenomenon is not observed in the cases depicted in Figures \ref{fig:traction_separation_bourdin_2d} and \ref{fig:traction_separation_gary}.

% This example is designed to verify the cohesive response of an uniaxial specimen in the presence of a pressure. The setup consists of a bar, under a displacement controlled load, in a pressurized chamber. The bar is assumed to be made of a linear elastic material that undergoes cohesive fracture, with a traction-separation law $F(s)$. A schematic of the problem, after the cracking process begin, is shown in Figure \ref{fig:1d_problem_schematic}.

% The possibility of retrieving a cohesive fracture behavior with a phase-field model was established for traction-free cracks in \cite{lorentz2011convergence}, with the use of the degradation function \eqref{cohesive_degradation}. However, up to this point, the influence of a pressure load in this scenario has not been investigated. So, to shed light in this question, the model in this example combines the use of the degradation function \eqref{cohesive_degradation} with the two formulations for incorporating pressure, described in Section \ref{sec:model}.

% First, the pressure load is introduced using the formulation \eqref{lvc}, which has been employed in several studies, such as \cite{bourdin2012variational, wheeler2014augmented, peco2017influence, wilson2016phase, jiang2022phase}. This formulation requires the use of an indicator function $I(d)$, that identify the regions where the pressure load is applied. The cases $I(d) = d$, used for example in \cite{bourdin2012variational}, $I(d) = d^2$, which appeared in \cite{jiang2022phase} and $I(d) = 2d-d^2$, chosen in \cite{wheeler2014augmented} are considered.

% Then, the results are then compared with the ones obtained using the proposed formulation \eqref{uvc}, with the simplest indicator function $I(d) = d$. 

% To model the cohesive fracture behavior, the phase-field model uses the degradation function \eqref{cohesive_degradation}. In the absence of pressure, it is shown in \cite{lorentz2011convergence} that this choice allows the phase-field model to represent, as $\ell \rightarrow 0$, the physics of 

% As discussed in Section \ref{sec:model} (subsection 2.2), it was shown in \cite{lorentz2011convergence, lorentz2011gradient, geelen2019phase, wu2017unified} that, by using the degradation function \eqref{cohesive_degradation} one can approximate certain families of cohesive responses with a phase-field model of fracture. The additional assumption that $F(s)$ is a member of one of these families is then necessary. CLARIFY THAT (31) IS USED WITH NO SPLIT.

% The first numerical example consists of a bar, under a displacement controlled load, in a pressurized environment. The bar is assumed to be made of a linear elastic material, that undergoes cohesive fracture, with a traction-separation law $F(s)$. A schematic of the problem, after the cracking process begin, is shown in Figure \ref{fig:1d_problem_schematic}. The bar has length $2L = 400$ mm and width $2W = 2$ mm. The material properties are given in Table \ref{material_properties_p1}. 

% {\color{blue} Quick analysis of the problem

% \begin{equation}\tag{Total deformation}
%     L\dfrac{\sigma}{E} + \dfrac{s}{2} = L + \Delta u
% \end{equation}

% \begin{equation}\tag{Force balance}
%     \sigma = -p + F(s)
% \end{equation}

% \begin{equation}\tag{Nonlinear equation for s}
%     F(s) + \dfrac{sE}{2L} = E\left( 1 + \dfrac{\Delta u}{L} \right) + p
% \end{equation}

% }

% As discussed in Section \ref{sec:model} (subsection 2.2), it was shown in \cite{lorentz2011convergence, lorentz2011gradient, geelen2019phase, wu2017unified} that, by using the degradation function \eqref{cohesive_degradation} one can approximate certain families of cohesive responses with a phase-field model of fracture. The additional assumption that $F(s)$ is a member of one of these families is then necessary. CLARIFY THAT (31) IS USED WITH NO SPLIT.

% The value of this example resides on the fact that the traction-separation law $F(s)$ is an intrinsic property of the material, and therefore, should be independent of any applied loads and pressures. {\color{red}To the author's knowledge, although phase-field models for quasi-brittle fracture have been used to simulate pressure-driven cracks in \cite{jiang2022phase, li2022hydro}, this simple verification has not been performed.}

% \begin{figure}[!htbp]
% \centering
% \begin{minipage}{.48\textwidth}
%   \centering
%   \includegraphics[width=\linewidth]{images/gary_several_ps.png}
%   \caption{Traction-separation response with proposed model.}
%   \label{fig:1d_bar_gary}
% \end{minipage}%
% \hfill
% \begin{minipage}{.48\textwidth}
%   \centering
%   \includegraphics[width=\linewidth]{images/bourdin_several_ps.png}
%   \caption{Traction-separation response with existing models.}
%   \label{fig:1d_bar_bourdin}
% \end{minipage}
% \end{figure}

% \begin{figure}[h]
% % \centering
% \begin{subfigure}{.33\textwidth}
%   \centering
%   \includegraphics[width=\linewidth]{images/traction_separation/traction_separation_bourdin.pdf}
%   \caption{}
%   \label{fig:traction_separation_quadratic}
% \end{subfigure}%
% \begin{subfigure}{.33\textwidth}
%   \centering
%   \includegraphics[width=\linewidth]{images/traction_separation/traction_separation_quadratic.pdf}
%   \caption{}
%   \label{fig:traction_separation_bourdin}
% \end{subfigure}%
% \begin{subfigure}{.33\textwidth}
%   \centering
%   \includegraphics[width=\linewidth]{images/traction_separation/traction_separation_gary.pdf}
%   \caption{}
%   \label{fig:traction_separation_gary}
% \end{subfigure}
%   \caption{(a) Quadratic indicator function; (b) Linear indicator function; and (c) Proposed approach. } 
%   \label{fig:traction_separation_results}
% \end{figure}



%Gamma convergence figure - could add 3 of these
% \begin{figure}[h]
%     \centering
%     \includegraphics[width=.6\linewidth]{images/traction_separation/gamma_convergence_gary.pdf}
%     \caption{Verify reg. length independence, p = 0.5MPa.}
%     \label{fig:gamma_convergence}
% \end{figure}
%\FloatBarrier
\subsection{Crack nucleation from a pressurized hole}

Consider a square plate of dimensions $L \times L$, with a circular hole in the center subjected to an internal pressure $p$, as shown in Figure \ref{fig:cavity_schematic}. This problem is motivated by oil and gas wellbore systems.   Far field stresses $\sigma_V$ and $\sigma_H$ are applied as tractions on the boundaries as shown. 
%This setup can be viewed, for example, as the cross-section of an oil and gas wellbore. 
The pressure is increased until it reaches a ``breakdown pressure" $p_b$. When that happens, cracks initiate in the direction parallel to the maximum \textit{in-situ} stress. Assuming $\sigma_H > \sigma_V$, this is expected to occur along a horizontal axis passing through the center of the hole.  In this work, the pressure in the hole is assumed to follow the crack faces as the fracture grows into the interior of the domain.  

\begin{figure}[h]
% \centering
\begin{subfigure}{.49\textwidth}
  \centering
  \includegraphics[width=0.8\linewidth]{images/2d_nucleation/cavity_schematic.pdf}
  \caption{}
  \label{fig:cavity_schematic}
\end{subfigure}%
\begin{subfigure}{.49\textwidth}
  \centering
  \includegraphics[width=0.71\linewidth]{images/2d_nucleation/quarter_mesh.pdf}
  \caption{}
  \label{fig:quarter_mesh}
\end{subfigure}%
  \caption{(a) Problem schematic; (b) Mesh used in the computations, exploiting symmetry. } 
  \label{fig:initiation_problem_setup}
\end{figure}

\begin{table}[h]
\centering
\caption{Material properties, geometric parameters and applied loads for crack initiation problem}
\begin{tabular}[t]{lcc}
\hline
&Value &Unit \\
\hline
Young's modulus ($\text{E}$)&19.0$\times10^3$&MPa\\
Poisson's ratio ($\nu$)&0.2&--\\
Nucleation energy ($\psi_c$)&7.96$\times10^{-4}$&$\text{mJ mm}^{-3}$\\
Critical fracture energy  ($G_c$)&7.70$\times10^{-2}$&$\text{mJ mm}^{-2}$\\
Cavity radius ($R$)&400&mm\\
Specimen length ($L$)&5.0$\times10^{3}$&mm\\
Horizontal stress ($\sigma_H$)&5.0&MPa\\
Vertical stress ($\sigma_V$)&2.5&MPa\\
\hline
\end{tabular}
\label{material_properties_initiation}
\end{table}

%The propagation of the fracture after the breakdown pressure is reached depends on the nature of the applied load. If one assumes that the pressure applied to the cavity follows the fracture as it grows (for example, if the cavity is filled by a high pressure gas), then crack propagation will be unstable and result in complete failure of the specimen. However, if the cracks are assumed to be traction-free, it is possible to obtain stable growth, and total failure of the specimen will occur only if the applied pressure is further increased. This latter scenario has been investigated using a phase-field model of fracture, in the work of Tanné \cite{tanne2017variational}. In this work, the former case is considered.

The material properties selected for this problem, along with the dimensions and loading parameters are listed in Table \ref{material_properties_initiation}.  The material properties are taken to be representative of a Bebertal sandstone, as inspired by the experiments of \cite{stoeckhert2015fracture}.
%In {\color{purple}terms of material properties, a Bebertal sandstone is used,} following the experiments of \cite{stoeckhert2015fracture}. They are listed, alongside with the geometric parameters and applied loads in Table \ref{material_properties_initiation}. 
The symmetry of the problem is exploited to reduce the computational domain to the top-left quarter.  An unstructured triangular mesh is used, with local refinement along the $x$-axis, as shown in Figure \ref{fig:quarter_mesh}. The element size in the refined area is $10$mm, whereas the phase-field regularization length is $\ell = 40$mm. For the results reported in this section, the phase-field model employs the cohesive formulation\cite{lorentz2011convergence, geelen2019phase} using the degradation function \eqref{cohesive_degradation} and the spectral split of  \cite{miehe2010phase}.  

%, that uses the degradation function \eqref{cohesive_degradation} is again employed.  As the applied loads give rise to considerable compression in the domain, the split proposed in \cite{miehe2010phase} is used in the model.  

%, as it provides a bound for crack initiation that is independent of the regularization length. {\color{purple}Since the load has a large compressive part, a decomposition of the strain is needed.} The spectral split proposed in \cite{miehe2010phase} is then used.

Intuitively, the magnitude of the pressure load required to initiate fracture in this problem is expected to be independent of whether or not the pressure follows the crack evolution. After initiation, the pressure effects become important and the fracture propagates unstably. Due to this unstable behavior, it is very difficult to numerically capture the crack path after the pressure $p_b$ is reached. In order to have a glimpse into what this path looks like, a viscous term $\eta \dot d$ is added to the phase-field equation, as in \cite{miehe2010phase}, with $\eta = 10^{-3}\  \text{mJ}\cdot\text{mm}^{-3}\cdot$s. 

It bears emphasis that the equations \eqref{disp equation} and \eqref{damage equation ch2} indicate that, in the absence of any damage, the proposed model for pressurized cracks reduces to the standard phase-field fracture model for traction-free cracks. Therefore, one should expect the proposed model to capture fracture initiation properly in this scenario. On the other hand, for the \eqref{lvc} formulation, this only occurs if the indicator function satisfies $I'(0) = 0$. Among the many works which use the \eqref{lvc} formulation, only a few such as \cite{jiang2022phase, peco2017influence} used an indicator function satisfying this condition. In \cite{jiang2022phase}, the authors were indeed able to predict fracture initiation from pressurized holes. To highlight the implications of having $I'(0) \neq 0$ in the model \eqref{lvc}, the results for this problem will also be presented using the  \eqref{lvc} formulation with the indicator function $I(d) = d$.

% WHICH INDICATOR FUNCTION? HOW TO SEPARATE THE I(d) = d2 CASE
%  Although it leads to the not so interesting case of unstable propagation, it illustrates a limitation of  the formulation \eqref{lvc}, which, in these conditions, will predict spurious damage growth in the domain boundaries. That happens due to the presence of the term $p\nabla\cdot u$ in equation \eqref{wet damage equation box 2}, even when the material is intact. This issue is circumvented when the formulation \eqref{uvc} is applied since no modification to the evolution equation for damage is made. 

% Since the fracture propagates unstably, it is very difficult to numerically capture its path after the pressure $p_b$ is reached. In order to have a glimpse on what the path looks like, a viscous term $\eta \dot d$ is added to the phase-field equation, as in \cite{miehe2010phase}, with $\eta = 10^{-3}\  $MPa$\cdot$s.

%For an infinite plate, the breakdown pressure can be estimated by the Hubbert and Willis formula \cite{hubbert1957mechanics}, which assumes that failure occurs when the maximum hoop stress in the cavity reaches the tensile strength $\sigma_c$. However, due to the finite size of the computational domain, a more precise verification consists in comparing the maximum value of $\sigma_{\theta\theta}$ prior to crack nucleation to $\sigma_c$, from \eqref{eq:sigmacrit-from-psicrit}.  

The final damage patterns obtained using the \eqref{uvc} formulation and the \eqref{lvc} formulation  are shown in Figure \ref{fig:damage_profiles}. With the \eqref{uvc} formulation, damage localizes along the midplane when the hoop stress is approximately 85\% of $\sigma_c$.  This is not unexpected, as the expression \eqref{eq:sigmacrit-from-psicrit} is based on a one-dimensional state of stress and strain which differs significantly from the state near the corner of the hole.    The same comparison is not performed for the simulation using the model \eqref{lvc}, since damage forms only on the boundary in the first steps leading to spurious rigid body motion. 

% However, since in most application cases the pressure load comes from some invading fluid, the viscosity of this fluid will stabilize the fracture evolution and this additional term will not be necessary.

%Table \ref{results_initiation} compares the prescribed values of $\sigma_c$ and $\psi_c$ with those obtained in the simulation using the proposed model. The simulation values are computed at the onset of damage initiation, in the location where the damage begins. {\color{red}(i.e we look at the first point that sees $d > 0$ and take its $\Psi_e$ and $\sigma_c$ at that instant. )} The same comparison is not performed for the simulation using the model \eqref{lvc}, since damage forms only on the boundary in the first steps leading to spurious rigid body motion. The final damage profiles are shown in Figure \ref{fig:damage_profiles}. 

%\begin{table}[ht]
%\centering
%\caption{Comparison of results}
%\begin{tabular}[t]{lcccc}
%\hline
%&$\sigma_c$(Simulation) & $\sigma_c$(Prescribed) & $\psi_c$(Simulation) & $\psi_c$(Prescribed) \\
%\hline
%\eqref{uvc} formulation & 4.73 & 5.60 & 7.98e-4 & 7.96e-4\\
%\hline
%\end{tabular}
%\label{results_initiation}
%\end{table}

\begin{figure}[h]
% \centering
\begin{subfigure}{.49\textwidth}
  \centering
  \includegraphics[width=0.7\linewidth]{images/2d_nucleation/gary_crack.png}
  \caption{}
  \label{fig:damage_profile_gary}
\end{subfigure}%
\begin{subfigure}{.49\textwidth}
  \centering
  \includegraphics[width=0.86\linewidth]{images/2d_nucleation/bourdin_crack.png}
  \caption{}
  \label{fig:damage_profile_bourdin}
\end{subfigure}%
  \caption{(a) Final crack pattern using proposed model; (b) Damage field using the model from \cite{bourdin2012variational}. } 
  \label{fig:damage_profiles}
\end{figure}

The main takeaway is that the proposed model \eqref{uvc} allows one to study crack nucleation and subsequent propagation under a pressure load, whereas formulation \eqref{lvc} leads to spurious damage formation if $I'(0) \neq 0$. The presence of the term $p\nabla \cdot \textbf{u}I'(d)$ in the damage equation \eqref{wet damage equation box2} drives crack formation in areas which are not stressed. For this specific problem, this issue can be circumvented using for example $I(d) = d^2$, as shown in \cite{jiang2022phase}, but this option introduces a spurious dependence of the cohesive response of the material on the applied pressure, as indicated in the last section (Figure \ref{fig:traction_separation_bourdin_d2}).

%Although the proposed formulation \eqref{uvc} displays a good qualitative comparison with the theory in this example, the error in the predicted $\sigma_c$ must also be explained. It comes from the fact that, in the cohesive phase-field formulation used herein, damage initiation is governed by a strain energy envelope $\psi_e^+(\bs\epsilon(\textbf{u}))\le\psi_c$. In this two-dimensional setting, and using the spectral split to compute $\psi_e^+$, the relationship used to define $\sigma_c = \psi_c/2E$ does not hold. Since the cohesive model relies $\psi_c$ as a bound for damage initiation, the comparison between the simulated and prescribed values for this quantity will be much better than the same comparison for $\sigma_c$ in two or three dimensions. The general applicability of the strain energy as a bound for fracture initiation is debatable in certain circumstances, such as in cases with nearly incompressible materials \cite{kumar2020revisiting}. In future work, phase-field formulations specifically tailored to match experimentally-tested strength envelopes, such as \cite{kumar2020revisiting, de2021nucleation, navidtehrani2022general} will be investigated.

% The direct comparison between \eqref{hubbert_and_willis} and numerical simulations with phase-field requires further explanation. Two key assumptions behind \eqref{hubbert_and_willis} are that crack initiation is governed by the uniaxial tensile strength $T$ and that the domain is infinite. Both of these are not satisfied in the numerical simulations. The domain is finite, with $L = 5$ m and crack initiation is governed by the strain energy envelope, with a threshold $\psi_c$. In a uniaxial case, this threshold is equivalent to a tensile strength of $\sigma_c = \sqrt{2E'\psi_c}$, and this value is calibrated to match $T$. However, the stress state in the cavity problem is not uniaxial, and therefore, all components of the stress will play a role in crack initiation. Due to these differences, the discrepancy between the values of $p_b$ coming from \eqref{hubbert_and_willis} and the model \eqref{u_equation}-\eqref{d_equation} are not concerning. When the critical strain energy measured in the simulation is compared with what is prescribed as a material parameter, the discrepancy is minimal.  



% In our second example, we will consider the case of a plate with a hole, in which the whole is uniformly pressurized on its contour. For example, this scenario could represent a cross section of a pipe or a pressure vessel. If cracks start to form in the plate, the high-pressure gas will occupy the aperture space and apply pressure in the crack faces, therefore, it is important to account for these loads. 

% This problem is particularly interesting because it involves two different modellings of the same pressure load. When the pressure is acting in the walls of the hole, it is modeled with Neumann boundary conditions, whereas as soon as cracks start to form, it has to be account also in these newly formed surfaces, with a phase-field regularization. 

% To correctly predict crack initiation, the phase-field regularization of the pressure load must be designed so that, in the early stages, before any damage kicks in, it effectively recover the underlying phase-field model without any pressure effects, even though there are terms involving the pressure in the governing equations. This only happens if all terms involving the pressure are weighted by the damage variable. 

% If we observe the governing equations for our proposed model (\ref{u_equation} - \ref{d_equation}), that condition is satisfied. Whereas, for the system (\ref{u_equation_bourdin} - \ref{d_equation_bourdin}), the presence of the term $p\nabla \cdot u$ violates it. The main consequence, as we will see, is that damage initiation will happen at much lower loading conditions and in areas which are not at the highest stresses.

% DISCUSS REGULARIZATION LENGTH AND ELEMENT SIZE

% $\ell = 0.04;\ h = 0.01$

% ADD SCHEMATIC OF THIS PROBLEM

% \begin{figure}[!htbp]
% \centering
% \includegraphics[width=0.4\linewidth]{images/2d_nucleation/2d_initiation_schematic.pdf}
% \caption{Problem description.}
% \label{fig:p2_schematic}
% \end{figure}

% ADD RESULTS OF THIS PROBLEM

% \begin{figure}[!htbp]
% \centering
% \includegraphics[width=0.7\linewidth]{images/p_results_plot.png}
% \caption{Comparison of critical loads.}
% \label{fig:p2_schematic}
% \end{figure}

% \begin{figure}[!htbp]
% \centering
% \includegraphics[width=\linewidth]{images/p2_results.png}
% \caption{Crack patterns.}
% \label{fig:p2_schematic}
% \end{figure}

% These results highlight what is arguably the most remarkable advantage of our proposed model. As we can see, the cracks nucleate exactly at the stress concentrations, and exactly when the critical stress, which is an material property in the cohesive phase-field model is reached, even though the original equations are modified to account for pressure loads in the cracks. That is definitely a desirable outcome, since before the onset of damage, we expect this system to behave exactly as a system that does not account for pressure inside the cracks.

% On the other hand, using a model like (\ref{u_equation_bourdin} - \ref{d_equation_bourdin}), leads to crack nucleation at a location which doesn't have the highest stresses. This is a consequence of the term $p\nabla \cdot u$ acting as a driving force to damage formation, even though all the pressure effects prior to the onset of damage are already being accounted for in the Neumann boundary conditions. Worse than that, the threshold for damage initiation is well below the critical stress prescribed by the cohesive model, which definitely does not agree with the assumption that the pressure effects only start after damage initiates.

\subsection{Stable propagation of a pre-existing crack}

%One of the essential features of the phase-field approach for fracture is its ability to generalize Griffith's law in the limit $\ell \rightarrow 0$. In the first investigation of pressurized fracture with the phase-field method, Bourdin et al \cite{bourdin2012variational} verified this property for the model \eqref{lvc} by investigating a toughness-dominated hydraulic fracture problem. In this last example, this feature will be investigated for the the proposed model \eqref{uvc}. 

% The use of a hydraulic fracture example to assess this property leads to two difficulties. First, an analytical solution is only known for infinite domains, and therefore, one needs to use either a very large computational domain incurring in high computational costs or a Dirichlet-to-Neumann mapping in the external boundaries, as in \cite{wilson2016phase}. Second, to obtain stable growth, the fluid pressure must be treated as a variable. Then, a coupled mass conservation equation must be solved as in \cite{santillan2018phase}, or alternatively, an integral constraint can be used, as in \cite{bourdin2012variational, chukwudozie2013variational}. In either case, the aperture of the cracks must be resolved, which poses an additional challenge \cite{yoshioka2020crack}.

%In order to do that, a manufactured problem with stable crack growth is constructed. It consists of applying the so-called ``surfing" boundary conditions \cite{hossain2014effective} to 

Consider a strip of material with a pressurized crack, as shown in  Figure \ref{fig:surfing_schematic}.
The rectangular strip has a width $W$, height $H$ and a crack with initial size $a$ (values provided in Table \ref{material_properties_propagation}), and is loaded by the ``surfing" boundary condition $\widetilde{U_y}(x,y,t)$ on its top and bottom surfaces \cite{hossain2014effective}. The boundary condition is given by
\begin{equation}\label{surfing_bc_1}
    \widetilde{U_y}(x,y,t) = U_y(x-Vt,y),  
\end{equation}
where
\begin{equation}\label{surfing_bc_2}
    U_y(x,y) = \hat{U}_y(r,\theta) = \dfrac{\sqrt{G_cE'}}{2\mu}\sqrt{\dfrac{r}{2\pi}}(\kappa - \cos\theta)\sin\dfrac{\theta}{2},
\end{equation}
and where $r$ and $\theta$ are polar coordinates with respect to the origin, taken to be the midpoint of the left edge of the domain. The constant $V>0$ is the target crack speed, prescribed by moving the boundary condition following \eqref{surfing_bc_1}. The Kolosov constant is defined as $\kappa = 3-4\nu$ in plane strain and the shear modulus $\mu = E/(2+2\nu)$. The pressure $p$ applied to the crack faces as the crack evolves is given by
\begin{equation}
    p = \dfrac{1}{2}\sqrt{\dfrac{G_cE'}{\pi a}},
\end{equation}
 in which $a$ denotes the initial crack length. This value corresponds to half the critical pressure for an infinite plate with a pressurized crack of size $a$. This magnitude ensures that the applied pressure is considerably large, but not so large as to drive the problem beyond the stable propagation regime. 
 
 To calculate the energy release rate, the domain form of the J-Integral \eqref{j_integral_theorem} developed in Section \ref{sec:j_integral} is used. The function $q$ is constructed by taking advantage of the finite element interpolation. In essence, the domain for the J-Integral is taken to be a single rectangular region of dimensions $a \times H/2$, centered on the initial crack tip. The value of $q$ for all nodes outside of this region is set to $0$, while $q = 1$ for all nodes inside. Using the finite element interpolation, this gives rise to a $q$ function whose value changes continuously from $0$ to $1$ on the elements cut by the rectangular path. This function is illustrated in Figure \ref{fig:integration_domain}\footnote{Due to mesh refinement near the crack surface, the width of the band where $0 < q < 1$ diminishes near the horizontal centerline of the domain.}.

\begin{figure}[h]
% \centering
\begin{subfigure}{.49\textwidth}
  \centering
  \includegraphics[width=0.8\linewidth]{images/2d_propagation/surfing_schematic.pdf}
  \caption{}
  \label{fig:surfing_schematic}
\end{subfigure}%
\begin{subfigure}{.49\textwidth}
  \centering
  \vspace{1.06cm}
  \includegraphics[width=0.8\linewidth]{images/2d_propagation/q_field_legend.png}
  \vspace{1.06cm}
  \caption{}
  \label{fig:integration_domain}
\end{subfigure}%
  \caption{(a) Geometry and boundary conditions for pressurized crack propagation problem; (b) J-Integral domain function $q$. } 
  \label{fig:surfing_problem_setup}
\end{figure}

\begin{table}[h]
\centering
\caption{Parameters used for pressurized crack propagation problem}
\begin{tabular}[t]{lcc}
\hline
&Value &Unit \\
\hline
Young's modulus ($\text{E}$)&3.0$\times10^4$&MPa\\
Poisson's ratio ($\nu$)&0.2&--\\
Critical fracture energy  ($G_c$)&0.12&$\text{mJ mm}^{-2}$\\
Initial crack length ($a$)&1.6&m\\
Specimen width ($W$)&8.0&m\\
Specimen height ($H$)&4.0&m\\
Target crack speed ($V$)&0.4&m/s\\
\hline
\end{tabular}\label{material_properties_propagation}
\end{table}

% The setup for this fracture propagation problem is the following. A rectangular strip with width $W$, height $H$ and a pre-crack of size $a$ is loaded by the ``surfing" boundary conditions in its top and bottom surfaces (Figure \ref{fig:surfing_schematic}). A pressure $p$, given by

% \begin{equation}
%     p = \dfrac{1}{2}\sqrt{\dfrac{G_cE'}{\pi a}},
% \end{equation}

% is applied on the crack faces.

% \begin{figure}[h]
% % \centering
% \begin{subfigure}{.49\textwidth}
%   \centering
%   \includegraphics[width=0.8\linewidth]{images/2d_propagation/surfing_schematic.pdf}
%   \caption{}
%   \label{fig:surfing_schematic}
% \end{subfigure}%
% \begin{subfigure}{.49\textwidth}
%   \centering
%   \vspace{1.06cm}
%   \includegraphics[width=0.8\linewidth]{images/2d_propagation/q_field_wigly_legend.png}
%   \vspace{1.06cm}
%   \caption{}
%   \label{fig:integration_domain}
% \end{subfigure}%
%   \caption{(a) Problem schematic; (b) J-Integral domain function $q$ FIX LEGEND WITH LETTER D - do we really need this 2nd figure now?} 
%   \label{fig:surfing_problem_setup}
% \end{figure}

In order to verify that Griffith's law is approached as $\ell \rightarrow 0$, simulations are performed for this problem using a sequence of decreasing regularization lengths, ranging from $\ell = a/20$ to $\ell = a/160$. The mesh is locally refined along the $x$-axis, where the element size is set to $h = \ell/4$. The symmetry of the problem is exploited and only the response in the top half of the domain is simulated. 
% {\color{blue}In the early times ($t < 1$), the boundary condition \eqref{surfing_bc_1}-\eqref{surfing_bc_2} is ramped up with a factor of $t$, as that seem to facilitate convergence of the numerical solver.}{\color{red} This is how the surfing BC is implemented in raccoon, maybe we don't need to give this level of detail to the reader as it might confuse them?}
In terms of constitutive choices of the phase-field model, the AT-1 formulation is employed without any decomposition of the strain. 
%Although in general the phase-field system is non-convex, we found that, in this case, a simple monolithic approach based on Newton's method was able to converge, allowing for much faster simulations and tighter residual tolerances, which helped visualize the convergence of the results with respect to $\ell$. This scheme is therefore used in all cases of this subsection. 
As in the previous examples, this problem is analyzed using the formulations \eqref{lvc} and \eqref{uvc}, and the following choices of indicator function $I(d)$:
\begin{itemize}
    \item $I(d) = d$
    \item $I(d) = d^2$
    \item $I(d) = 2d-d^2$ 
\end{itemize}

To evaluate how well the models approach Griffith's law, the ratio between the energy release rate measured by the J-Integral and the effective critical fracture energy $G^{eff}_c = (1+2h/c_0\ell) G_c$ \footnote{in fact, phase-field cracks actually dissipated a slightly larger energy per unit length in numerical models. A correction factor of $\left(1+\dfrac{2h}{c_0\ell}\right)$ is then applied to $G_c$, following \cite{yoshioka2020crack}. The factor of 2 here comes from the symmetry boundary condition employed.} is plotted in Figures \ref{fig:prop_bourdin} and \ref{fig:prop_gary}.  In all figures, the time is scaled by a characteristic time $\tau$, defined as $\tau = a/V$. %Since the loading stage before the onset of propagation is not particularly important in this example, most of the that part ($t < 0.3\tau$) is omitted for better visualization of the results during the propagation phase.
The results using the traditional \eqref{lvc} formulation are presented in Figure \ref{fig:prop_bourdin}. They indicate convergence towards $J/G^{eff}_c$ = 1 as the regularization length is reduced, especially when the indicator function $I(d)=d$ is used. This is expected given the results obtained in \cite{bourdin2012variational}. Nevertheless, these results serve to verify the implementation of the J-Integral presented in Section \ref{sec:j_integral}. They also provide an estimate for how small the regularization length has to be in order to achieve a certain level of accuracy with these types of phase-field models.

\begin{figure}[h]
% \centering
\begin{subfigure}{.33\textwidth}
  \centering
  \includegraphics[width=\linewidth]{images/2d_propagation/zoom_bourdin_I_d.pdf}
  \caption{}
  \label{fig:prop_bourdin_d}
\end{subfigure}%
\begin{subfigure}{.33\textwidth}
  \centering
  \includegraphics[width=\linewidth]{images/2d_propagation/zoom_bourdin_I_d2.pdf}
  \caption{}
  \label{fig:fig:prop_bourdin_d2}
\end{subfigure}%
\begin{subfigure}{.33\textwidth}
  \centering
  \includegraphics[width=\linewidth]{images/2d_propagation/zoom_bourdin_I_2d.pdf}
  \caption{}
  \label{fig:prop_bourdin_2d}
\end{subfigure}
  \caption{Reference results with the \ref{lvc} formulation. Curves with $\ell = a/160$ are not shown, as they are almost identical to the ones with $\ell = a/80$. (a) $I(d) = d$; (b) $I(d) = d^2$; (c) $I(d) = 2d-d^2$  } 
  \label{fig:prop_bourdin}
\end{figure}

\begin{figure}[h]
% \centering
\begin{subfigure}{.33\textwidth}
  \centering
  \includegraphics[width=\linewidth]{images/2d_propagation/zoom_gary_I_d.pdf}
  \caption{}
  \label{fig:prop_gary_d}
\end{subfigure}%
\begin{subfigure}{.33\textwidth}
  \centering
  \includegraphics[width=\linewidth]{images/2d_propagation/zoom_gary_I_d2.pdf}
  \caption{}
  \label{fig:prop_gary_d2}
\end{subfigure}%
\begin{subfigure}{.33\textwidth}
  \centering
  \includegraphics[width=\linewidth]{images/2d_propagation/zoom_gary_I_2d.pdf}
  \caption{}
  \label{fig:prop_gary_2d}
\end{subfigure}
  \caption{Results with the proposed formulation \eqref{uvc} (a) $I(d) = d$; (b) $I(d) = d^2$; (c) $I(d) = 2d-d^2$  } 
  \label{fig:prop_gary}
\end{figure}

% \begin{figure}[ht]

% \begin{subfigure}{.50\textwidth}
%   \centering
%   \includegraphics[width=\linewidth]{images/2d_propagation/zoom_fine_gary_I_d.pdf}
%   \caption{}
%   \label{fig:refined_gary_I_d}
% \end{subfigure}%
% \begin{subfigure}{.50\textwidth}

% \end{subfigure}
%   \caption{(a) Proposed formulation with $I(d) = d$. (b) Zooming in the portion where the crack begins to propagate.  } 
%   \label{fig:prop_gary_zoom}
% \end{figure}

\begin{figure}
  \centering
  \includegraphics[width=0.7\linewidth]{images/2d_propagation/zoom_fine_gary_I_d.pdf}
  \captionof{figure}{Convergence of the proposed formulation with $I(d) = d$.}
  \label{fig:refined_gary_I_d}
\end{figure}

\begin{table}[h]
\centering
\begin{tabular}{ccc}
\hline
$\ell/a$ &Error &$\text{Error}_{k+1}/\text{Error}_{k}$ \\
\hline
1/40  & 0.067 & --\\
1/80  & 0.052 & 0.78\\
1/160 & 0.040 & 0.77\\
1/320 & 0.031 & 0.77\\
\hline
\end{tabular}
\caption{Absolute error in $J$ vs.\ $G_c^{eff}$ for the pressurized crack propagation problem, as a function of regularization length.}
\label{table:convergence_check}
\end{table}

For the case of proposed formulation \eqref{uvc}, the results shown in Figure \ref{fig:prop_gary} indicate a slower convergence towards a 
$J/G^{eff}_c = 1$ response. In contrast with the \eqref{lvc} formulation, the curves converge from above, and therefore, the fracture toughness is slightly overestimated when larger regularization lengths are used. Nevertheless, they all seem to approach a Griffith-like response in the limit $\ell \rightarrow 0$. In Figure \ref{fig:refined_gary_I_d}, an even finer result, using \eqref{uvc} with $I(d)=d$ and $\ell = a/320$ is added, to ensure that the convergence rates indicated in Figure \ref{fig:prop_gary} persist. In Table \ref{table:convergence_check}, the relative errors are provided, indicating a convergence rate of approximately $0.4$ with respect to $\ell$.

One potential explanation for the slower convergence rate is related to the different assumptions regarding the trial cracks, as discussed in Section \ref{sec:model}. Although the different assumptions converge to the same propagation rule in the limit of an infinitesimal crack increment, in the discretized case, the minimal crack increment is finite and related to the mesh spacing $h$ and regularization length $\ell$. In this case, a slightly different propagation behavior, resulting in slower convergence rates towards $J/G^{eff}_c = 1$ is not surprising. 
%In future work, a correction term, dependent on $h$ and $\ell$ will be investigated, in an attempt to improve this convergence rates and allow for better results in coarser meshes.

\section{Summary and conclusions}

This manuscript presents a new approach for developing model-based simulations of hydraulic fracturing. 
The method relies on a global problem that captures flow and deformation fields, and a local problem that captures crack growth with the aid of a phase-field method.  The two problems are coupled through the transfer of displacement and pressure fields, as well as updates to the crack geometry.  This multi-resolution approach allows the phase-field method to be used as a tool to update the crack geometry without the need to explicitly reconstruct the crack aperture from the diffuse representation.  

The accuracy of the multi-resolution approach is evaluated through several numerical examples. These include the well-known KGD problem, for which a good agreement with analytical solutions is obtained in both the toughness and viscosity regimes. A problem with a curved crack trajectory around a stiff inclusion also demonstrates the overall robustness of the approach.  Several areas for future work present themselves. These include, for example, an enhancement of the algorithm to identify the sharp crack tip from the diffuse damage field; treatment of problems involving crack branching and merging; three-dimensional geometries and crack nucleation in arbitrary locations.



\section*{Acknowledgments}

This work was performed under the auspices of the U.S.\ Department of Energy by Lawrence Livermore National Laboratory under Contract DE-AC52-07NA27344. This research was supported by the Exascale Computing Project (17-SC-20-SC), a collaborative effort of the U.S.\ Department of Energy Office of Science and the National Nuclear Security Administration. The support is gratefully acknowledged. 

\section*{Compliance with Ethical Standards}

The authors of this manuscript do not have any financial or other conflicts-of-interest to disclose.  


\bibliographystyle{elsarticle-num} 
\bibliography{ref_clean.bib}

\end{document}