\abstract

This work explores the application of the phase-field method to the simulation of fluid-driven fracture and proposes a framework to make simulations faster and more robust. It begins by addressing a modeling challenge related to the application of pressure loads on the diffuse crack interfaces. During the verification process, a new J-Integral for pressurized fractures in the phase-field context is developed.

Then, the focus turns into a hybrid method to describe fracture propagation. The so-called multi-resolution method uses a combination of enrichment schemes with the phase-field method to address the complex fluid-fracture interaction that happens in hydraulic fracture. On one hand, the phase-field method alleviate some of the difficulties associated with the geometric evolution of the fracture, which are usually the limiting aspect of purely enrichment-based schemes. On the other hand, the discrete representation allows for a better treatment of the fluid loads and crack apertures, which are the main challenges associated with phase-field approaches. 

The multi-resolution method is first presented in a simplified scheme to treat two-dimensional problems, which is used to verify the framework against well-known analytical solutions. It is then extended to three-dimensions. A robust algorithm to handle planar cracks in 3D is developed and its extension to nonplanar cases is discussed. Opportunites for improvements and extensions are then discussed, paving the road for future work in the topic.

% Over the past few decades, the phase-field method for fracture has seen widespread appeal due to the many benefits associated with its ability to regularize a sharp crack geometry. Along the way, several different models for including the effects of pressure loads on the crack faces have been developed. This work investigates the performance of these models and compares them to a relatively new formulation for incorporating crack-face pressure loads. It is shown how the new formulation can be obtained either by modifying the trial space in the traditional variational principle or by postulating a new functional that is dependent on the rates of the primary variables. The key differences between the new formulation and existing models for pressurized cracks in a phase-field setting are highlighted. Model-based simulations developed with discretized versions of the new formulation and existing models are then used to illustrate the advantages and differences. In order to analyze the results, a domain form of the J-Integral is developed for diffuse cracks subjected to pressure loads. Results are presented for a one-dimensional cohesive crack, steady crack growth, and crack nucleation from a pressurized enclosure.

% We present a multi-resolution approach for constructing model-based simulations of hydraulic fracturing, wherein flow through porous media is coupled with fluid-driven fracture. The approach consists of a hybrid scheme that couples a discrete crack representation in a global domain to a phase-field representation in a local subdomain near the crack tip. The multi-resolution approach addresses issues such as the computational expense of accurate hydraulic fracture simulations and the difficulties associated with reconstructing crack apertures from diffuse fracture representations. In the global domain, a coupled system of equations for displacements and pressures is considered. The crack geometry is assumed to be fixed and the displacement field is enriched with discontinuous functions. Around the crack tips in the local subdomains, phase-field sub-problems are instantiated on the fly to propagate fractures in arbitrary, mesh independent directions. The governing equations and fields in the global and local domains are approximated using a combination of finite-volume and finite element discretizations. The efficacy of the method is illustrated through various benchmark problems in hydraulic fracturing, as well as a new study of fluid-driven crack growth around a stiff inclusion.

% We present an extension of the multi-resolution scheme [1] for simulating three-dimensional hydraulic fracture problems. This approach handles crack propagation using the phase-field method in subdomains instantiated on the fly around the crack front. The bulk physics are discretized over a global domain, where the fractures are assumed to be fixed and represented by embedded discontinuities. This separation imparts the approach with the benefits of the phase-field method in handling complex crack paths, without requiring the extraction of crack openings from a diffuse crack surface representation. Challenges that appear when dealing with three-dimensional fracture surfaces are tackled by a new front-tracking algorithm. Large-scale problems are investigated using an efficient implementation of the method in the high-performance GEOSX [2] code.