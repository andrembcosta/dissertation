\section{Notation}

In what follows, deterministic scalar, vectors, second-order tensors, and fourth-order tensors are denoted by $a$ (or $A$), $\bta$ (or $\btA$), $\bfA$, and $\mathbb{A}$, respectively.

Let $\body$ be a collection of points $\btX \in \mathbb{R}^d$, $d \in \{1, 2, 3\}$. Scalar- and vector-valued random fields defined on the probability space $(\Theta, \Sigma, P)$, indexed by $\body$, are denoted as $\{ A(\btX), \btX \in \body \}$ and $\{ \btA(\btX), \btX \in \body \}$, respectively.
At any fixed material point $\btX \in \body$, $a(\btX)$ and $\bta(\btX)$ are random variables defined on the probability space $(\Theta, \Sigma, P)$. For any fixed $\theta \in \Theta$, $a(\theta)$ and $\bta(\theta)$ are realizations of the random variables.
Similarly, $\btX \mapsto a(\btX;\theta)$ and $\btX \mapsto \bta(\btX;\theta)$ are realizations of the random fields $\{ A(\btX), \btX \in \body \}$ and $\{ \btA(\btX), \btX \in \body \}$.

Einstein summations are assumed wherever applicable unless otherwise stated. For any vectors $\bta$ and $\btb$ of the same size, the inner product is defined as $\bta \cdot \btb = a_ib_i$ where $a_i$ and $b_i$ are components of the vectors. The associated vector norm is $\norm{\bta}^2 = \bta \cdot \bta$. Similarly, for any second-order tensors $\bfA$ and $\bfB$ of the same size, the inner product is defined as $\bfA : \bfB = \tr(\bfA^T \bfB)$. The associated Frobenius norm reads $\norm{\bfA} = \sqrt{\bfA : \bfA}$. Other matrix norms will be distinguished by subscripts.
The outer (cross) product of two vectors is written as $\bta \otimes \btb = a_ib_j$. The time derivative is denoted by an over-dot, e.g. $\dot{a}$.

Macaulay brackets are denoted by triangle brackets $\macaulay{a}_\pm$ and are defined as $\macaulay{a}_\pm \equiv (a\pm\abs{a})/2$. Partial derivative is denoted by a subscript starting with a comma, i.e. $ a_{,bc} \equiv \partial^2 a / \partial b \partial c $.
