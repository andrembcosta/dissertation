\abstract

Fracture is a common phenomenon in engineering applications. Many types of fracture exist, including, but not limited to, brittle fracture, quasi-brittle fracture, cohesive fracture, and ductile fracture. Predicting fracture has been one of the most challenging research topics in computational mechanics. The variational treatment of fracture and its associated phase-field regularization have been employed with great success for modeling fracture in brittle materials. Extending the variational statement to describe other types of fracture and multiphysics coupling has proven less straightforward. Main challenges that remain include how to best construct a total potential that is both mathematically sound and physically admissible, and how to properly describe the coupling between fracture and other physics.

The research presented in this dissertation aims at addressing the aforementioned challenges. A variational framework is proposed to describe fracture in general dissipative solids. In essence, the variational statement is extended to account for large deformation kinematics, inelastic deformation, dissipation mechanisms, dynamic effects, and thermal effects. The proposed variational framework is shown to be consistent with conservations and laws of thermodynamics, and it provides guidance and imposes restrictions in construction of multiphysics models. Within the proposed variational framework, several models are instantiated to address practical engineering problems. A brittle and quasi-brittle fracture model is used to investigate fracture evolution in polycrystalline materials; A cohesive fracture model is applied to revisit soil dessication; A novel ductile fracture model is proposed and successfully applied to simulate some challenging benchmark problems; A creep fracture model is developed to simulate spallation of oxide scale on high temperature heat exchangers.
