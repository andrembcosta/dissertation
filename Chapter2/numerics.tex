\section{Finite Element Implementation}
\label{sec:fem_implementation}

In this Section, the details of the finite element discretization used to obtain approximations to the solution of the proposed model \eqref{uvc} are described. For analogous equations for the model \eqref{lvc}, the reader is referred to \cite{jiang2022phase}.
%The method of choice is the finite element method, as in most of the works regarding the phase-field approach for fracture. 
First, the strong form of the governing equations, derived from the general free-energy \eqref{PF general form} using the KKT\cite{karush1939minima, kuhn1951nonlinear} conditions is presented.

% To obtain approximate solutions to the proposed model, we will use the finite element method. In order to lay out the equations for the approximate solution, let's first start by re-writing the strong form of the problem with our choice of free-energy (\ref{PF general form}). 

\begin{mdframed}[
  frametitle={Strong form},
  frametitlebackgroundcolor=gray!20,
  backgroundcolor=gray!5,
  linewidth=0pt,
  nobreak=true
  ]
  \vspace{-1em}
  %\small{
  \begin{align}
    &\text{Linear momentum balance: }\nonumber \\ & \nabla \cdot \bs\sigma -p\nabla d + \textbf{b} = {\color{red} \bs 0},                       \hspace{2cm} \forall x \in \Omega, \\ 
                                     & \bs\sigma = \frac{\partial\psi_e}{\partial \bs\epsilon},   \hspace{3.8cm}                 \forall x \in \Omega, \\
                                     & \bs\sigma\cdot \textbf{n} = \textbf{t},                             \hspace{4.2cm}              \partial_t\Omega,  
                                     \\
                                     & \textbf{u} = \textbf{u}_g,                                                  \hspace{4.4cm}     \partial_u\Omega ,                \\
    &\text{Fracture evolution: }\nonumber \\      
    & \dot d\left(\nabla\cdot\frac{2G_c\ell}{c_0}\nabla d - \frac{G_c}{c_0\ell}\alpha'(d) - g'(d)\psi_e^+ (\bs\epsilon)\right) = 0,                                        \forall x \in \Omega,           \\
    & \nabla \cdot \frac{2G_c\ell}{c_0}\nabla d - \frac{G_c}{c_0\ell}\alpha'(d) - g'(d)\psi_e^+ (\bs\epsilon) \le 0,                                         \forall x \in \Omega,  \label{damage ineq}         \\
                                     & \dot d \ge  0,                                     \hspace{4.4cm}     \forall x \in \Omega, \\
                                     & \nabla d\cdot\textbf{n} = 0,                      \hspace{3.8cm}       \partial\Omega, \\
                                         & d(0,\textbf{x}) = d_0,                         \hspace{3.9cm}           \Omega.                                  
    \end{align}
    %}
\end{mdframed}

% \begin{mdframed}[
%   frametitle={Strong form},
%   frametitlebackgroundcolor=gray!20,
%   backgroundcolor=gray!5,
%   linewidth=0pt,
%   nobreak=true
%   ]
%   \vspace{-1em}
%   \begin{align}
%     \text{Linear momentum balance: } & \nabla \cdot \bs\sigma -p\nabla d + \textbf{b} = 0,                       &  & \forall x \in \Omega,                          \\
%                                      & \bs\sigma = \frac{\partial\psi_e}{\partial \bs\epsilon},                   &  & \forall x \in \Omega, \\
%                                      & \bs\sigma\cdot \textbf{n} = \textbf{t},                                          &  & \partial_t\Omega,  
%                                      \\
%                                      & \textbf{u} = \textbf{u}_g,                                                      &  & \partial_u\Omega ,                \\
%     \text{Fracture evolution: }      & \dot d\left(\nabla\cdot\frac{2G_c\ell}{c_0}\nabla d - \frac{G_c}{c_0\ell}\alpha'(d) - g'(d)\psi_e^+ (\bs\epsilon)\right) = 0,                                       &  & \forall x \in \Omega,           \\
%     & \nabla \cdot \frac{2G_c\ell}{c_0}\nabla d - \frac{G_c}{c_0\ell}\alpha'(d) - g'(d)\psi_e^+ (\bs\epsilon) \le 0,                                       &  & \forall x \in \Omega,  \label{damage ineq}         \\
%                                      & \dot d \ge  0,                                         &  & \forall x \in \Omega, \\
%                                      & \nabla d\cdot\textbf{n}_0 = 0,                                         &  & \partial\Omega, \\
%                                          & d(0,\textbf{x}) = d_0,                                         &  & \Omega.                                  
%     \end{align}
% \end{mdframed}

For the derivation of an equivalent weak form, trial spaces for $\textbf{u}$ and $d$ are first defined.  Although the derivation is confined to quasi-static loadings, the spaces are indexed by a discrete load step parameter $t$.   The trial spaces are given by

  \begin{align}
    \boldsymbol{\mathcal{U}}_t & = \{ \textbf{u} \in \mathcal{H}^1(\Omega)^d \mid \textbf{u} = \overline{\textbf{u}}_t \text{ on } \partial_u\Omega \}, \\
    \mathcal{D}_t      & = \{ d \in \mathcal{H}^1(\Omega) \mid d_{t-1}(x) \le d_t(x) \le 1,\ \forall x \in \Omega \}, 
  \end{align}

\noindent and the accompanying weighting spaces $\boldsymbol{\mathcal{V}}$ and $\mathcal{C}$ are

  \begin{align}
    \boldsymbol{\mathcal{V}} & = \{ \textbf{w} \in \mathcal{H}^1(\Omega)^d \mid \textbf{w} = \boldsymbol{0} \text{ on } \partial_u\Omega \}, \\
    \mathcal{C}      & = \{ c \in \mathcal{H}^1(\Omega) \mid c(x) \ge 0,\ \forall x \in \Omega \}.                                                     
  \end{align}

The condition of monotonicity in the space $\mathcal{D}_t$ is used to prevent damage healing and is the weak enforcement of the condition $\dot d \ge 0$, in a time discrete setting. Denoting the inner product in $\mathcal{H}^1(\Omega)$ and $\mathcal{H}^1(\Omega)^d$ by $\left( \cdot, \cdot \right)$ and it's restriction in the boundary by $\left<\cdot,\cdot\right>$, the weak form of the problem can be written as

% \footnote{ In this weak form, the inequality governing the damage evolution is written as an equality for simplicity. That is because in the solution strategy it is indeed treated as an equality but only over the active set, namely the set of points where $\dot d > 0$.}

\begin{mdframed}[
    frametitle={Weak form},
    frametitlebackgroundcolor=gray!20,
    backgroundcolor=gray!5,
    linewidth=0pt,
    nobreak=true
  ]
  Find $\textbf{u} \in \boldsymbol{\mathcal{U}}_t$ and $d \in \mathcal{D}_t$, such that $\forall \textbf{w} \in \boldsymbol{\mathcal{V}}$ and $\forall c \in \mathcal{C}$,
  
    \begin{align}
      \left( \nabla \textbf{w}, \bs\sigma \right) - \left( \textbf{w}, p\nabla d \right) - \left( \textbf{w}, \textbf{b} \right) - \left< \textbf{w}, \textbf{t} \right>_{\partial_t\Omega} & = 0, \\
      \frac{2\ell}{c_0}\left( \nabla c, G_c\nabla d \right) + \frac{1}{c_0\ell}\left( c, G_c\alpha'(d) \right) + \left( c, g'(d)\psi_e^+ (\bs\epsilon(\textbf{u})) \right) & = 0,\label{weak damage equation}
    \end{align}
 
\noindent with the initial damage condition,
 
    \begin{align}
      \left( c, d(0,\textbf{x}) - d_0 \right)                & = 0.
    \end{align}
 
\end{mdframed}

Observe that \eqref{weak damage equation} is an equality rather than an inequality, such as \eqref{damage ineq}.  This reflects a view ahead, toward discretization, where in the present work  the irreversibility constraint is enforced with an active-set strategy.  The active set strategy effectively partitions the domain into active (where $\dot{d}=0$) and inactive (where $\dot{d}>0$) parts. Only the inactive part requires a discretization of the damage condition \eqref{damage ineq}, where it is indeed treated as an equality. A detailed description of this constrained optimization algorithm is given by Heister et al. in \cite{heister2015primal}, and some additional details pertinent to phase-field for fracture discretizations can be found in Hu et al. \cite{hu2020phase}.

Finally, these function spaces can be discretized over a finite element mesh, that give rise to the discrete function spaces $\boldsymbol{\mathcal{U}}^h_t \subset \boldsymbol{\mathcal{U}}_t$, $\boldsymbol{\mathcal{V}}^h \subset \boldsymbol{\mathcal{V}}$, $\mathcal{D}^h_t \subset \mathcal{D}_t$, $\mathcal{C}^h \subset \mathcal{C}$. These are then used to construct the discrete form of the problem using the Galerkin method:

\begin{mdframed}[
    frametitle={Spatially discretized form},
    frametitlebackgroundcolor=gray!20,
    backgroundcolor=gray!5,
    linewidth=0pt,
    nobreak=true
  ]
  Find $\textbf{u}^h \in \boldsymbol{\mathcal{U}}^h_t$ and $d^h \in \mathcal{D}^h_t$, such that $\forall \textbf{w}^h \in \boldsymbol{\mathcal{V}}^h$ and $\forall q^h \in \mathcal{C}^h$,
  \begin{subequations}
    \begin{align}
      \left( \nabla \textbf{w}^h, \bs\sigma^h \right) - \left( \textbf{w}^h, p\nabla d^h \right) - \left( \textbf{w}^h, \textbf{b} \right) - \left< \textbf{w}^h, \textbf{t} \right>_{\partial_t\Omega} & = 0, \label{eq: semidiscrete momentum balance ch2}   \\
      \frac{2\ell}{c_0}\left( \nabla c^h, G_c\nabla d^h \right) + \frac{1}{c_0\ell}\left( c^h, G_c\alpha'(d^h) \right) + \left( c^h, g'(d^h)\psi_e^+ (\bs\epsilon(\textbf{u}^h)) \right) & = 0, \label{eq: semidiscrete fracture evolution}
    \end{align}
  \end{subequations}
  with the initial damage condition,
  \begin{subequations}
    \begin{align}
      \left( c^h, d^h(0,\textbf{x}) - d_0 \right)                & = 0.
    \end{align}
  \end{subequations}
\end{mdframed}

In this work, bilinear finite elements are used to approximate the damage and displacement fields. 
% The crack irreversibility condition is enforced with a primal-dual active set (PDAS) strategy. 
% A detailed description of the PDAS is given by Heister et al in \cite{heister2015primal}, and some additional details pertinent to phase-field for fracture discretizations can be found in Hu et al \cite{hu2020phase}.
The coupling between the two discrete equations \ref{eq: semidiscrete momentum balance ch2} and \ref{eq: semidiscrete fracture evolution} is handled by an alternating minimization scheme. A detailed description of this scheme is given in \cite{hu2020phase}.  {\color{blue} This solution scheme is implemented using RACCOON \cite{raccoon}, a massively parallel finite element code specializing in phase-field fracture problems. RACCOON is built upon the MOOSE framework \cite{gaston2009moose, permann2020moose} developed at the Idaho National Laboratory.}

