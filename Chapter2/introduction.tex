\section{Introduction}
\label{sec:introduction}

The propagation of pressurized fractures is a physical phenomena of interest or concern in many different fields of engineering. Some examples include hydraulic fracture (fracking) treatments in the oil and gas industry \cite{li2015review, mair2012shale}, pressure vessel rupture \cite{shinmura1997fluid}, fracture in concrete dams \cite{wang2017experimental} and fuel fracture in nuclear reactors \cite{capps2021critical, turnbull2015assessment}. Therefore, predictive simulation tools for this phenomena have been intensively studied in recent years. One of these tools is the phase-field method for fracture \cite{bourdin2000numerical}. Initially developed for traction-free cracks, the method has since been extended to account for pressure loading on the surfaces of cracks, as in \cite{bourdin2012variational, wheeler2014augmented, mikelic2015quasi, peco2017influence, jiang2022phase}.  These various formulations exhibit real differences in terms of their structure and form when it comes to how the pressure loads are incorporated. The objective of this work is to examine the impact of the various choices, and to compare them to 
a relatively new formulation for pressurized crack surfaces in a phase-field for fracture context \cite{hu2021variationalthesis}.  

The main contributions of this work are: (a) to show that established formulations for pressure-driven fracture in the phase-field
context have limitations when cohesive processes are involved; (b) to demonstrate that the new formulation, derived from variational principles, can address these limitations and be easily combined with phase-field models of cohesive fracture; and (c) to illustrate the advantages and disadvantages of the various models in terms of accuracy in obtaining various quantities of interest.  

Phase-field methods for fracture regularize sharp crack representations through the use of a scalar phase or damage field whose evolution is governed by minimization principles.  
Such methods first appeared, in different forms, in the works of Bourdin et al. \cite{bourdin2000numerical} and Karma et al. \cite{karma2001phase}. The model introduced in Bourdin et al. \cite{bourdin2000numerical} was obtained by a regularization of the variational formulation of fracture developed in Francfort and Marigo \cite{francfort1998revisiting}, using ideas from Ambrosio and Tortorelli \cite{ambrosio1990approximation}. It has been widely adopted in the mechanics community and extended for use in a variety of fracture mechanics problems,  such as ductile failure \cite{alessi2014gradient, ambati2015phase, miehe2016phase, borden2016phase, hu2021variationalpaper}, hydraulic fracture \cite{wilson2016phase, chukwudozie2019variational, mikelic2015phase1, santillan2018phase, miehe2016phase}, dessication problems \cite{maurini2013crack, heider2020phase, cajuhi2018phase, hu2020frictionless}, dynamic fracture\cite{bourdin2011time, borden2012phase, hofacker2013phase, schluter2014phase, li2016gradient, kamensky2018hyperbolic, moutsanidis2018hyperbolic}, fracture in biomaterials \cite{wu2020fracture, raina2016phase, nagaraja2021phase, gultekin2016phase, gultekin2018numerical} and many more. Some recent reviews can be found in \cite{ambati2015review, wu2020phase, francfort2021variational}.

With regard to the use of the phase-field method for hydraulic fracture problems, one challenge concerns how best to incorporate surface loads that result from pressures on crack faces that are diffuse. One approach is to regularize the resulting surface tractions with an approach that is very similar to how the crack surface energy is regularized. Early work along these lines focused on crack surfaces loaded by constant pressures, as in Bourdin et al. \cite{bourdin2012variational} and Wheeler et al.\cite{wheeler2014augmented}. Since these early developments, these models have been extensively used since then for the study of pressurized fractures, for example in \cite{tanne2022loss, zulian2021large, yoshioka2019comparative, yoshioka2020crack}. They were also extended and modified to account for fluid flow inside the fractures and poroelasticity in the surrounding medium \cite{miehe2016phase, mikelic2015phase1, chukwudozie2019variational, wilson2016phase, santillan2018phase, heider2017phase, li2022hydro}. The reader is referred to the recent review by  Heider \cite{heider2021review} for additional works on phase-field methods for hydraulic fracture.  The various models all employ some form of ``indicator function" that assists in the regularization of the surface load itself.  Despite several different indicator functions being proposed, the implication of the particular choice of indicator on the accuracy of the models has yet to be thoroughly examined.  
 
In this manuscript, a new formulation for the study of pressurized fractures, first proposed in the thesis of Hu~\cite{hu2021variationalthesis} is also examined. In particular, it is studied in combination with a cohesive version of the phase-field for fracture method, which was proposed in the recent works of \cite{lorentz2011convergence, geelen2019phase, wu2017unified}.  This facilitates the study of pressurized fracture in quasi-brittle materials and reduces the sensitivity of the effective strength to the regularization length. To ensure that the cohesive fracture behavior is preserved, the implicit traction-separation law is evaluated for a simple one-dimensional problem and shown to be insensitive to the applied pressure with the new formulation. Fracture initiation and propagation examples are also examined to highlight advantages and limitations of the model. 

As part of the analysis conducted to evaluate the various formulations, the J-integral is used to verify the extent to which mode-I crack propagation occurs when the energy release rate reaches the critical fracture energy.  
The contour form of the J-integral and its modifications for some common cases of phase-field fracture has been examined by others, see e.g.\ the work of \cite{sicsic2013gradient}, \cite{ballarini2016closed} and \cite{hossain2014effective}.  In the case of pressurized cracks, the contour version of the J-integral is not path independent. Many prior works have focused on developing domain forms of the J-integral for sharp cracks that are domain independent \cite{li1985comparison, shih1986energy}.  In this work, a domain form of the J-integral that is suitable for pressurized phase-field cracks is developed for the first time.  

The paper is organized as follows. In Section \ref{sec:model}, a simple model for pressure-induced fracturing is presented and the new phase-field formulation is derived in two different ways.  Section \ref{sec:j_integral} provides the derivation of the domain form of the J-Integral for pressurized phase-field cracks. In Section \ref{sec:fem_implementation}, the discretization scheme using finite elements is presented. Then, in Section \ref{sec:results} some fundamental examples involving crack nucleation and propagation are used to illustrate the performance of the various models and choices of indicator functions. Finally, some concluding remarks and directions for future work are discussed in the last section.

