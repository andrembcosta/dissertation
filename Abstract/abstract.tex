\abstract

Example text

The research presented in this dissertation aims at addressing
the aforementioned challenges. A variational framework is proposed 
to describe fracture in general dissipative solids. In essence, 
the variational statement is extended to account for large deformation
kinematics, inelastic deformation, dissipation mechanisms, dynamic 
effects, and thermal effects. The proposed variational framework
is shown to be consistent with conservations and laws of 
thermodynamics, and it provides guidance and imposes restrictions
on the construction of models for coupled field problems. Within 
the proposed variational framework, several models are instantiated 
to address practical engineering problems. A brittle and quasi-brittle
fracture model is used to investigate fracture evolution in 
polycrystalline materials; a cohesive fracture model is applied 
to revisit soil desiccation; a novel ductile fracture model is 
proposed and successfully applied to simulate some challenging 
benchmark problems; and a creep fracture model is developed to
simulate the spallation of oxide scale on high temperature heat
exchangers.
