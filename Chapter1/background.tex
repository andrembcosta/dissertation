\section{Background}

%% Motivation for fluid-driven fracture

    % Shale gas extraction

    % EGS

    % Carbon storage

%% Literature review on traditional approaches (including 3D)

    % Detournay's analytical methods

    % Armando's work

    % Imperial College group

    % Chinese group

%% Phase-field method for hydraulic fracture

    % A few paragraphs about the phase-field method and its advantages

    % Bourdin and Wheeler's early work

    % Other derived works coupling poromechanics

    % Thorough comparision with tables and etc

    % Some limitations and possible improvements

%% Hybrid approaches

    % Giovannardi including newest work from Formmagia on HF

    % Rudy's continuous discontinuous

    % Muixí's work

    % Recent work from chinese group

%% Commercial codes and intro on GEOSX

    % Check review paper from Bin Chen

    % Is there any review on HPC, GPU computing ?

%% Fracture mechanics basics
In this dissertation, four types of fracture are of interest: brittle fracture, quasi-brittle fracture, cohesive fracture, and ductile fracture. Brittle and quasi-brittle fracture center around the Griffith-type description of fracture, in the sense that the stress field at the crack tip is singular. In particular, brittle fracture is characterized with the complete loss of tensile strength upon cracsk nucleation. Under displacement-controlled uni-axial tension, a brittle specimen forms a through crack immediately after crack nucleation. Quasi-brittle fracture, in contrast, shows a gradual decay of tensile stress which is often measured in the form of a traction-separation law. Although the celebrated Griffith theory of fracture captures the behavior of a wide range of brittle and quasi-brittle materials, it bears emphasis that most materials are not perfectly brittle in the Griffith sense. Cohesive fracture, for example, describes materials that possess a \emph{fracture process zone} ahead of the crack tip. The fracture process zone (FPZ) is a lumped description of small-scale yielding, void coalescence, and micro-crack merging. Typically, the material inside the FPZ only partially loses its load-carrying capacity, and therefore cohesive fracture also displays a gradual decay of tensile stress after crack initiation. The boundary between the definitions of quasi-brittle fracture and cohesive fracture is inherently vague. In this dissertation, cohesive fracture is assumed to have a substantially larger fracture process zone than quasi-brittle fracture, and at the engineering scale, numerical models for these two types of fracture are often used interchangeably. Lastly, ductile fracture describes the softening due to fracture after a stage of plastic hardening. Most metals and fiber-reinforced cementitious composites are ductile and are subject to ductile failure,

%% Briefly mention other methods to model fracture
A number of models have been proposed to model fracture. What follows is a brief, non-comprehensive survey of numerical methods to model fracture. Particle-based and mesh-free discretization techniques, such as the discrete element method (DEM), lattice spring methods, and peridynamics, have been developed for fracture simulation. The key advantage of these methods lies in the fact that cracks can nucleate and grow in an unguided manner. However, the computational cost of these methods can be very high and the interface conditions are not straightforward to apply. Within the context of the finite element method (FEM), the most commonly used element techniques to model cracks are the extended finite element method (XFEM) and the cohesive zone method (CZM). Using XFEM, fracture are represented as discrete discontinuities and the solution space with discontinuities are enriched using the partition of unity method \cite{babuaka1997, Dolbow99}. Singularities associated with fracture can be represented using XFEM enrichment at relatively low cost. However, managing the data structure and tracking the evolution of complex fracture surfaces, particularly in 3D, are quite tedious. Using CZM, cracks are modeled by cohesive elements inserted between element boundaries, and softening is governed by a traction-separation law at crack surfaces \cite{needleman_1992, ortiz_1999}. For polycrystalline material, it is convenient to use CZM to model intergranular crack initiation and propagation \cite{KAMAYA2007, KAMAYA2009}. Nonetheless, CZM suffers from mesh dependency because the representation of crack propagation is restricted to element boundaries.

%% Origin of phase-field fracture models and how it works
The variational approach to fracture was proposed by \citet{Francfort98}, and the phase-field implementation is attributed to \citet{Bourdin2000}.
\citet{bourdin2008variational} provides an overview.
\citet{pham2013onset} used a crack geometric function that is linear in the phase-field variable, resulting in a purely elastic response prior to crack initiation.  This fracture energy functional was subsequently referred to as the \texttt{AT-1} functional by Bourdin and co-workers (meant to evoke \textit{Ambrosio} and \textit{Tortorelli}, who initially introduced this and related functionals in the context of image segmentation \cite{ambrosio1990approximation}). \citet{wu2017unified} generalized the \textit{Ambrosio-Tortorelli} functional \cite{ambrosio1990approximation} to a family of second-order polynomials with various phase-field profiles, with specific choices of recovering the widely adopted \texttt{AT-1} and \texttt{AT-2} models.
In the variational approach, crack surfaces are represented by a surface density function in terms of an auxiliary phase field. This naturally gives rise to a regularization involving a length scale parameter. Recent works have illustrated the potential of phase-field models to be predictive for a wide range of fracture problems. Phase-field models for fracture have succeeded in capturing complex crack patterns, including branching and merging in both two and three dimensions \cite{karma_2001, karma_2004, henry_2004, spatschek_2007, amor_2009}.

%% Seminal works on the variational framework
The general variational framework was first proposed by \citet{biot1956thermoelasticity} in the context of thermoelasticity, followed by extensive investigations in coupled thermoelastic and thermoviscoelastic systems \cite{herrmann1963variational,ben1965variational,oden2012variational,molinari1987global,batra1989principle,matsubara2021variationally}. Well-defined variational principles for dissipative solids with and without heat conduction also exist \cite{ortiz_1999,yang2006variational}. However, the case of thermal-mechanical-fracture coupling in dissipative materials has received far less attention.

%% What variational really means
It is worth noting that there are fundamental differences between variational consistency and thermodynamic consistency. In fact, thermodynamically consistent models are not necessarily variational (and vice versa). To satisfy thermodynamics, the model \emph{must} satisfy mass conservation, linear momentum conservation, angular momentum conservation, as well as the first and the second laws of thermodynamics. Violating any of the aforementioned conservations or thermodynamic laws results in a physically inadmissible model. However, since the second law of thermodynamics is an inequality statement, there are infinitely many ways to satisfy it when modeling a dissipative solid. In contrast, the variational approach (including, but not limited to, variational approaches to fracture) is based on the idea that, upon careful construction of a potential, its variational statements are thermodynamically consistent, implying that the dissipation attains its maximum at the solution corresponding to the variational statements. In other words, the variational approach sets a more restrictive stage for model construction because it places more constraints on the constitutive behavior of the material. Most notably, if a potential is constructed within the variational framework to account for thermomechanical coupling, the heat generation due to its dissipation is directly predicted by the variational framework and cannot be specified arbitrarily.

%% Benefits of a variational model
The restrictions imposed by the variational approach lead to several beneficial aspects. First, the variational framework enables application of the tools of calculus of variations to the analysis of the solution. In particular, the direct method of calculus of variations informs conditions for the existence and uniqueness of solutions (e.g. \cite{dal2012introduction}). Second, the localization of the fracture and the plastic zone can be studied within the framework of free-discontinuity problems \cite{braides1998approximation,gariepy2001functions}. Third, in numerical methods, discretization of the variational statement of the problem leads to a symmetric operator, which can save storage and potentially accelerate the assembly process. Furthermore, energy-based line search methods can be directly applied to the system of equations; many powerful numerical optimization algorithms within standard linear algebra packages (e.g. PETSc \cite{petsc-web-page}, TAO \cite{benson2003tao}, Trilinos \cite{heroux2005overview}, and Matlab \cite{higham2016matlab}) can be utilized out-of-the-box; and the time-discretized variational problem leads to robust and efficient variational constitutive update algorithms \cite{ortiz_1999}.
