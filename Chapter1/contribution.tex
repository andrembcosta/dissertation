\section{Research contributions}

%% summary
This dissertation revisits the computational modeling of hydraulic fracturing with emphasis on the phase-field method for fracture, and proposes a new hybrid scheme that overcomes the two main limitations of pure phase-field formulations.
In particular, it provides a means to calculate the fracture aperture while simultaneously limiting the expense of the phase-field method to a small region in the vicinity of the crack front.  
%Namely, the computation of the fracture aperture and the expensive computational cost. 
This new scheme, when compared to standard discrete fracture models, removes the need for a phenomenological criterion for crack propagation by relying on the minimization of an energy functional to drive fracture growth.
Fundamental modeling aspects, related to the phase-field description of fluid-driven fracture are also discussed. The main contributions of each chapter are summarized below.

\begin{itemize}
  \item %% contributions in chapter 2: Revised formulation of pressurized cracks
        In \Cref{section: Chapter2}, we present a new phase-field formulation for including crack-face pressure loads and discuss how this formulation can be achieved by modifying the trial space in the traditional variational principle or by introducing a new functional that depends on the rates of the primary variables. The key differences between the new formulation and existing models for pressurized cracks in a phase-field setting are emphasized. Model-based simulations developed with discretized versions of the new formulation and existing models are then used to illustrate the advantages and differences. In order to analyze the results, a domain form of the J-Integral is developed for diffuse cracks subjected to pressure loads.
  \item %% contributions in chapter 3: Multi-resolution approach in 2D
        In \Cref{section: Chapter3}, we propose a multi-resolution approach for constructing model-based simulations of fluid-driven fracture. The approach consists of a hybrid scheme that couples a discrete crack representation in a global domain to a phase-field representation in a local subdomain near the crack tip. The multi-resolution approach addresses issues such as the computational expense of accurate hydraulic fracture simulations and the difficulties associated with reconstructing crack apertures from diffuse fracture representations. The efficacy of the method is illustrated through various benchmark problems in hydraulic fracturing, as well as a new study of fluid-driven crack growth around a stiff inclusion.
  \item %% contributions in chapter 4: Extension to 3D planar or possibly non-planars
        In \Cref{section: Chapter4}, we present an extension of the multi-resolution approach to three dimensions, which involves a modification to the propagation scheme used in \Cref{section: Chapter3}. First the extension to planar cracks in 3D is explored. The new algorithm is detailed and an efficient implementation in the HPC solver GEOS is developed. Some preliminary results are presented, involving multiple cracks are presented. Finally, the extension to non-planar fracture geometries is discussed, and an algorithm with further modifications is presented. A version is then implemented and tested for a very simple hydraulic fracture problem. Further steps are discussed, building on the main limitations observed. 

\end{itemize}
