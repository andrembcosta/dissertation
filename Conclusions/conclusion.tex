\chapter{Conclusions and Future Work}
\label{section: Conclusions}

In this dissertation, methods for computer simulation of fluid-driven fracture are studied. In particular, the phase-field approach, whose application in this context is still in its infacy, is explored as a technology to improve the accuracy and robustness of models. Some known challenges associated with the use of the phase-field, namely the application of tractions on the diffuse surfaces and the computation of fracture apertures are circunvented by the novel contributions of this work. 

Chapter \ref{section: Chapter2} reviews various models for incorporating pressure loads on diffuse crack faces. This includes the analysis of a new formulation that can be obtained by assuming that the pressure load does not act on the virtual extension of a crack. The new formulation is referred to as the 
 ``unloaded virtual crack formulation"\eqref{uvc}. A new form of the J-Integral for pressurized cracks in the phase-field context is also derived. This J-Integral proves to be particularly useful in verifying the accuracy of the various models for propagating cracks,

The new \eqref{uvc} formulation allows for an unified treatment of crack nucleation and propagation in scenarios involving either brittle or cohesive fracture, and provides for better accuracy in some problems compared to existing approaches. As it allows for the use of the same governing equation for the damage parameter, its computational implementation within existing phase-field solvers is also simpler. 

Chapter \ref{section: Chapter3} presents a new approach for simulating fluid-driven fracture. The method relies on a global problem that captures flow and deformation fields, and a local problem that captures crack growth with the aid of a phase-field method. The two problems are coupled through the transfer of displacement and pressure fields, as well as updates to the crack geometry. This multi-resolution approach allows the phase-field method to be used as a tool to update the crack geometry without the need to explicitly reconstruct the crack aperture from the diffuse representation.  

The accuracy of the multi-resolution approach is evaluated through several numerical examples. These include the well-known KGD problem, for which a good agreement with analytical solutions is obtained in both the toughness and viscosity regimes. A problem with a curved crack trajectory around a stiff inclusion also demonstrates the overall robustness of the approach.  

Finally, Chapter \ref{section: Chapter4} extends the multi-resolution method proposed in Chapter \ref{section: Chapter3} to three dimensions. It begins with the simpler, yet useful case of planar fractures. A modified algorith is presented and its implementation in the HPC code GEOS is developed. Representative examples are shown to highlight the capabilities of the model and also the robustness of the code. Then, the case of non-planar fractures is investigated. The previous algorithm for planar fractures in 3D is revisited and several modifications are provided to allow for more complex geometry. A simplified implementation is then tested in a 2.5D problem, helping to identify the main challenges and paving the way for future work in the topic.

Multiple areas for future work naturally present themselves throughout the dissertation and can lead to new research topics in this field.

\begin{itemize}
    \item The \eqref{uvc} formulation from Chapter \ref{section: Chapter2} is shown to be capable of predicting fracture nucleation. However, it suffers from the well-known issues of energectic phase-field models which can not represent the true strength-based resistance of fracture of real-world materials. Therefore, incorporating such model in a strength-based formulation such as \cite{kumar2020revisiting} is key to improve simulations of fracture nucleation. 
    
    \item Also in Chapter \ref{section: Chapter2}, modifications to accelerate the convergence of the model with respect to the phase-field parameter $\ell$ will also important, given that the current form requires very fine grids to achieve accurate results.
    
    \item For the multi-resolution approach, the directions of future work can be split in two parts. The first concerns the robustness of the algorithm developed for planar fractures, with the aim to turn it into a practical tool for engineering simulation. This line of work includes a revision of the algorithm to handle unstructured grids and additional physics. It also concerns the numerical solver, which would benefit a lot from a physics-informed linear solver as well as from a stabilization procedure. Additional verification examples, such as the ones shown in Chapter \ref{section: Chapter3} are also needed.
    
    \item The second direction is the treatment of non-planar fractures. For this task, the main challenge appears to be the identification of fractured elements from the damage field. Based on the discussion and preliminary results shown in Chapter \ref{section: Chapter4}, global algorithms to reconstruct are sharp geometry from the phase-field are likely to help considerably in this task. They will also help handling the issue of gaps that can appear when stitiching new fracture cells to the current discrete crack. 
\end{itemize}

% Chapter2 future work
% In future work, its applicability to problems involving plastic deformation and strength-based fracture nucleation will be studied. In addition to that, 

% \section{Acknowledgments}

% %The authors would like to acknowledge the support and funding from Duke University and Argonne National Laboratory, which was essential to the development of this work.  
% The partial support of A.\ Costa and  J.E.\ Dolbow by the National Science Foundation, through grant CMMMI-1933367 to Duke University, is gratefully acknowledged.  Author T.\ Hu gratefully acknowledges the support of Argonne National Laboratory.  Argonne National Laboratory is managed and operated by UChicago Argonne LLC.\ for the U.S.\ Department of Energy under Contract No.\ DE-AC02-06CH11357.

%Chapter3 future work

% Several areas for future work present themselves. These include, for example, an enhancement of the algorithm to identify the sharp crack tip from the diffuse damage field; treatment of problems involving crack branching and merging; three-dimensional geometries and crack nucleation in arbitrary locations.